% %% %%%%%%%%%%%%%%%%%%%%%%%%%%%%%%%%%%%%%%%%%%%%%%%%%%%%%%%%%%
% material.tex
%
% Author:  Mauricio Matamoros
% License: MIT
%
% %% %%%%%%%%%%%%%%%%%%%%%%%%%%%%%%%%%%%%%%%%%%%%%%%%%%%%%%%%%%

%!TEX root = ../practica.tex
%!TEX root = ../references.bib

% CHKTEX-FILE 1
% CHKTEX-FILE 13
% CHKTEX-FILE 46

\section{Material}%
\label{sec:material}
Se asume que el alumno cuenta con un una Raspberry Pi con sistema operativo Raspbian e interprete de Python instalado.
Se aconseja encarecidamente el uso de \textit{git} como programa de control de versiones.

\begin{itemize}[noitemsep]
	\item 1 microcontrolador RP2040 (Raspberry Pico) con firmware MicroPython precargado.
	\item 1 sensor de temperatura LM35 en encapsulado TO-220 o TO-92
	\item 2 Diodos 1N914
	\item 2 resistencia de 10k$\Omega$
	\item 1 resistencia de 12k$\Omega$\footnotemark
	\item 1 resistencia de 18k$\Omega$
	\item 1 Condensador de 0.1$\mu$F
	\item 1 protoboard o circuito impreso equivalente
	\item 1 fuente de alimentación regulada a 5V y al menos 2 amperios de salida
	\item 1 cable micro-USB/USB-C para programar el RP2040
	\item Cables y conectores varios
\end{itemize}
\footnotetext{La resistencia de 12k$\Omega$ puede reemplazarse con resistencias de 13k$\Omega$ a 20k$\Omega$ dependiendo del voltaje de los diodos.} %chktex 42
