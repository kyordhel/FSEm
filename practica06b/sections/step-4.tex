% %% %%%%%%%%%%%%%%%%%%%%%%%%%%%%%%%%%%%%%%%%%%%%%%%%%%%%%%%%%%
% step-4.tex
%
% Author:  Mauricio Matamoros
% License: MIT
%
% %% %%%%%%%%%%%%%%%%%%%%%%%%%%%%%%%%%%%%%%%%%%%%%%%%%%%%%%%%%%

%!TEX root = ../practica.tex
%!TEX root = ../references.bib

% CHKTEX-FILE 1
% CHKTEX-FILE 13
% CHKTEX-FILE 46

\subsection{Paso 4: Bitácora de temperatura via \IIC}%
\label{sec:step4}

Con la Raspberry Pi configurada, basta con generar los dos programas para transferir las temperaturas registradas en el RP2040 a la Raspberry Pi que se encargará de almacenar esta información en un archivo o bitácora.

Primero, es necesario configurar al RP2040 como dispositivo esclavo e inicializar el bus \IIC, tal como se muestra en los \Cref{lst:pico-code-i2c-def,lst:pico-code-i2c-setup}.

Python es un lenguaje MUY lento que carece de tipos nativos primitivos y por ende hace un uso excesivo del \emph{heap}, lo que hace imposible gestionar interrupciones de forma rápida y segura.
Por este motivo las comunicaciones via \IIC son \textbf{\textsc{síncronas}} y requieren verificar el estado de los registros internos de forma contínua en un loop.

Además, MicroPython sólo ofrece soporte para utilizar los módulos \IIC en configuración maestro.
Luego entonces es necesario realizar la configuración y la lectura y escritura de datos a nivel registro, mecanismo que se conoce como envoltorio o \emph{wrapping}.
Dicho código se provee por conveniencia en el archivo \code{i2cslave.py}.

\lstinputlisting[%
	language=Python,
	caption={\texttt{pico-code-i2c.py:14--19} --- Dirección asignada al dispositivo esclavo},
	label={lst:pico-code-i2c-def},
	linerange={14-14,18-19},%CHKTEX 8
	]{src/pico-code-i2c.py}

\lstinputlisting[%
	language=Python,
	caption={\texttt{pico-code-i2c.py:51--56} --- Configuración del bus \IIC y funciones de control},
	label={lst:pico-code-i2c-setup},
	firstline=51,
	lastline=56]{src/pico-code-i2c.py}

El envío de datos se realiza byte por byte, por lo que es necesario convertir la medición de temperatura (\emph{objeto tipo float}) en un arreglo de bytes que pueda ser transmitido.
Esto se hace con la función \code{pack} de \code{ustruct}, librería dedicada a la conversión de objetos de python a arreglos de bytes que representen los tipos primitivos que operan en C \Cref{lst:pico-code-i2c-reqh}.

\lstinputlisting[%
	language=Python,
	caption={\texttt{pico-code-i2c.py:25--33} --- Envío de datos},
	label={lst:pico-code-i2c-reqh},
	firstline=25,
	lastline=33]{src/pico-code-i2c.py}

Del lado de la Raspberry Pi, primero ha inicializarse el bus \IIC mediante el uso de la librería \texttt{smbus}\footnotemark y posteriormente se realizarán las lecturas en un poleo o bucle infinito, cada una de las cuales se irá almacenando en un archivo bitácora.
La inicialización del bus requiere de una simple línea (véase \Cref{lst:rpi-code-i2c-setup}).
\footnotetext{La implementación de la práctica utiliza \texttt{smbus2} que es una reimplementación codificada exclusivamente en Python de la librería \texttt{smbus} que es un \emph{wrapper} de la \emph{smbuslib} de C.}

\lstinputlisting[%
	language=Python,
	caption={\texttt{raspberry-code-i2c.py:26} --- Configuración del bus \IIC},
	label={lst:rpi-code-i2c-setup},
	linerange={9-10,19-21}% CHKTEX 8
	]{src/raspberry-code-i2c.py}

La conversión de un arreglo de bytes a punto flotante en Python no es inmediata.
Para esta opreación se utilizará la librería \texttt{struct} (véase \Cref{lst:rpi-code-i2c-setup}) que tomará los cuatro paquetes de 1 byte recibidos vía \IIC del RP2040 y los convertira en un \emph{float}, como se muestra en el \Cref{lst:rpi-code-i2c-read}.
Nótese el símbolo $<$ (menor qué) a la izquierda del especificador de formato $f$, el cual se utiliza para definir el endianness de la transmisión de la información.

\lstinputlisting[%
	language=Python,
	caption={\texttt{raspberry-code-i2c.py:28--35} --- Lectura de flotantes del bus \IIC},
	label={lst:rpi-code-i2c-read},
	linerange={23-35} %CHKTEX 8
	]{src/raspberry-code-i2c.py}

El resto del programa es trivial, pues consiste sólo en la escritura del \emph{timestamp UNIX} y el valor de temperatura registrado en un archivo de texto y la lectura de datos del RP2040 cada segundo.

Por conveniencia, los códigos completos de los programas de ejemplo se encuentran en los \Cref{sec:appendix1,sec:appendix2,sec:appendix3}.
