% %% %%%%%%%%%%%%%%%%%%%%%%%%%%%%%%%%%%%%%%%%%%%%%%%%%%%%%%%%%%
% experimens.tex
%
% Author:  Mauricio Matamoros
% License: MIT
%
% %% %%%%%%%%%%%%%%%%%%%%%%%%%%%%%%%%%%%%%%%%%%%%%%%%%%%%%%%%%%

%!TEX root = ../practica.tex
%!TEX root = ../references.bib

% CHKTEX-FILE 1
% CHKTEX-FILE 13
% CHKTEX-FILE 46

\section{Experimentos}%
\label{sec:experiments}

\begin{enumerate}
	\item{} [6pt] Modifique el código de la \cref{sec:step4} para que la Raspberry Pi imprima en pantalla los valores de temperatura leídos.
	\item{} [4pt] Modifique el código de la \cref{sec:step4} la Raspberry Pi grafique el histórico de temperaturas registradas, leyendo los valores almacenados e ingresados en la bitácora.
	\item{} [+5pt] Con base en lo aprendido, modifique el código de la \cref{sec:step4} para que la Raspberry Pi sirva una página web donde se pueda observar la gráfica de temperatura (histórico) desde la bitácora con resolución de hasta 1 minuto.
\end{enumerate}
