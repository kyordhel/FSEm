% %% %%%%%%%%%%%%%%%%%%%%%%%%%%%%%%%%%%%%%%%%%%%%%%%%%%%%%%%%%%
% step-2.tex
%
% Author:  Mauricio Matamoros
% License: MIT
%
% %% %%%%%%%%%%%%%%%%%%%%%%%%%%%%%%%%%%%%%%%%%%%%%%%%%%%%%%%%%%

%!TEX root = ../practica.tex
%!TEX root = ../references.bib

% CHKTEX-FILE 1
% CHKTEX-FILE 13
% CHKTEX-FILE 46

\subsection{Paso 2: Lectura del sensor LM35}%
\label{sec:step2}
Antes de proceder, verifique conexiones con un multímetro en busca de corto circuitos.
En particular verifique que exista una impedancia muy alta entre los pines \VCC, \GND y \textsc{ADC\_VREF} del RP2040.

Para leer la temperatura con el RP2040 se necesitan convertir los valores discretos leídos por el ADC del microcontrolador en valores de temperatura.
Esto se puede realizar mediante un simple análisis debido a la linearidad del LM35.
Se tienen dos lecturas en el ADC: V\textsubscript{OUT+} y V\textsubscript{OUT-}, de las cuales la segunda es la referencia del LM35 y por lo tanto, la diferencia entre estos voltajes será proporcional a la temperatura en escala centígrada.
Esto expresado matemáticamente es:

\begin{equation*}
	T[^{o}C] \propto V_{diff} = V_{OUT+}-V_{OUT-}
\end{equation*}
\noindent o bien
\begin{equation*}
	T[^{o}C] = k \times V_{diff} = k \times \big( V_{OUT+}-V_{OUT-} \big)
\end{equation*}

\noindent lo que implica que en $T = 0^{o}C; V_{OUT+}=V_{OUT-} \rightarrow V_{diff} = 0$

\medskip
Es necesario entonces calcular la constante de proporcionalidad $k$.
Sabemos que el ADC entregará lecturas de 0 a 4096 para los voltajes registradoes entre \GND y {ADC\_VREF} (0V y 2.72V respectivamente), además de que $1^{o}C = 0.01V$.
Luego entonces
\begin{align*}
	T[^{o}C] &= V_{diff} \times \frac{2.72[V]}{ 4096 \times 0.01[\tfrac{V}{^{o}C}] }  \\
	T[^{o}C] &= V_{diff} \times \frac{2.72}{ 40.96 }[^{o}C]  \\
\end{align*}

\noindent o bien, generalizando para todo voltaje de referencia:
\begin{equation*}
	T[^{o}C] = V_{diff} \times \frac{V_{REF}}{ 40.96 }[^{o}C]
\end{equation*}

Esta fórmula de conversión de unidades deberá programarse en el microcontrolador que adquiera los valores discretos de temperatura del sensor.

\medskip
\begin{greenbox}{\large TIP: Calibración}
	El voltaje de referencia $V_{REF}$ variará respecto a su valor teórico dependiendo de la tolerancia de las resistencias (típicamente ±5\%) y de las impedancias de los demás componentes conectados, incluyendo al RP2040 mismo.

	\smallskip

	Se recomienda usar siempre un multímetro para corroborar el voltaje entre $V_{REF}$ y tierra y poder actualizar dicho valor en el código.\footnotemark%
\end{greenbox}
\footnotetext{El valor real de $V_{REF}$ variará entre 2.4V y 3.1V con resistencias con tolerancia de 5\%, equivalente a una variación de 70°C.}
\medskip

El programa de ejemplo para el RP2040\footnotemark{} se presenta a continuación:

\begin{minipage}{\linewidth}
\lstinputlisting[%
	language=Python,
	caption={\texttt{pico-code-adc.py}:23--38, \texttt{read\_temp} function},
	label={lst:pico-code-readtemp},
	% style=py_without_comments,
	showlines=false,
	firstline=23,
	lastline=38]{src/pico-code-adc.py}
\end{minipage}


En ocasiones los valores pueden fluctuar ligeramente debido a ruido o variaciones de voltaje.
Para evitar este tipo de imprecisiones es común utilizar técnicas de filtrado, y uno de los métodos más simples y comúnes es el promedio de varias lecturas consecutivas tal y como se muestra a continuación:

\begin{minipage}{\linewidth}
\lstinputlisting[%
	language=Python,
	caption={\texttt{pico-code-adc.py}:41--49, \texttt{read\_avg\_temp} function},
	label={lst:pico-code-readavgtemp},
	% style=py_without_comments,
	firstline=41,
	lastline=49]{src/pico-code-adc.py}
\end{minipage}

Otro método mucho más eficaz y seguro es llevar un registro de las últimas $N$ lecturas del sensor en un buffer circular y estimar el siguiente valor probable, descartando aquellas lecturas que estén fuera de rango posible, es decir cuando $\Delta{}t \geq \epsilon{}_0$.

\VerbatimFootnotes{}
\footnotetext{%
	Para cargar el código de Python en el RP2040 utilice el editor
	\href{https://thonny.org/}{Thonny}.
	Para que los cambios sean persistentes, salve el archivo dentro del microcontrolador con el nombre \code{main.py}.
	Recuerde que no es posible programar el RP2040 via \IIC{}.
}