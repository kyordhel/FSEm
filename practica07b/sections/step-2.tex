% %% %%%%%%%%%%%%%%%%%%%%%%%%%%%%%%%%%%%%%%%%%%%%%%%%%%%%%%%%%%
% step-2.tex
%
% Author:  Mauricio Matamoros
% License: MIT
%
% %% %%%%%%%%%%%%%%%%%%%%%%%%%%%%%%%%%%%%%%%%%%%%%%%%%%%%%%%%%%
%!TEX root = ../practica.tex
%!TEX root = ../references.bib

% CHKTEX-FILE 1
% CHKTEX-FILE 13
% CHKTEX-FILE 46
\subsection{Paso 2: Configuración de comunicaciones \IIC}%
\label{sec:step2}
La comunicación está dividida en dos etapas: la configuración del dispositivo maestro (la Raspberry Pi)
y la configuración del dispositivo esclavo (el RP2040).
Se comenzará con la configuración del dispositivo maestro.

\subsubsection*{Configuración del dispositivo maestro}

Primero ha de configurarse la Raspberry Pi para funcionar como dispositivo maestro o \emph{master} en el bus \IIC.
Para esto, inicie la utilidad de configuración de la Raspberry Pi con el comando

\begin{Verbatim}
# raspi-config
\end{Verbatim}

\noindent y seleccione la opción 5: Opciones de Interfaz (\emph{Interfacing Options}) y active la opción \texttt{P5} para habilitar el \IIC.

A continuación, verifique que el puerto \IIC no se encuentre en la lista negra.
Edite el archivo \\\texttt{/etc/modprobe.d/raspi-blacklist.conf} y revise que la línea \texttt{blacklist spi-bcm2708} esté comentada con \#.

% $ cat /etc/modprobe.d/raspi-blacklist.conf
\begin{lstlisting}[
	language=conf,
	caption={/etc/modprobe.d/raspi-blacklist.conf},
	label={lst:raspi-blacklist.conf},
	numbers=none
]
# blacklist spi and i2c by default (many users don't need them)
# blacklist i2c-bcm2708
\end{lstlisting}

Como paso siguiente, se habilita la carga del driver \IIC.
Esto se logra agregando la línea \texttt{i2c-dev} al final del archivo \texttt{/etc/modules} si esta no se encuentra ya allí.

Por último, se instalan los paquetes que permiten la comunicación mediante el bus \IIC y se habilita al usuario predeterminado \emph{pi} (o cualquier otro que se esté usando) para acceder al recurso.

\begin{Verbatim}
# apt-get install i2c-tools python-smbus
# adduser pi i2c
\end{Verbatim}

Reinicie la Raspberry Pi y pruebe la configuración ejecutando \texttt{i2cdetect -y 1} para buscar dispositivos conectados al bus \IIC.
Debería ver una salida como la siguiente:

\begin{Verbatim}
\$ i2cdetect -y 1
     0  1  2  3  4  5  6  7  8  9  a  b  c  d  e  f
00:          -- -- -- -- -- -- -- -- -- -- -- --
10: -- -- -- -- -- -- -- -- -- -- -- -- -- -- --
20: -- -- -- -- -- -- -- -- -- -- -- -- -- -- --
30: -- -- -- -- -- -- -- -- -- -- -- -- -- -- --
40: -- -- -- -- -- -- -- -- -- -- -- -- -- -- --
50: -- -- -- -- -- -- -- -- -- -- -- -- -- -- --
60: -- -- -- -- -- -- -- -- -- -- -- -- -- -- --
70: -- -- -- -- -- -- -- --
\end{Verbatim}


\subsubsection*{Configuración del dispositivo esclavo}

A continuación, es necesario configurar el RP2040 para funcionar como dispositivo esclavo o \emph{slave} en el bus \IIC.

Debido a que los desarrolladores de MicroPython consideran que el RP2040 es un dispositivo demasiado poderoso para actuar esclavo,
el firmware sólo considera la configuración del periférico como maestro.
Así, será necesario configurar a nivel registro el periférico utilizando las primitivas de acceso directo a memoria \texttt{machine.mem32} de MircoPython y realizar todas las transacciones a nivel registro y sin posibilidad de gestionar de forma eficiente las interrrupciones.

Por fortuna para los estudiantes, en este manual se incluye la librería \texttt{i2cslave.py} que realiza justamente eso.
Para usarla, basta con cargar el archivo y crear un objeto de tipo \texttt{I2CSlave} y especificar el número de periférico a utilizar (\IIC{}0 o \IIC{}1) y los pines que se utilizarán como SDA y SCL, tal como se muestra en el \Cref{lst:rp2040-test-i2c}.

\lstinputlisting[%
	language=Python,
	caption={\texttt{rp2040-test-i2c.py:23} --- Dirección asignada al dispositivo esclavo},
	label={lst:rp2040-test-i2c},
	firstline=23,
	lastline=23]{src/rp2040-test-i2c.py}

Al no existir soporte para interrupciones (las comunicaciones \IIC{} son demasiado rápidas para Python), será necesario realizar las operaciones de lectura y escritura mediante los métodos:
\begin{itemize}[nosep] %CHKTEX-FILE 36
	\item \texttt{rxBufferCount()} que devuelve el número de bytes almacenados en el buffer de recepción,
	\item \texttt{read()} que realiza una lectura síncrona de los datos en el buffer,
	y
	\item \texttt{write(data)} que realiza una transmisión sincrona de los datos proporcionados como un tipo \texttt{bytes} o \texttt{bytearray}.
\end{itemize}
