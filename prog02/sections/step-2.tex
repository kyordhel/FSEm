% %% %%%%%%%%%%%%%%%%%%%%%%%%%%%%%%%%%%%%%%%%%%%%%%%%%%%%%%%%%%
% step-2.tex
%
% Author:  Mauricio Matamoros
% License: MIT
%
% %% %%%%%%%%%%%%%%%%%%%%%%%%%%%%%%%%%%%%%%%%%%%%%%%%%%%%%%%%%%

%!TEX root = ../main.tex
%!TEX root = ../references.bib

\subsection{Paso 2: Led parpadeante}%
\label{sec:step2}
El código mostrado en \Cref{src:blink} muestra cómo se haría parpadear un LED mediante tiempos de espera o \emph{sleeps} utilizando la Raspberry Pi.

\smallskip
\lstinputlisting[%
	language=Python,
	linerange={18-40}, % chktex 8
	caption={\texttt{blink.py}},
	label={src:blink}
]{src/blink.py}
\smallskip

Estudie el código y véalo en funcionamiento, ejecutándolo de la siguiente manera:
\begin{Verbatim}[fontsize=\footnotesize]
./blink.py
\end{Verbatim}
