% %% %%%%%%%%%%%%%%%%%%%%%%%%%%%%%%%%%%%%%%%%%%%%%%%%%%%%%%%%%%
% step-4.tex
%
% Author:  Mauricio Matamoros
% License: MIT
%
% %% %%%%%%%%%%%%%%%%%%%%%%%%%%%%%%%%%%%%%%%%%%%%%%%%%%%%%%%%%%

%!TEX root = ../main.tex
%!TEX root = ../references.bib

\subsection{Paso 4: Configuración de la Raspberry Pi como punto de acceso inalámbrico}%
\label{sec:ap}

Para operar como punto de acceso la Raspberry Pi necesita tener instalado el software apropiado, incluyendo un servidor DHCP para proporcionar a los dispositivos que se conecten una dirección IP. %CHKTEX 13

Se comienza por instalar los paquetes \texttt{DNSMasq} y \texttt{HostAPD}:

\begin{Verbatim}
# apt-get install dnsmasq hostapd
# pip3 install -U python-magic
\end{Verbatim}

Si están ejecutándose los servicios, deténgalos a fin de poder reconfigurarlos
\begin{Verbatim}
# systemctl stop dnsmasq
# systemctl stop hostapd
\end{Verbatim}


\subsubsection{Configuración del adaptador y el cliente DHCP}%
\label{sec:ap-adapter}
Para configurar una red independiente con servidor DHCP la Raspberry Pi debe tener asignada una dirección IP estática en el adaptador inalámbrico que proveerá la conexión.
Debido a que la Raspberry Pi tiene un procesador pequeño, se configurará para servir en una red privada clase C, es decir con direcciones IP del tipo 198.168.x.x.
Así mismo, se supondrá que el dispositivo inalámbrico utilizado es \texttt{wlan0}.
En el caso de Raspbian ejecutándose en una máquina virtual, utilice el puerto ethernet predeterminado \texttt{eth0}.

Para configurar la dirección IP estática edite el archivo de configuración \texttt{/etc/dhcpcd.conf} como superusuario:

\begin{Verbatim}
interface wlan0
    static ip_address=192.168.1.254/24
    nohook wpa_supplicant
\end{Verbatim}

A continuación, reinicie el cliente DHCP

\begin{Verbatim}
# service dhcpcd restart
\end{Verbatim}

\subsubsection{Configuración del servidor DHCP}%
\label{sec:ap-dhcp}

\begin{importantbox}{\bfseries IMPORTANTE}
Configurar un servidor DHCP en su computadora de trabajo podría dejarle sin acceso a la red local y sin salida a internet. No realice estas configuraciones si no trabaja con una Raspberry Pi física.
\end{importantbox}

El siguiente paso consiste en configurar el servidor DHCP, provisto por el servicio \texttt{dnsmasq}.

De manera predeterminada el archivo de configuración \texttt{/etc/dnsmasq.conf} contiene mucha información que no es necesaria, por lo que es más fácil comenzar desde cero.
Respáldelo y cree uno nuevo con el siguiente texto:

\VerbatimInput[%
	samepage=true,
	label=\texttt{/etc/dnsmasq.conf},
]{src/dnsmasq.conf}

Esta configuración proporcionará 20 direcciones IP entre 192.168.1.200 y 192.168.1.220, válidas durante 24 horas.

Ahora debe iniciarse el servidor DHCP

\begin{Verbatim}
systemctl start dnsmasq
\end{Verbatim}

\subsubsection{Configuración del punto de acceso}%
\label{sec:ap-hotspot}
\begin{importantbox}{\bfseries IMPORTANTE}
Configurar un punto de acceso en su computadora de trabajo o en una máquina vrtual podría dejarle sin acceso a la red local y sin salida a internet. No realice estas configuraciones si no trabaja con una Raspberry Pi física.
\end{importantbox}

Para configurar el punto de acceso se debe editar el archivo de configuración \texttt{/etc/hostapd/hostapd.conf} con los parámetros adecuados.

Respáldelo y cree uno nuevo con el siguiente texto:

\VerbatimInput[%
	samepage=true,
	label=\texttt{/etc/hostapd/hostapd.conf},
]{src/hostapd.conf}

La configuración ingresada configura la Raspberry Pi para crear una red inalámbrica tipo 802.11g en el canal 5 de nombre \emph{Raspbberry} y contraseña \texttt{12345678} con seguridad WPA2.

Los modos de operación posibles son:
\begin{itemize}[nosep]
\item a = IEEE 802.11a (5 GHz)
\item b = IEEE 802.11b (2.4 GHz)
\item g = IEEE 802.11g (2.4 GHz)
\end{itemize}

\paragraph*{Importante:} Tanto el nombre de la red o SSID y la contraseña no deben entrecomillarse. La contraseña debe tener entre 8 y 64 caracteres. \textbf{Cambie el SSID a \texttt{Raspberry\_Apellido} para evitar conflictos.}\medskip

Ahora edite el archivo \texttt{/etc/default/hostapd} y reemplace la línea que comienza con \texttt{\#DAEMON\_CONF} con:

\begin{Verbatim}
DAEMON_CONF="/etc/hostapd/hostapd.conf"
\end{Verbatim}

\subsubsection{Configuración e inicio del punto de acceso}%
\label{sec:ap-hotspot}
Finalmente, habilite los servicios para iniciar el punto de acceso:

\begin{Verbatim}
# systemctl unmask hostapd
# systemctl enable hostapd
# systemctl start hostapd
\end{Verbatim}

Verifique que los servicios se están ejecutando

\begin{Verbatim}
# systemctl status hostapd
# systemctl status dnsmasq
\end{Verbatim}

\paragraph*{Nota:} El servicio \texttt{hostapd} requiere acceso exclusivo a la tarjeta de red inalámbrica que podría estar ocupada por el proceso \texttt{wpa\_supplicant}.
Si \texttt{hostapd} se reusara a iniciar indicando un error tal como \emph{Could not configure driver mode nl80211 driver initialization failed}, termine los procesos que puedan estar utilizando la tarjeta de red inalámbrica, por ejemplo ejecutando \texttt{killall wpa\_supplicant}.

