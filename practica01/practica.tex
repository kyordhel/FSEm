% CHKTEX-FILE 46
% %% %%%%%%%%%%%%%%%%%%%%%%%%%%%%%%%%%%%%%%%%%%%%%%%%%%%%%%%%%%
% practica.tex
%
% Author:  Mauricio Matamoros
% License: MIT
%
% %% %%%%%%%%%%%%%%%%%%%%%%%%%%%%%%%%%%%%%%%%%%%%%%%%%%%%%%%%%%

\documentclass[letterpaper,10.5pt]{article}
% %% %%%%%%%%%%%%%%%%%%%%%%%%%%%%%%%%%%%%%%%%%%%%%%%%%%%%%%%%%
%
% packages.tex
%
%  Author: Mauricio Matamoros
%  Date:   2020.02.28
%
%  Contiene la lista de paquetes requeridos para generar
%  el archivo-reporte de las prácticas de laboratorio
%
% %% %%%%%%%%%%%%%%%%%%%%%%%%%%%%%%%%%%%%%%%%%%%%%%%%%%%%%%%%%
% Archivo principal de LaTeX
%!TEX root = ../reporte.tex

\usepackage[utf8]{inputenc}                  % Soporte para utf8
\usepackage[T1]{fontenc}                     % Soporte extendido de caracteres unicode
\usepackage[english,spanish,mexico]{babel}   % Define el idioma del documento a español (México) con soporte para inglés
% Standard packages
\usepackage{float}                           % Imágenes flotantes en el documento
\usepackage{ifthen}                          % Soporte if-then en macros
\usepackage{xspace}                          % Soporte de autoespaciado en macros
\usepackage{xstring}                         % Operaciones con cadenas en macros
\usepackage{wrapfig}                         % Permite colocar texto al rededor de figuras y otros flotantes
\usepackage{booktabs}                        % Embellece tablas
\usepackage{csquotes}                        % Entrecomillado automático y manejo de citas textuales
\usepackage{fancyhdr}                        % Permite reconfigurar encabezado y pie de página
\usepackage{fancyvrb}                        % Define estilos para entornos Verbatim
\usepackage{geometry}                        % Permite reconfigurar la geometría del documento
\usepackage{graphicx}                        % Permite insertar imágenes en varios formatos
\usepackage{lastpage}                        % Referencia a la última página del documento
\usepackage{listings}                        % Define estilos para entornos de código de programación (sintaxis)
\usepackage{multicol}                        % Manejo de texto en varias columnas
\usepackage{tabularx}                        % Tablas con ancho de columna variable
\usepackage{algorithm}                       % Entorno para escribir algoritmos
\usepackage{algpseudocode}                   % Entorno para escribir algoritmos en pseudocódigo
\usepackage[justification=centering]{subcaption} % Permite imágenes en viñetas
\usepackage[all]{nowidow}                    % Control de viudas y huérfanas
\usepackage[inline]{enumitem}                % Añade opciones de configuración a listas
\usepackage[usenames,dvipsnames]{xcolor}     % Permite el uso de colores en el documento
% Referencing
\usepackage{varioref}                        % Gestión de referencias variables
\usepackage{hyperref}                        % Gestión de referencias e hipervínculos
\usepackage[noabbrev,nameinlink,spanish]{cleveref} % Gestión de referencias cruzadas inteligentes con hipervínculos
\usepackage[square, comma, numbers, sort&compress]{natbib} % Gestión de referencias bibliográficas

% %% %%%%%%%%%%%%%%%%%%%%%%%%%%%%%%%%%%%%%%%%%%%%%%%%%%%%%%%%%
%
% macros.tex
%
%  Author: Mauricio Matamoros
%  Date:   2020.02.28
%
%  Contiene los macros personalizados definidos para
%  el archivo-reporte de las prácticas de laboratorio
%
% %% %%%%%%%%%%%%%%%%%%%%%%%%%%%%%%%%%%%%%%%%%%%%%%%%%%%%%%%%%
% Archivo principal de LaTeX
%!TEX root = ../reporte.tex

%CHKTEX-FILE 9
%CHKTEX-FILE 21

\newcommand{\lpar}{(}
\newcommand{\rpar}{)}
\newcommand{\textcommand}[1]{\texttt{\textbackslash#1\{\}}}
\newcommand{\textenviron}[2][y]{\texttt{\textbackslash{}begin\{#2\} #1 \textbackslash{}end\{#2\}}}


\input{setup/colorboxes}
%CHKTEX-FILE 1
%CHKTEX-FILE 7
%CHKTEX-FILE 9
% Default fixed font does not support bold face
\DeclareFixedFont{\ttb}{T1}{txtt}{bx}{n}{8} % for bold
\DeclareFixedFont{\ttm}{T1}{txtt}{m}{n}{8}  % for normal

% Custom colors
\usepackage{color}
\definecolor{keywordsColor}{rgb}{0,0,0.5}
\definecolor{customColor}{rgb}{0.6,0,0}
\definecolor{stringColor}{rgb}{0,0.5,0}

% Code highlighting python
\renewcommand{\ttdefault}{pcr}
\lstset{
	language=Python,                              % the language of the code (can be overrided per snippet)
	backgroundcolor=\color{white},                % choose the background color
	basicstyle=\footnotesize\ttfamily,            % the size of the fonts that are used for the code
	breakatwhitespace=false,                      % sets if automatic breaks should only happen at whitespace
	breaklines=true,                              % sets automatic line breaking
	captionpos=t,                                 % sets the caption-position to bottom
	commentstyle=\color{gray},                    % comment style
	deletekeywords={},                            % if you want to delete keywords from the given language
%	escapeinside={\%*}{*)},                       % if you want to add LaTeX within your code
	extendedchars=true,                           % lets you use non-ASCII characters; for 8-bits encodings only, does not work with UTF-8
	frame=tb,                                     % adds a frame around the code
	keepspaces=true,                              % keeps spaces in text, useful for keeping indentation of code (possibly needs columns=flexible)
	keywordstyle=\color{keywordsColor}\bfseries,  % keyword style
	numbers=left,                                 % where to put the line-numbers; possible values are (none, left, right)
	numbersep=5pt,                                % how far the line-numbers are from the code
	numberstyle=\tiny\color{gray},                % the style that is used for the line-numbers
	rulecolor=\color{black},                      % if not set, the frame-color may be changed on line-breaks within not-black text (e.g. comments (green here))
	showspaces=false,                             % show spaces everywhere adding particular underscores; it overrides 'showstringspaces'
	showstringspaces=false,                       % underline spaces within strings only
	showtabs=false,                               % show tabs within strings adding particular underscores
	stepnumber=1,                                 % the step between two line-numbers. If it's 1, each line will be numbered
	stringstyle=\color{stringColor},              % string literal style
	tabsize=2,                                    % sets default tabsize to 2 spaces
	title=\lstname,                               % show the filename of files included with \lstinputlisting; also try caption instead of title
	columns=fixed,                                % Using fixed column width (for e.g. nice alignment)
	otherkeywords={self},                         % if you want to add more keywords to the set
	emphstyle=\color{customColor}\bfseries,       % Custom highlighting style
	emph={__init__,__main__,True,False,None},     % Custom highlighting keywords
	xleftmargin=1cm,                              % Left margin
	xrightmargin=1cm,                             % Right margin
	% Unicode compatibility
	inputencoding=utf8,
	literate={%
	            {Á}{{\'a}}1 {É}{{\'E}}1 {Í}{{\'I}}1 {Ó}{{\'O}}1 {Ú}{{\'U}}1%
	            {á}{{\'a}}1 {é}{{\'e}}1 {í}{{\'i}}1 {ó}{{\'o}}1 {ú}{{\'u}}1%
	            {À}{{\`A}}1 {È}{{\'E}}1 {Ì}{{\`I}}1 {Ò}{{\`O}}1 {Ù}{{\`U}}1%
	            {à}{{\`a}}1 {è}{{\`e}}1 {ì}{{\`i}}1 {ò}{{\`o}}1 {ù}{{\`u}}1%
	            {Ä}{{\"A}}1 {Ë}{{\"E}}1 {Ï}{{\"I}}1 {Ö}{{\"O}}1 {Ü}{{\"U}}1%
	            {ä}{{\"a}}1 {ë}{{\"e}}1 {ï}{{\"i}}1 {ö}{{\"o}}1 {ü}{{\"u}}1%
	            {Â}{{\^A}}1 {Ê}{{\^E}}1 {Î}{{\^I}}1 {Ô}{{\^O}}1 {Û}{{\^U}}1%
	            {â}{{\^a}}1 {ê}{{\^e}}1 {î}{{\^i}}1 {ô}{{\^o}}1 {û}{{\^u}}1% CHKTEX 19
	            {Ã}{{\~a}}1 {Ẽ}{{\~E}}1 {Ĩ}{{\~I}}1 {Õ}{{\~O}}1 {Ũ}{{\~U}}1 {Ñ}{{\~N}}1%
	            {ã}{{\~a}}1 {ẽ}{{\~e}}1 {ĩ}{{\~i}}1 {õ}{{\~o}}1 {ũ}{{\~u}}1 {ñ}{{\~n}}1%
	            {œ}{{\oe}}1 {Œ}{{\OE}}1 {æ}{{\ae}}1 {Æ}{{\AE}}1 {ß}{{\ss}}1%
	            {ç}{{\c c}}1 {Ç}{{\c C}}1 {ø}{{\o}}1 {å}{{\r a}}1 {Å}{{\r A}}1%
	            {€}{{\EUR}}1 {£}{{\pounds}}1 {×}{{\(\times\)}}1% CHKTEX 21
	            {°}{{\textsuperscript{o}}}1%
	            {¹}{{\textsuperscript{1}}}1%
	            {²}{{\textsuperscript{2}}}1%
	            {³}{{\textsuperscript{3}}}1%
	            {⁴}{{\textsuperscript{4}}}1% CHKTEX 19
	            {⁵}{{\textsuperscript{5}}}1% CHKTEX 19
	            {⁶}{{\textsuperscript{6}}}1% CHKTEX 19
	            {⁷}{{\textsuperscript{7}}}1% CHKTEX 19
	            {⁸}{{\textsuperscript{8}}}1% CHKTEX 19
	            {⁹}{{\textsuperscript{9}}}1% CHKTEX 19
	            {⁰}{{\textsuperscript{0}}}1% CHKTEX 19
%	            {A}{{\textAlpha}}1
	            {α}{{\textalpha}}1%
%	            {B}{{\textBeta}}1
	            {β}{{\textbeta}}1%
	            {Γ}{{\textGamma}}1
	            {γ}{{\textgamma}}1%
	            {Δ}{{\textDelta}}1
	            {δ}{{\textdelta}}1% CHKTEX 19
%	            {E}{{\textEpsilon}}1
	            {ϵ}{{\textepsilon}}1%
%	            {Z}{{\textZeta}}1
	            {ζ}{{\textzeta}}1%
%	            {H}{{\textEta}}1
	            {η}{{\texteta}}1%
	            {Θ}{{\textTheta}}1
	            {θ}{{\texttheta}}1%
%	            {I}{{\textIota}}1
	            {ι}{{\textiota}}1%
%	            {K}{{\textKappa}}1
	            {κ}{{\textkappa}}1%
	            {Λ}{{\textLambda}}1
	            {λ}{{\textlambda}}1%
%	            {M}{{\textMu}}1
	            {μ}{{\textmu}}1%
%	            {N}{{\textNu}}1
	            {ν}{{\textnu}}1%
	            {Ξ}{{\textXi}}1
	            {ξ}{{\textxi}}1%
%	            {O}{{\textOmikron}}1
%	            {o}{{\textomikron}}1%
	            {Π}{{\textPi}}1
	            {π}{{\textpi}}1%
%	            {P}{{\textRho}}1
	            {ρ}{{\textrho}}1%
	            {Σ}{{\textSigma}}1
	            {σ}{{\textsigma}}1%
%	            {T}{{\textTau}}1
	            {τ}{{\texttau}}1%
	            {ϒ}{{\textUpsilon}}1
	            {υ}{{\textupsilon}}1%
	            {Φ}{{\textPhi}}1
	            {ϕ}{{\textphi}}1%
%	            {X}{{\textChi}}1
	            {χ}{{\textchi}}1%
	            {Ψ}{{\textPsi}}1
	            {ψ}{{\textpsi}}1%
	            {Ω}{{\textOmega}}1
	            {ω}{{\textomega}}1%
	            {ζ}{{\varsigma}}1%
%	            {}{{\straightphi}}1%
%	            {}{{\scripttheta}}1%
%	            {}{{\straighttheta}}1%
%	            {}{{\straightepsilon}}1%
	         },
}

\lstdefinestyle{c_with_comments}%
{
	language     = c,
	morecomment  = [l]{//},
	morecomment  = [s]{/*}{*/},
	breaklines,
}

\lstdefinestyle{c_without_comments}%
{
	style        = c_with_comments,
	% numbers      = none,
	% keepspaces   = false,
	morecomment  = [l][\nullfont]{//},
	morecomment  = [is]{//}{\^^M},
	morecomment  = [is]{/*}{*/},
}

\lstdefinelanguage{conf}
{
	basicstyle=\ttfamily\small,
	columns=fullflexible,
	morecomment=[s][\color{Orchid}\bfseries]{[}{]},
	morecomment=[l]{\#},
	morecomment=[l]{;},
	commentstyle=\color{gray}\ttfamily,
	% morekeywords={},
	% otherkeywords={=,:},
	% keywordstyle={\color{Green}\bfseries}
}

% \captionsetup[lstlisting]{font={small,tt}}
\captionsetup[lstlisting]{%
	font={small},
}



\DefineVerbatimEnvironment{Verbatim}{Verbatim}{%
	fontsize=\footnotesize,%
	frame=leftline,%
	framesep=2em,    % separation between frame and text
}

\RecustomVerbatimCommand{\VerbatimInput}{VerbatimInput}{%
	fontsize=\footnotesize,
%	frame=lines,            % top and bottom rule only
	frame=leftline,         % left rule only
	numbers=left,           % Line numbers on the left
	numbersep=0.25em,       % Gap between numbers and verbatim lines
	xleftmargin=4em,        % Indentation to add at the start of each line
	xrightmargin=4em,       % Right margin to add after each line
	framesep=0.5em,         % separation between frame and text
	rulecolor=\color{Gray}, % Color of the lines
	labelposition=topline,  %
	samepage=false,         % When true, prevents verbatim environment from
	                        % being broken between pages
%	commandchars=\|\(\),    % escape character and argument delimiters for
	                        % commands within the verbatim
%	commentchar=*           % comment character
}


\hypersetup{
	hidelinks,
	colorlinks=true,
	linkcolor=Blue,
	filecolor=OliveGreen,
	urlcolor=RoyalPurple,
	pdfauthor={Mauricio Matamoros},
%	pdftitle={Práctica 0X – Fundamentos de Sistemas Embebidos},
% 	pdfsubject={The Subject},
% 	pdfkeywords={Some Keywords},
% 	pdfproducer={Latex with hyperref, or other system},
% 	pdfcreator={pdflatex, or other tool}
}

\captionsetup{%
	font=small
}

\geometry{%
	margin=2cm,
	% top=3cm,
	bottom=3cm,
	% left=2cm,
	% right=2cm,
	% inner=2cm,
	% outer=2cm,
	% headheight=,
	% footsep=,
	% footskip=,
}

\pagestyle{fancy}
\renewcommand{\headrulewidth}{0.0pt}
\lhead{}
\chead{}
\rhead{}
\lfoot{}
\cfoot{}
\rfoot{Página~\thepage~de~\pageref{LastPage}}

\crefname{table}{tabla}{tablas}
\Crefname{table}{Tabla}{Tablas}
\crefname{section}{sección}{secciones}
\Crefname{section}{Sección}{Secciones}
\crefname{subsection}{subsección}{subsecciones}
\Crefname{subsection}{Subsección}{Subsecciones}
\crefname{listing}{código de ejemplo}{códigos de ejemplo}
\Crefname{listing}{Código de Ejemplo}{Códigos de Ejemplo}
\renewcommand*{\lstlistingname}{Código ejemplo}


\author{\footnotesize Autor: José Mauricio Matamoros de Maria y Campos}
\title{Práctica 1:\\Instalación de Raspbian en la Raspberry Pi\\
{\large Fundamentos de Sistemas Embebidos}}
\date{}


% Document body
\begin{document}
\maketitle

\section{Objetivo}%
\label{sec:objective}
El alumno aprenderá a instalar un sistema operativo basado en Linux, como sistema operativo embebido, en una tarjeta microcontroladora.

\section{Introducción}%
\label{sec:introduction}
Raspbian es el sistema operativo más popular para Raspberry Pi, además de ser el único con soporte oficial.
Raspbian es una distribución de Linux basada en Debian, optimizado para la Raspberry Pi y que permite a esta operar como una PC.~La distro incorpora terminal y navegador web entre otros programas.

\section{Instrucciones}%
\label{sec:instructions}

Instalar Raspbian en la Raspberry Pi es sencillo.
Basta con descargar Raspbian (ahora Raspberry Pi Os) y grabar la imagen de disco en una tarjeta de memoria microSD, desde la cual arrancará el sistema operativo.

Para esta práctica se necesitará:
\begin{itemize}[nosep]
	\item Una tarjeta de memoria microSD de al menos 4 GB (se recomiendan 8GB)
	\item Una computadora capaz de leer y escribir tarjetas microSD (o bien un adaptador para la misma) y conexión a internet para descargar la imagen de Raspbian.
	\item Una Raspberry Pi 2B o posterior
	\item Un monitor con soporte para HDMI
	\item Un teclado USB
	\item Un mouse USB
	\item Una fuente de alimentación de 5V@1A con adaptador microUSB
\end{itemize}

\paragraph*{Importante:} Si no cuenta con monitor, teclado y mouse, aún es posible instalar Raspbian en la Raspberry Pi. Consulte el \Cref{sec:anex1}.

% %% %%%%%%%%%%%%%%%%%%%%%%%%%%%%%%%%%%%%%%%%%%%%%%%%%%%%%%%%%%%%%%%%%%
%
% Step 1
%
% %% %%%%%%%%%%%%%%%%%%%%%%%%%%%%%%%%%%%%%%%%%%%%%%%%%%%%%%%%%%%%%%%%%%
\subsection{Paso 1: Descargar Raspbian}%
\label{sec:step1}
Ingrese a \url{https://www.raspberrypi.com/software/operating-systems/} y descargue alguna de las imágenes de Raspbian disponibles (véase \Cref{fig:raspbian-flavors}). Si se descargó un archivo \texttt{img.xz}, habrá que descomprimirlo para extraer la imagen.

\begin{figure}
	\centering%
	\includegraphics[width=0.8\columnwidth,height=8cm,keepaspectratio]{img/p01-01.png} %CHKTEX 8
	\caption{Versiones disponibles (o sabores) de Raspbian}
	\label{fig:raspbian-flavors} %CHKTEX 24
\end{figure}

\begin{greenbox}{Importante}
Si tiene una Raspberry Pi 3 o 4 descargue la versión de 64 bits. Es más eficiente.
\end{greenbox}

La versión a descargar dependerá de la capacidad de la memoria microSD y la cantidad de recursos de la tarjeta Raspbian.
Para tarjetas de 4GB se aconseja el sabor \emph{Raspberry Pi OS with desktop}, mientras que usuarios con Raspberries de sólo 512MB de RAM querrán instalar \emph{Raspberry Pi OS Lite}.

Ligas de acceso rápido se proporcionan a continuación por conveniencia:

\begin{itemize}[noitemsep]
	\item \href{https://downloads.raspberrypi.org/raspios_arm64/images/raspios_arm64-2023-02-22/2023-02-21-raspios-bullseye-arm64.img.xz}{Raspberry Pi OS with desktop}
	\item \href{https://downloads.raspberrypi.org/raspios_lite_arm64/images/raspios_lite_arm64-2023-02-22/2023-02-21-raspios-bullseye-arm64-lite.img.xz}{Raspberry Pi OS Lite}
\end{itemize}

% %% %%%%%%%%%%%%%%%%%%%%%%%%%%%%%%%%%%%%%%%%%%%%%%%%%%%%%%%%%%%%%%%%%%
%
% Step 2
%
% %% %%%%%%%%%%%%%%%%%%%%%%%%%%%%%%%%%%%%%%%%%%%%%%%%%%%%%%%%%%%%%%%%%%
\subsection{Paso 2: Escribir imagen en la microSD}%
\label{sec:step2}
Si no lo ha hecho, introduzca la memoria microSD en la computadora.
La memoria será reparticionada y cada partición será reformateada, por lo que se aconseja respaldar la información.

Escribir la imagen de Raspbian en la microSD requiere de un programa externo según el sistema operativo.

% %% %%%%%%%%%%%%%%%%%%%%%%%%%%%%%%%%%%%%%%%%%%%%%%%%%%%%%%%%%%%%%%%%%%
%
% Step 2a: Linux
%
% %% %%%%%%%%%%%%%%%%%%%%%%%%%%%%%%%%%%%%%%%%%%%%%%%%%%%%%%%%%%%%%%%%%%
\subsubsection{Escribir imagen usando Linux}%
Descargue \href{https://etcher.io/}{Etcher} en \texttt{\textasciitilde/Downloads}.
el archivo \texttt{AppImage} ya es ejecutable por lo que sólo hace falta darle permisos de ejecución y ejecutarlo; por ejemplo:

\begin{Verbatim}[fontsize=\footnotesize]
$ cd ~/Downloads
$ wget https://github.com/balena-io/etcher/releases/download/v1.14.3/balenaEtcher-1.14.3-x64.AppImage
$ chmod +x balenaEtcher-1.14.3-x64.AppImage
$ ./balenaEtcher-1.14.3-x64.AppImage
\end{Verbatim}


A continuación, siga los pasos de Etcher para grabar la imagen (véase \Cref{fig:write-image-linux})
\begin{enumerate}[noitemsep]
	\item Seleccione la imagen \texttt{IMG} de Raspbian
	\item Seleccione el medio en el cual se grabará la imagen de Raspbian (punto de montaje de la microSD)
	\item De click en \emph{Flash!} para empezar el proceso de grabado
\end{enumerate}

\begin{figure}[H]
	\centering%
	\begin{subfigure}[b]{0.5\linewidth}
		\centering
		\includegraphics[width=0.9\linewidth,height=8cm,keepaspectratio]{img/p01-02-linux-imagea.png} %CHKTEX 8
		\caption{Selección de imagen a grabar}
		\label{fig:write-image-linux-a} %CHKTEX 24
	\end{subfigure}%
	\begin{subfigure}[b]{0.5\linewidth}
		\centering
		\includegraphics[width=0.9\linewidth,height=8cm,keepaspectratio]{img/p01-02-linux-imageb.png} %CHKTEX 8
		\caption{Listo para grabar la imagen}
		\label{fig:write-image-linux-b} %CHKTEX 24
	\end{subfigure}
	\caption{Escritura de la imagen IMG de Raspbian en la microSD con Etcher}%
	\label{fig:write-image-linux} %CHKTEX 24
\end{figure}

\begin{importantbox}{Advertencia}
Tenga cuidado de no seleccionar su disco duro.
Este proceso destruirá toda la información del medio seleccionado.
\end{importantbox}

Cuando la escritura de datos en la microSD termine, notará que ésta ha sido reparticionada en \texttt{boot} (partición de arranque tipo \textit{FAT32}) y \texttt{rootfs} (partición raíz tipo \textit{EXT4}).

Si no cuenta con monitor, teclado o mouse, consulte el \Cref{sec:annex1}.
De otro modo, desmonte la tarjeta microSD e insértela en la Raspberry Pi.

% %% %%%%%%%%%%%%%%%%%%%%%%%%%%%%%%%%%%%%%%%%%%%%%%%%%%%%%%%%%%%%%%%%%%
%
% Step 2b: Windows
%
% %% %%%%%%%%%%%%%%%%%%%%%%%%%%%%%%%%%%%%%%%%%%%%%%%%%%%%%%%%%%%%%%%%%%
\subsubsection{Escribir imagen usando Windows}%
Descargue e instale \href{https://sourceforge.net/projects/win32diskimager/}{Win32 Disk Imager}.
Si no ha descmprimido la imagen de Raspbian, proceda a hacerlo con un programa que descomprima archivos Zip como \href{https://www.7-zip.org/download.html}{7zip} (Windows 7 y posterior deberí soportar descompresión zip).

A continuación, siga los pasos de Win32 Disk Imager para grabar la imagen (véase \Cref{fig:write-image-windows})
\begin{enumerate}[noitemsep]
	\item Seleccione la imagen IMG de Raspbian en \emph{Image File}
	\item Seleccione la unidad en la cual se grabará la imagen de Raspbian (letra asignada a la microSD) en \emph{Device}
	\item De click en \emph{Write} para iniciar el proceso de escritura.
	\item Win32 Disk Imager mostrará una advertencia, de click en \emph{Yes} para continuar.
\end{enumerate}

\begin{figure}[H]
	\centering%
	\begin{subfigure}[b]{0.5\linewidth}
		\centering
		\includegraphics[width=0.9\linewidth,height=8cm,keepaspectratio]{img/p01-02-windows-imagea.png} %CHKTEX 8
		\caption{Selección de imagen a grabar}
		\label{fig:write-image-windows-a} %CHKTEX 24
	\end{subfigure}%
	\begin{subfigure}[b]{0.5\linewidth}
		\centering
		\includegraphics[width=0.9\linewidth,height=8cm,keepaspectratio]{img/p01-02-windows-imageb.png} %CHKTEX 8
		\caption{Listo para grabar la imagen}
		\label{fig:write-image-windows-b} %CHKTEX 24
	\end{subfigure}
	\caption{Escritura de la imagen IMG de Raspbian en la microSD con Win32 Disk Imager}%
	\label{fig:write-image-windows} %CHKTEX 24
\end{figure}

Cuando la escritura de datos en la microSD termine, Windows asignará una letra al nuevo volúmen de datos de la microSD, llamada \texttt{boot}.
Desmonte la tarjeta microSD e insértela en la Raspberry Pi.

% %% %%%%%%%%%%%%%%%%%%%%%%%%%%%%%%%%%%%%%%%%%%%%%%%%%%%%%%%%%%%%%%%%%%
%
% Step 3
%
% %% %%%%%%%%%%%%%%%%%%%%%%%%%%%%%%%%%%%%%%%%%%%%%%%%%%%%%%%%%%%%%%%%%%
\clearpage
\subsection{Paso 3: Configurar Raspbian}%
Para configuar Raspbian, la Raspberry Pi deberá tener una tarjeta de memoria microSD con una imagen de Raspbian precargada y estar conectada a un monitor vía el puerto HDMI incorporado.
Además, se precisa de un teclado y apuntador (mouse) USB para realizar el proceso de configuración.
A continuación, conecte la Raspberry Pi y espere entre 1 y 3 minutos a que el sistema operativo cargue.

\begin{figure}[H]
	\centering
	\includegraphics[width=0.9\linewidth,height=8cm,keepaspectratio]{img/p01-03-desktop.png} %CHKTEX 8
	\caption{Escritorio de Raspbian}
	\label{fig:raspbian-desktop} %CHKTEX 24
\end{figure}

Una vez terminado el proceso de arranque, siga el asistente para configurar Raspbian, tal como se muestra en la \Cref{fig:setup-wizard}.


\begin{figure}[H]
	\centering%
	\begin{subfigure}[b]{0.33\linewidth}
		\centering
		\includegraphics[width=0.9\linewidth,height=33mm,keepaspectratio]{img/p01-03-wizard-1.png} %CHKTEX 8
		\caption{Inicio del\\Asistente de Conficuración\\~}
		\label{fig:setup-wizard-step-1} %CHKTEX 24
	\end{subfigure}%
	\begin{subfigure}[b]{0.33\linewidth}
		\centering
		\includegraphics[width=0.9\linewidth,height=33mm,keepaspectratio]{img/p01-03-wizard-2.png} %CHKTEX 8
		\caption{Selección de idioma\\y datos de localización\\~}
		\label{fig:setup-wizard-step-2} %CHKTEX 24
	\end{subfigure}%
	\begin{subfigure}[b]{0.33\linewidth}
		\centering
		\includegraphics[width=0.9\linewidth,height=33mm,keepaspectratio]{img/p01-03-wizard-3.png} %CHKTEX 8
		\caption{Cambio de contraseña\\del usuario principal \textit{pi}\\~}
		\label{fig:setup-wizard-step-3} %CHKTEX 24
	\end{subfigure}\\
	\begin{subfigure}[b]{0.33\linewidth}
		\centering
		\includegraphics[width=0.9\linewidth,height=33mm,keepaspectratio]{img/p01-03-wizard-4.png} %CHKTEX 8
		\caption{Calibración de pantalla\\~}
		\label{fig:setup-wizard-step-4} %CHKTEX 24
	\end{subfigure}%
	\begin{subfigure}[b]{0.33\linewidth}
		\centering
		\includegraphics[width=0.9\linewidth,height=33mm,keepaspectratio]{img/p01-03-wizard-5.png} %CHKTEX 8
		\caption{Actualización de Software\\Requiere conexión a internet}
		\label{fig:setup-wizard-step-5} %CHKTEX 24
	\end{subfigure}%
	\begin{subfigure}[b]{0.33\linewidth}
		\centering
		\includegraphics[width=0.9\linewidth,height=33mm,keepaspectratio]{img/p01-03-wizard-6.png} %CHKTEX 8
		\caption{Fin del asistente y reinicio\\~}
		\label{fig:setup-wizard-step-6} %CHKTEX 24
	\end{subfigure}\\
	\caption{Asistente de configuración de Raspbian}%
	\label{fig:setup-wizard} %CHKTEX 24
\end{figure}

Finalmente, se aconseja expandir la partición de Raspbian a su máxima capacidad.
Para ello, ejecute en una terminal la herramienta de configuración \texttt{sudo raspi-config} (véase \Cref{fig:raspberry-config-tool}), seleccione la opción 7, \textit{Opciones avanzadas} (Avanced Options), y luego la opción A1, \textit{Extender el sistema de archivos} (Expand Filesystem).


% %% %%%%%%%%%%%%%%%%%%%%%%%%%%%%%%%%%%%%%%%%%%%%%%%%%%%%%%%%%%%%%%%%%%
%
% Anex
%
% %% %%%%%%%%%%%%%%%%%%%%%%%%%%%%%%%%%%%%%%%%%%%%%%%%%%%%%%%%%%%%%%%%%%
\appendix%
% \clearpage%
\section{Configuración de la red alámbrica \texttt{eth0} con IP estática}%
\label{sec:setup-eth0}

\begin{importantbox}{Importante}
Configurar la red alámbrica requiere modificar un archivo en la partición \texttt{rootfs}, por lo que sólo puede hacerse desde un sistema basado en Unix con soporte para sistema de archivos \texttt{ext4} o directamente en la Raspberry Pi con monitor y teclado.
\end{importantbox}

Para acceder al archivo de configuración es necesario acceder a la partición \texttt{rootfs}.
Supóngase que la microSD está asociada a \texttt{/dev/mmcblk0}, las paticiones \texttt{boot} y \texttt{rootfs} serán entonces \texttt{/dev/mmcblk0p1} y \texttt{/dev/mmcblk0p2} respectivamente.
El proceso de montado de la partición \texttt{rootfs} es como sigue:

\begin{Verbatim}[fontsize=\footnotesize]
$ mkdir /media/[user]/rootfs
# mount /dev/mmcblk0p2 /media/[user]/rootfs
$ cd /media/[user]/rootfs
\end{Verbatim}

\noindent donde \texttt{[user]} es el nombre de usuario que utiliza en su sistema Linux.
Si no conoce el nombre de dispositivo de su $\mu$SD pruebe con el comando \texttt{lsblk}.

Nótese que los comandos marcados con \texttt{\#} deben ejecutarse con permisos de super usuario (i.e.~\emph{root}, o mediante \texttt{sudo} en Ubuntu).

Siga los siguientes pasos:

\begin{enumerate}
	\item Abra el archivo \texttt{dhcpcd.conf} localizado en el directorio \texttt{etc/} de la partición \texttt{rootfs} de la $\mu$SD con un editor de texto.
	Si trabaja directamente en la Raspberry Pi la ruta es \texttt{/etc/dhcpcd.conf}.

	\item Localice la línea \texttt{\# Example static IP configuration:}. Verá un código similar al siguiente:

\begin{Verbatim}[fontsize=\footnotesize]
# Example static IP configuration:
#interface eth0
#static ip_address=192.168.0.10/24
#static ip6_address=fd51:42f8:caae:d92e::ff/64
#static routers=192.168.0.1
#static domain_name_servers=192.168.0.1 8.8.8.8 fd51:42f8:caae:d92e::1
\end{Verbatim}

	\item Edite la configuración anterior de la siguiente manera:
\begin{Verbatim}[fontsize=\footnotesize]
# Example static IP configuration:
interface eth0
static ip_address=192.168.0.110/24
static routers=192.168.0.254
static domain_name_servers=132.248.204.1 132.248.10.2 8.8.8.8
\end{Verbatim}
	donde
	\begin{itemize}
		\item \texttt{static ip\_address} es la dirección IP de la Raspberry Pi, seguido de el código de la máscara de sub red (24 = 255.255.255.0).
		\item \texttt{static routers} es la dirección IP del punto de acceso de la red local, su puerta a la Internet.
		\item \texttt{static domain\_name\_servers} es la dirección de los servidores de resolución de nombres para acceder a la Internet.
	\end{itemize}
\end{enumerate}

\begin{greenbox}{Importante}
Coordínese con sus compañeros de clase para que cada Raspberry Pi tenga una dirección IP diferente, ya que de otro modo habrá una colisión y ninguna tendrá acceso a la red.

Se recomienda usar la IP de la computadora cuyo cable utiliza.
\end{greenbox}

\noindent
Finalmente reinicie la Raspberry Pi.



\section{Configuración de la red inalámbrica sin monitor}%
\label{sec:setup-wifi}
Para configurar la red inalámbrica sin un monitor necesita crear un archivo  llamado \texttt{wpa\_supplicant.conf} en la raíz de la partición \texttt{boot}.
Supóngase que la microSD está asociada a \texttt{/dev/mmcblk0}, las paticiones \texttt{boot} y \texttt{rootfs} serán entonces \texttt{/dev/mmcblk0p1} y \texttt{/dev/mmcblk0p2} respectivamente.
El proceso de creación del archivo \texttt{wpa\_supplicant.conf} en la raíz de la partición \texttt{boot} es como sigue:

\begin{Verbatim}[fontsize=\footnotesize]
$ mkdir /media/[user]/boot
# mount /dev/mmcblk0p1 /media/[user]/boot
$ touch /media/[user]/boot/wpa_supplicant.conf
\end{Verbatim}

\noindent donde \texttt{[user]} es el nombre de usuario que utiliza en su sistema Linux.
Si no conoce el nombre de dispositivo de su $\mu$SD pruebe con el comando \texttt{lsblk}.

Nótese que los comandos marcados con \texttt{\#} deben ejecutarse con permisos de super usuario (i.e.~\emph{root}, o mediante \texttt{sudo} en Ubuntu).

A continuación proceda a editar el archivo \texttt{wpa\_supplicant.conf}

\begin{Verbatim}[fontsize=\footnotesize]
ctrl_interface=DIR=/var/run/wpa_supplicant GROUP=netdev
update_config=1
country=MX

network={
    ssid="<NOMBRE DE LA RED INALÁMBRICA>"
    psk="<CONTRASEÑA>"
    key_mgmt=WPA-PSK
}
\end{Verbatim}

O, si la red no tiene contraseña:

\begin{Verbatim}[fontsize=\footnotesize]
ctrl_interface=DIR=/var/run/wpa_supplicant GROUP=netdev
update_config=1
country=MX

network={
    ssid="<NOMBRE DE LA RED INALÁMBRICA>"
    key_mgmt=NONE
}
\end{Verbatim}

Si su red inalámbrica no tiene servidor DHCP requerirá de una IP estática.
Para esto es necesario modificar el archivo \texttt{dhcpcd.conf} tal como se describe en el \Cref{sec:setup-eth0}.

Siga los siguientes pasos:

\begin{enumerate}
	\item Abra el archivo \texttt{dhcpcd.conf} localizado en el directorio \texttt{etc/} de la partición \texttt{rootfs} de la $\mu$SD con un editor de texto.

	\item Localice la línea \texttt{\# Example static IP configuration:}. Verá un código similar al siguiente:

\begin{Verbatim}[fontsize=\footnotesize]
# Example static IP configuration:
#interface eth0
#static ip_address=192.168.0.10/24
#static ip6_address=fd51:42f8:caae:d92e::ff/64
#static routers=192.168.0.1
#static domain_name_servers=192.168.0.1 8.8.8.8 fd51:42f8:caae:d92e::1
\end{Verbatim}

	\item Copie el bloque de texto anterior y péguelo justo debajo, editando la configuración de la siguiente manera:
\begin{Verbatim}[fontsize=\footnotesize]
# Example static IP configuration:
interface wlan0
static ip_address=192.168.0.200/24
static routers=192.168.0.254
static domain_name_servers=132.248.204.1 132.248.10.2 8.8.8.8
\end{Verbatim}
	donde
	\begin{itemize}
		\item \texttt{static ip\_address} es la dirección IP de la Raspberry Pi, seguido de el código de la máscara de sub red (24 = 255.255.255.0).
		\item \texttt{static routers} es la dirección IP del punto de acceso de la red local, su puerta a la Internet.
		\item \texttt{static domain\_name\_servers} es la dirección de los servidores de resolución de nombres para acceder a la Internet.
	\end{itemize}
\end{enumerate}

\begin{greenbox}{Importante}
Coordínese con sus compañeros de clase para que cada Raspberry Pi tenga una dirección IP diferente, ya que de otro modo habrá una colisión y ninguna tendrá acceso a la red.
\end{greenbox}

Una vez completado este proceso, desmonte la memoria microSD e insértela en la Raspberry Pi.

\section{Instalación de Raspbian via SSH}%
\label{sec:annex1}
En caso de que no se cuente con un monitor, es posible instalar Raspbian vía SSH. %CHKTEX 13
Para ello es necesario habilitar conexiones vía SSH ya que estas se encuentran deshabilitadas por defecto.



\subsection{Habiliar conexiones SSH}%
\label{sec:annex1-ssh-enable}
Para habilitar SSH, se necesita crear un par de archivos:
\begin{enumerate*}[label=\roman*\rpar]
	\item un archivo vacío llamado \texttt{SSH}
	y
	\item un archivo llamado \texttt{userconf.txt} con el usuario y contraseña para conexiones SSH; % CHKTEX 13
\end{enumerate*}
ambos en la raíz de la partición \texttt{boot}.

Supóngase que la microSD está asociada a \texttt{/dev/mmcblk0}, las paticiones \texttt{boot} y \texttt{rootfs} serán entonces \texttt{/dev/mmcblk0p1} y \texttt{/dev/mmcblk0p2} respectivamente.
El proceso de creación del archivo \texttt{SSH} en la raíz de la partición \texttt{boot} es como sigue:

\begin{Verbatim}[fontsize=\footnotesize]
$ mkdir /media/[user]/boot
# mount /dev/mmcblk0p1 /media/[user]/boot
$ touch /media/[user]/boot/SSH
\end{Verbatim}

\noindent donde \texttt{[user]} es el nombre de usuario que utiliza en su sistema Linux.
Si no conoce el nombre de dispositivo de su $\mu$SD pruebe con el comando \texttt{lsblk}.

Nótese que los comandos marcados con \texttt{\#} deben ejecutarse con permisos de super usuario (i.e.~\emph{root}, o mediante \texttt{sudo} en Ubuntu).

A continuación use \texttt{nano} para crear el archivo \texttt{userconf.txt} y escriba el siguiente texto:
\begin{Verbatim}[fontsize=\scriptsize]
pi:$6$9Qp0F2Al16JhvGmB$JLOIQjLdBIgRAbI6iMl.bylomaIshgHlNTB51oHUSNbTW3D5l6hEnPr6HEBtMo/0IKwlGkc7.FlFOhaMwPKB1/
\end{Verbatim}

Esta cadena habilitará iniciar sesión con el usuario \texttt{pi} y la contraseña \texttt{raspberry}.

Teclear la cadena encriptada es un proceso tedioso y propenso a errores, por lo que una forma más simple de hacer esto es ejecutar el comando

\begin{Verbatim}[fontsize=\footnotesize]
echo "pi:$(openssl passwd -6)" > /media/[user]/boot/userconf.txt
\end{Verbatim}

Con lo que podrá definir cualquier usuario y contraseña que desee.

Una vez completado este proceso, desmonte la memoria microSD e insértela en la Raspberry Pi.

\subsection{Configurar Raspbian vía SSH}
Para configuar Raspbian via SSH, la Raspberry Pi deberá estar conectada a la red local vía un cable Ethernet y tener una tarjeta de memoria microSD con una imagen de Raspbian precargada.

A continuación, conecte la Raspberry Pi y espere entre 1 y 3 minutos a que el sistema operativo cargue.
Utilice un escaner de IP o consulte su enrutador para conocer la IP asignada a la Raspberry Pi.

\begin{figure}[H]
	\centering%
	\includegraphics[width=9cm]{img/p01-appendix01.png} %CHKTEX 8
	\caption{Dirección IP de una Raspbery Pi}
	\label{fig:raspberry-ip} %CHKTEX 24
\end{figure}

Una vez conozca la dirección IP de la Raspberry Pi, conéctese a ésta mediante SSH. %CHKTEX 13
Secure Shell le advertirá que no puede verificar la autenticidad del certificado, por lo que pedirá que confirme la conexión tecleando \texttt{yes}, como se muestra a continuación.

\begin{Verbatim}[fontsize=\footnotesize]
$ ssh pi@192.168.1.126

    The authenticity of host '192.168.1.126 (192.168.1.126)' can't be established.
    ECDSA key fingerprint is SHA256:1nrpQeTIb+Gzg4aIJ0WE+V0aLUQgDnQbxOGraWf0Kso.
    Are you sure you want to continue connecting (yes/no)?
\end{Verbatim}

Teclee \texttt{yes} y presione Enter.
De inmediato se le solicitará la contraseña

\begin{Verbatim}[fontsize=\footnotesize]
    Warning: Permanently added '192.168.1.126' (ECDSA) to the list of known hosts.

    pi@192.168.1.126's password:
\end{Verbatim}

Teclee \texttt{raspberry}, la contraseña por default en Raspbian, y presione Enter.
Se concretará la conexión.

\begin{Verbatim}[fontsize=\footnotesize]
    Linux raspberrypi 4.19.75-v7+ #1270 SMP Tue Sep 24 18:45:11 BST 2019 armv7l

    The programs included with the Debian GNU/Linux system are free software;
    the exact distribution terms for each program are described in the
    individual files in /usr/share/doc/*/copyright.

    Debian GNU/Linux comes with ABSOLUTELY NO WARRANTY, to the extent
    permitted by applicable law.
    Last login: Thu Feb  6 18:28:53 2020

    SSH is enabled and the default password for the 'pi' user has not been changed.
    This is a security risk - please login as the 'pi' user and type 'passwd' to set a new password.
\end{Verbatim}

Finalmente, para configurar su Raspbian, ejecute \texttt{sudo raspi-config}  para iniciar la herramienta de configuración (véase \Cref{fig:raspberry-config-tool}).
Se aconseja definir el idioma, localización, y cambiar la contraseña del usuario por defecto \textit{pi}.

\begin{figure}[H]
	\centering%
	\includegraphics[width=0.8\linewidth,height=5cm,keepaspectratio]{img/p01-appendix02.png} %CHKTEX 8
	\caption{Herramienta de configuración de Raspbian}
	\label{fig:raspberry-config-tool} %CHKTEX 24
\end{figure}

\end{document}


