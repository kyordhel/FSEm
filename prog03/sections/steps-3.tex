% %% %%%%%%%%%%%%%%%%%%%%%%%%%%%%%%%%%%%%%%%%%%%%%%%%%%%%%%%%%%
% step-3.tex
%
% Author:  Mauricio Matamoros
% License: MIT
%
% %% %%%%%%%%%%%%%%%%%%%%%%%%%%%%%%%%%%%%%%%%%%%%%%%%%%%%%%%%%%

%!TEX ROOT=../main.tex
%!TEX ROOT=../references.bib

% CHKTEX-FILE 1
% CHKTEX-FILE 46

\subsection{Paso 3: Lectura del sensor LM35}%
\label{sec:step3}

Para leer la temperatura se necesita convertir los valores discretos leídos por el ADC en valores de temperatura.
Esto se puede realizar mediante un simple análisis debido a la linearidad del LM35.
En un ADC típico se obtienen dos lecturas: V\textsubscript{OUT+} y V\textsubscript{OUT-}, de las cuales la segunda es la referencia del LM35 y por lo tanto, la diferencia entre estos voltajes será proporcional a la temperatura en escala centígrada.
Esto expresado matemáticamente es:

\begin{equation*}
	T[^{o}C] \propto V_{diff} = V_{OUT+}-V_{OUT-}
\end{equation*}

\noindent o bien

\begin{equation*}
	T[^{o}C] = k \times V_{diff} = k \times \big( V_{OUT+}-V_{OUT-} \big)
\end{equation*}

\noindent lo que implica que en $T = 0^{o}C; V_{OUT+}=V_{OUT-} \rightarrow V_{diff} = 0$

\medskip
Es necesario entonces calcular la constante de proporcionalidad $k$.
Sabemos que el ADC entregará lecturas de 0 a 1024 para los voltajes registradoes entre \GND y \textsc{Aref}.
Por ejemplo, con el LM35 en una configuración básica ($V_{OUT-}$ en \GND), $V_{Ref+} = 2.72V$ y considerando que que $1^{o}C = 0.01V$ se tendría:
\begin{align*}
	T[^{o}C] &= V_{diff} \times \frac{2.72[V]}{ 1024 \times 0.01[\tfrac{V}{^{o}C}] }  \\
	         &= V_{diff} \times \frac{2.72}{ 10.24 }[^{o}C]  \\
	         &= 0.266 V_{diff}[^{o}C]  \\
\end{align*}

\noindent o bien, generalizando para todo voltaje de referencia:
\begin{equation*}
	T[^{o}C] = V_{diff} \times \frac{V_{Ref+}}{ 10.24 }[^{o}C]
\end{equation*}

Esta fórmula de conversión, o su equivalente según el divisor de voltaje utilizado, deberá programarse en la Raspberry Pi que adquiera los valores discretos de temperatura del sensor.
