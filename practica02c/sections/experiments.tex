% %% %%%%%%%%%%%%%%%%%%%%%%%%%%%%%%%%%%%%%%%%%%%%%%%%%%%%%%%%%%
% experimens.tex
%
% Author:  Mauricio Matamoros
% License: MIT
%
% %% %%%%%%%%%%%%%%%%%%%%%%%%%%%%%%%%%%%%%%%%%%%%%%%%%%%%%%%%%%

%!TEX root = ../practica.tex
%!TEX root = ../references.bib

% CHKTEX-FILE 1
% CHKTEX-FILE 13
% CHKTEX-FILE 46

\section{Experimentos}%
\label{sec:experiments}

\begin{enumerate}
	\item{} [6pt] Modifique el código de la \cref{sec:step4} para que el brillo del led centinela se vea atenuado.
	\item{} [4pt] Modifique el código de la \cref{sec:step5} para que la temperatura se reporte tanto en grados Farenheit como en Centígrados.
% 	\item{} [4pt] Modifique el código de la \cref{sec:step4} la Raspberry Pi grafique el histórico de temperaturas registradas, leyendo los valores almacenados e ingresados en la bitácora.
	\item{} [+3pt] Con base en lo aprendido, modifique el código de la \cref{sec:step4} atenúe el brillo del led usando un PWM en lugar de retardos.
	\item{} [+2pt] Usando Thonny cargue un archivo \code{main.py} en el RP2040 para que su programa se inicie de forma automática y autónoma.
\end{enumerate}

\noindent
\textbf{Nota:} Anexe en su reporte las imágenes o enlaces a videos correspondientes a cada experimento como evidencia de su ejecución.
