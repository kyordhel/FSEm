% %% %%%%%%%%%%%%%%%%%%%%%%%%%%%%%%%%%%%%%%%%%%%%%%%%%%%%%%%%%%
% material.tex
%
% Author:  Mauricio Matamoros
% License: MIT
%
% %% %%%%%%%%%%%%%%%%%%%%%%%%%%%%%%%%%%%%%%%%%%%%%%%%%%%%%%%%%%

%!TEX root = ../practica.tex
%!TEX root = ../references.bib

% CHKTEX-FILE 1
% CHKTEX-FILE 13
% CHKTEX-FILE 46

\section{Material}%
\label{sec:material}
Se asume que el alumno cuenta con un una computadora con sistema operativo basado en Linux (se recomienda Xubuntu) e interprete de Python instalado.

\begin{itemize}[noitemsep]
	\item 1 Tarjeta microcontroladora Raspberry Pi Pico o UNIT DualMCU RP2040~+~ESP32 (recomendado)
	\item 1 cable USB-C con soporte para datos\footnotemark{}
	% \item  resistencia de 330$\Omega$
	% \item 1 protoboard o circuito impreso equivalente
	% \item 1 fuente de alimentación regulada a 5V y al menos 1 amperios de salida
	\item Cables y conectores varios
\end{itemize}
\footnotetext{
El mercado ofrece un gran número de cables económicos tipo USB-C, especialmente para carga rápida de celulares, tablets y laptops.
Estos cables, sin embargo, han sido diseñados para la transmisión de potencia y no de datos, por lo que pueden no contar con las líneas de transmisión de datos para programar un microcontrolador.
Verifique que su cable USB-C cumple con la especificación de transporte de datos y no sólo con la de transporte de potencia.} %chktex 42
