% %% %%%%%%%%%%%%%%%%%%%%%%%%%%%%%%%%%%%%%%%%%%%%%%%%%%%%%%%%%%
% step-2.tex
%
% Author:  Mauricio Matamoros
% License: MIT
%
% %% %%%%%%%%%%%%%%%%%%%%%%%%%%%%%%%%%%%%%%%%%%%%%%%%%%%%%%%%%%

%!TEX root = ../practica.tex
%!TEX root = ../references.bib

% CHKTEX-FILE 1
% CHKTEX-FILE 13
% CHKTEX-FILE 46

\subsection{Paso 2: Instalar MicroPython en el RP2040}%
\label{sec:step2}
Antes de proceder, verifique que su cable adaptador USB-C soporta transmisión de datos.

\medskip{}

El RP2040 viene precargado con un microfirmware especial que permite dos modos de arranque:
\begin{enumerate*}[label=\roman*\rpar]
	\item el arranque normal en modo microcontrolador que comenzará la ejecución de las microinstrucciones grabadas en la memoria \textsc{Flash} de programa
	y
	\item el arranque en modo carga, que inicializará al microcontrolador como si fuera un dispositivo de almacenamiento masivo (ej.~una memoria USB) donde se puede copiar el firmware compilado para el microcrontrolador.
\end{enumerate*}
Lejos atrás quedaron los tiempos en que se requería de un dispositivo programador especial o de software complejo para interactuar con el microcontrolador.

Para arrancar el microcontrolador RP2040 en modo carga, basta con presionar el botón \textsc{Bootsel} en la tarjeta antes de energizar el microcontrolador conectándolo a una computadora, tal como muestra la \Cref{fig:boot-btn}.
Se requiere de un cable adaptador USB tipo C para este propósito.

\begin{figure}[H]
	\centering%
	\begin{subfigure}[b]{0.45\textwidth}
		\centering%
		\includegraphics[width=\columnwidth,height=8cm,keepaspectratio]{img/dualMCU-boot.png} %CHKTEX 8
		\caption{Boton BOOT en la \textsc{Unit} DualMCU}
		\label{fig:dualMCU-boot} %CHKTEX 24
	\end{subfigure}
	\hfill
	\begin{subfigure}[b]{0.45\textwidth}
		\centering%
		\includegraphics[width=\columnwidth,height=8cm,keepaspectratio]{img/pico-boot.png} %CHKTEX 8
		\caption{Boton BOOTSEL en la Raspberry Pi Pico}
		\label{fig:pico-boot} %CHKTEX 24
	\end{subfigure}
	\caption{Descarga de la última versión de MicroPython}
	\label{fig:boot-btn} %CHKTEX 24
\end{figure}

\begin{greenbox}{Importante}
	Si cuenta con una tarjeta una tarjeta \textsc{Unit} DualMCU, verifique que el interruptor USB-Selector esté activado en modo RP2040.
	Esta tarjeta cuenta, además, con un botón \textsc{Rst} que sirve para reiniciar el RP2040 evitando el desgaste innecesario del conector USB.
\end{greenbox}

\noindent
Para grabar MicroPython en el RP2040 siga los siguientes pasos:

\begin{enumerate}[nosep]
	\item Presione el botón BOOT/\textsc{Bootsel} de la tarjeta.
	\item Sin soltar el botón, conecte la tarjeta a la computadora usando el cable USB-C.
	\item Espere dos segundos antes de soltar el botón BOOT/\textsc{Bootsel}.
	\item Espere unos segundos a que la computadora reconozca el dispositivo.\footnotemark{}
	El volumen tendrá un nombre parecido a RPI-RP2.
	\item Copie el archivo de firmware, por ejemplo el \code{rp2-pico-20230426-v1.20.0.uf2} al dispositivo de almacenamiento masivo (la tarjeta).
\end{enumerate}
\footnotetext{
	Si la computadora no ha reconocido la tarjeta como un dispositivo de almacenamiento masivo luego de aproximadamente medio minuto, verifique que su cable tenga soporte para transporte de datos.
}

\medskip{}

\noindent
Tan pronto como la tarjeta detecte que se ha cargado un nuevo firmware, ésta se reiniciará automáticamente.
