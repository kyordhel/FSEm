% %% %%%%%%%%%%%%%%%%%%%%%%%%%%%%%%%%%%%%%%%%%%%%%%%%%%%%%%%%%%
% step-1.tex
%
% Author:  Mauricio Matamoros
% License: MIT
%
% %% %%%%%%%%%%%%%%%%%%%%%%%%%%%%%%%%%%%%%%%%%%%%%%%%%%%%%%%%%%

%!TEX root = ../practica.tex
%!TEX root = ../references.bib

% CHKTEX-FILE 1
% CHKTEX-FILE 13
% CHKTEX-FILE 46

\subsection{Paso 1: Descargar MicroPython}%
\label{sec:step1}

Ingrese a \url{https://micropython.org/download/?port=rp2} y seleccione el modelo de su tarjeta controladora.
Si no está seguro, simplemente de click en la imagen correspondiente a la Raspberry Pi Pico (\url{https://micropython.org/download/rp2-pico/}).
A continuación descargue último firmware disponible de MicroPython, por ejemplo la \href{https://micropython.org/resources/firmware/rp2-pico-20230426-v1.20.0.uf2}{versión 1.20.0}.
Deberá ser un archivo extensión \code{uf2}.

\begin{figure}[H]
	\centering%
	\begin{subfigure}[b]{0.45\textwidth}
		\centering%
		\includegraphics[width=\columnwidth,height=8cm,keepaspectratio]{img/upy-dl-01.png} %CHKTEX 8
		\caption{Selección de la Raspberry Pi Pico}
		\label{fig:upy-dl-a} %CHKTEX 24
	\end{subfigure}
	\hfill
	\begin{subfigure}[b]{0.45\textwidth}
		\centering%
		\includegraphics[width=\columnwidth,height=8cm,keepaspectratio]{img/upy-dl-02.png} %CHKTEX 8
		\caption{Firmwares disponibles}
		\label{fig:upy-dl-b} %CHKTEX 24
	\end{subfigure}
	\caption{Descarga de la última versión de MicroPython}
	\label{fig:upy-dl} %CHKTEX 24
\end{figure}

\begin{greenbox}{Importante}
	Si tiene una Raspberry Pi Pico-W descargue el firmware para la Pico-W.
	De otro modo no podrá usar la tarjeta inalámbrica
\end{greenbox}

