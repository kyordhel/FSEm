% %% %%%%%%%%%%%%%%%%%%%%%%%%%%%%%%%%%%%%%%%%%%%%%%%%%%%%%%%%%%
% intro-lm35.tex
%
% Author:  Mauricio Matamoros
% License: MIT
%
% %% %%%%%%%%%%%%%%%%%%%%%%%%%%%%%%%%%%%%%%%%%%%%%%%%%%%%%%%%%%

%!TEX root = ../practica.tex
%!TEX root = ../references.bib


\subsection{Thonny}%
\label{sec:intro-thonny}

Thonny es un IDE\footnote{Entorno de desarrollo integrado, por sus siglas en inglés \emph{Integrated Development Environment}}
de código abierto
para Python desarrollado por la Universidad de Tartu, Estonia, pensado para ser amigable con el principiante.
Además de las capacidades estándar de construcción y ejecución de programas, cuenta con un soporte extendido para animación de programas,
permitiendo ilustrar los conceptos de variables, control de flujo, evaluacíón de expresiones, llamada a funciones, recursión, referencias y montículos, objetos (incluyendo clases y funciones como valores), datos compuestos (listas, diccionarios y conjuntos) y operaciones de entrada/salida sobre archivos~\Citep{annamaa2015a,annamaa2015b}.

Thonny puede llevar registro de las acciones del usuario con suficiente detalle para reproducir el proceso de elaboración de un programa, permitiéndole a los estudiantes elegir el nivel de detalle con el que desean analizar la reproducción~\Citep{annamaa2015a,annamaa2015b}.

Una de las ventajas de Thonny sobre cualquier otro editor de Python es que, a partir de la versión 4.0, incluye soporte para interactuar con varios  microcontroladores como el RP2040, el ESP32 y el ESP8266; todos de amplia difusión en el mercado.
Además, Thonny permite realizar operaciones en el sistema de archivos virtual del RP2040 (MicroPython convierte el RP2040 en una suerte de memoria USB de unos cuantos kB).
Para este fin, basta con contectar la tarjeta controladora a la PC, iniciar Thonny y seleccionar las opciones \emph{View} y \emph{Files}.
De igual manera Thony permite copiar archivos de y a la tarjeta controladora para ser utilizados por MicroPython~\cite{bell2022MicroPython,bell2022}.

\begin{figure}[H]
	\centering%
	\includegraphics[width=\columnwidth,height=8cm,keepaspectratio]{img/thonny-files.png} %CHKTEX 8
	\caption{Thonny permite visualizar, crear, copiar, cortar, pegar y, en general, manupular archivos dentro del RP2040.}
	\label{fig:thonny} %CHKTEX 24
\end{figure}
