% %% %%%%%%%%%%%%%%%%%%%%%%%%%%%%%%%%%%%%%%%%%%%%%%%%%%%%%%%%%%
% step-4.tex
%
% Author:  Mauricio Matamoros
% License: MIT
%
% %% %%%%%%%%%%%%%%%%%%%%%%%%%%%%%%%%%%%%%%%%%%%%%%%%%%%%%%%%%%

%!TEX root = ../practica.tex
%!TEX root = ../references.bib

% CHKTEX-FILE 1
% CHKTEX-FILE 13
% CHKTEX-FILE 46

\subsection{Paso 5: Temperatura del RP2040}%
\label{sec:step5}

La \cref{eqn:v-to-temp} tiene un problema fundamental: un convertidor ADC entrega valores discretos enteros que corresponden de forma proporcional y lineal al voltaje sensado entre $V_\text{Ref-}$ y $V_\text{Ref+}$.
En el caso del RP2040 $V_\text{Ref-} = \text{\textsc{Gnd}} = 0\text{V}$ y $V_\text{Ref+} = \text{\textsc{Vcc}} = 3.3\text{V}$ y, dado que la precisión del ADC es de 12bits, los valores registrados estarrían entre cero y $2^{12}-1 = 4095$.
Sin embargo, MicroPython ofrece el método \code{read_u16()} de las instancias de \code{ADC} que provee un entero no signado de 16 bits.
Así, las conversiones tendrán que tomar en cuenta un rango entre 0 y 65535.

Con esto en mente y considerando que $v = \frac{3.3V}{65535}x$, se puede reemplazar $v$ por $x$ en la \cref{eqn:v-to-temp} para obtener una fórmula de conversión directa del valor reportado del ADC a un valor de temperatura.

\noindent
Así, la \cref{eqn:v-to-temp} quedaría como:

\begin{align*}%
	\label{eqn:adc-to-temp}
	T &=%
		\Bigg( \frac{-1}{1.721\frac{\text{mV}}{\text{°C}}} \Bigg)
		\Bigg( \frac{3.3\text{V}}{65535} \cdot x \Bigg) +
		437.23\text{°C} \\
%
	  &=
		\Bigg( \frac{-1}{1.721\frac{\text{mV}}{\text{°C}}} \Bigg)
		\Bigg( \frac{3300\text{mV}}{65535} \cdot x \Bigg) +
		437.23\text{°C} \\
%
	  &=
		\frac{
			-3300\cancel{\text{mV}}
		}{
			65535 \times 1.721\frac{\cancel{\text{mV}}}{\text{°C}}
		}
		\cdot x +
		437.23\text{°C} \\
%
	  &=
		\frac{
			-3300
		}{
			112785.74\frac{1}{\text{°C}}
		}
		\cdot x +
		437.23\text{°C} \\
\end{align*}

\noindent
Lo que finalmente produce la ecuación:

\begin{equation}%
	\label{eqn:adc-to-temp}
	T = -29259\times10^{-6}x + 437.23
\end{equation}

\noindent
Esta ley se refleja en el código presentado en \cref{lst:temp}.

\medskip{}

\noindent
Ingrese el siguiente código en el editor de Thonny y ejecútelo

\lstinputlisting[%
	language=Python,
	caption={\texttt{temp.py}: --- Temperatura del RP2040},
	label={lst:temp},
	firstline=11
]{src/temp.py}


El \cref{lst:temp} configura primero el ADC para leer del canal 4 (sensor interno de temperatura) y luego define un bucle infinito con la cláusula \code{while} dentro del cual
\begin{enumerate*}[label=\roman*\rpar]
	\item se lee el valor del convertidor A/D,
	\item se convierte el valor sensado a temperatura en grados celcius
	\item se imprime la temperatura
	y, finalmente,
	\item se espera durante un segundo.
\end{enumerate*}

Por conveniencia, los códigos completos de los programas de ejemplo se encuentran en el \Cref{sec:appendix3}.
