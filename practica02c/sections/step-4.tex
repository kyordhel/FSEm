% %% %%%%%%%%%%%%%%%%%%%%%%%%%%%%%%%%%%%%%%%%%%%%%%%%%%%%%%%%%%
% step-4.tex
%
% Author:  Mauricio Matamoros
% License: MIT
%
% %% %%%%%%%%%%%%%%%%%%%%%%%%%%%%%%%%%%%%%%%%%%%%%%%%%%%%%%%%%%

%!TEX root = ../practica.tex
%!TEX root = ../references.bib

% CHKTEX-FILE 1
% CHKTEX-FILE 13
% CHKTEX-FILE 46

\subsection{Paso 4: Led parpadeante con el RP2040}%
\label{sec:step4}

Con la tarjeta controladora configurada y conectada, basta con copiar los programas al editor de Thonny y presionar el botón de ejecución (invocable vía la tecla F5).

\medskip{}

\noindent
Ingrese el siguiente código en el editor y ejecútelo

\lstinputlisting[%
	language=Python,
	caption={\texttt{blink.py}: --- Led parpadeante usando retardos},
	label={lst:blink},
	firstline=11
]{src/blink.py}


El \cref{lst:blink} configura primero el pin 25 del RP2040 como salida y luego define un bucle infinito con la cláusula \code{while} dentro del cual se alterna el estado del pin y se realiza una espera activa de 500ms.
El paquete \code{machine} contiene las clases que abstraen el acceso a todo el hardware del microcontrolador.
Es decir, los elementos de \code{machine} pueden variar entre diferentes integrados, por lo que el código de MicroPython no es necesariamente portable ni agnóstico respecto a hardware a diferencia de los programas hechos en CPython.

La línea 1 del \cref{lst:blink} importa la clase \code{Pin} del paquete \code{machine}, que posteriormente será usado en la línea 4 para crear un objeto tipo \code{Pin}.
MicroPython ocupa el paradigma de programación orientado a objetos para hacer el código más compacto y legible.
Para cambiar el estado del pin basta con llamar a las funciones \code{on()}, \code{off()} o \code{toggle()} sin necesidad de volver a especificar el número de pin o su configuración.\footnote{Para más información consúltese la documentación de MicroPython en \url{https://docs.micropython.org/en/latest/library/machine.Pin.html}}

El estudiante avezado habrá podido notar que el \cref{lst:blink} es bastante ineficiente.
La función \code{sleep_ms} se traducirá en una larga cadena de \code{NOP}s que mantendrán al procesador activo y ocupado haciendo nada, es decir, incapaz de realizar otras tareas.
Además, a menos que se programe en ensamblador, es imposible estimar de forma precisa los tiempos de ejecución del código en lenguajes de medio y alto nivel.

Para resolver este problema, el \cref{lst:blink-timer} hace uso de un temporizador o \emph{timer} de \textsc{software}\footnotemark{} para hacer parpadear el led~(véase la línea 9 del \cref{lst:blink-timer}).
En la siguiente línea el temporizador se inicializa con tres parámetros:
\begin{enumerate*}[label=\roman*\rpar]
	\item \code{freq}, que especifica la frecuencia de operación del timer en Hertz;
	\item \code{mode}, que indica si el timer debe correr sólo una vez (\code{Timer.ONE_SHOT}) o de forma periódica (\code{Timer.PERIODIC})
	y
	\item \code{callback}, que especifica el nombre la función que se llamará cuando el contador llegue a cero;
\end{enumerate*}
o \code{blik} en nuestro caso.
Para más información consúltese la documentación de MicroPython en \url{https://docs.micropython.org/en/latest/library/machine.Timer.html}
\footnotetext{
	Nótese que no se le proporciona el ID del timer a utilizar al constructor.
	Esto es debido a que la implementación actual del MicroPython para el RP2040 aún no tiene soporte para el uso de los timers físicos (Hardware) del integrado.
}

\lstinputlisting[%
	language=Python,
	caption={\texttt{blink-timer.py}: --- Led parpadeante usando temporizadores},
	label={lst:blink-timer},
	firstline=11
]{src/blink-timer.py}

Nótese que el objeto \code{Timer} se crea después de declararse la función \code{blink}.
Esto es debido a que los scripts de Python se ejecutan línea a línea, por lo que el intérprete no conocerá a la función \code{blink} si ésta se declara después.

Por conveniencia, los códigos completos de los programas de ejemplo se encuentran en los \Cref{sec:appendix1,sec:appendix2}.
