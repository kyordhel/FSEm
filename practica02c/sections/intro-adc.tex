% %% %%%%%%%%%%%%%%%%%%%%%%%%%%%%%%%%%%%%%%%%%%%%%%%%%%%%%%%%%%
% intro-adc.tex
%
% Author:  Mauricio Matamoros
% License: MIT
%
% %% %%%%%%%%%%%%%%%%%%%%%%%%%%%%%%%%%%%%%%%%%%%%%%%%%%%%%%%%%%

%!TEX root = ../practica.tex
%!TEX root = ../references.bib

% CHKTEX-FILE 1
% CHKTEX-FILE 13
% CHKTEX-FILE 46

\subsection{Convertidor Analógico---Digital y sensado de temperatura}%
\label{sec:intro-adc}
El RP2040 cuenta con un convertidor analógico-digital de 12 bits con cinco canales, cuatro de los cuales están anclados a los pines A0--A3 (GPIO 27--29) y un canal especial cuyo único propósito es medir la temperatura del integrado.

\noindent
De acuerdo con la hoja de especificaciones del RP20240:

\begin{displayquote}
	El sensor de temperatura mide el voltaje $V_{BE}$ de un diodo bipolar conectado al quinto canal del convertidor AD (AINSEL=4).
	Normalmente $V_{BE} = 0.706V$ cuando $T=27^{o}C$ con una pendiente de~$-1.721\frac{mV}{^{o}C}$~\Citep{datasheet:rp2040}.
\end{displayquote}

Con esta información, y suponiendo linearidad en la respuesta, se puede utilizar la ecuación de la recta $y = mx + b$ para extrapolar el valor de temperatura del integrado de la siguiente manera:

\begin{align*}
	y &= m x + b \\
	  &= m \left(x - x_0 \right) + y_0 \\
\end{align*}

\noindent
o, sustituyento la variable $x$ por la temperatura $T$ y $y$ por el voltaje $v$ y los valores
	$m = -1.721\frac{mV}{^{o}C}$,
	$T_{0} = 27^{o}C$ y
	$v_{0} = 0.706V$:

\begin{align*}
	v &= m \left(T - T_0 \right) + v_0 \\
%
	  &= -1.721\frac{\text{mV}}{\text{°C}} \left(T - 27\text{°C} \right) + 0.706\text{V} \\
%
	  &= \left(-1.721\frac{\text{mV}}{\text{°C}} \times T  \right) +
		\left(27\cancel{\text{°C}} \times 1.721\frac{\text{mV}}{\cancel{\text{°C}}} \right) +
		0.706\text{V} \\
%
	  &= -1.721T \frac{\text{mV}}{\text{°C}} + 46.467\text{mV} + 706\text{mV} \\
%
	  &= -1.721T \frac{\text{mV}}{\text{°C}} + 752.47\text{mV} \\
\end{align*}

\noindent
despejando a $T$:

\begin{equation*}
	\frac{v}{-1.721\frac{\text{mV}}{\text{°C}}} =
		\frac{\left(T\right)\cancel{\left(-1.721\frac{\text{mV}}{\text{°C}}\right)}}{\cancel{-1.721\frac{\text{mV}}{\text{°C}}}} +
	  	\frac{752.47\cancel{\text{mV}}}{-1.721\frac{\cancel{\text{mV}}}{\text{°C}}}
\end{equation*}

\noindent
finalmente:

\begin{equation}%
	\label{eqn:v-to-temp}
	T = \frac{-1}{1.721\frac{\text{mV}}{\text{°C}}} \cdot v + 437.23\text{°C}
\end{equation}

Así, bastará con convertir el valor reportado por el convertidor analógico-digital a su equivalente en milivoltios para obtener la temperatura del circuito integrado.
