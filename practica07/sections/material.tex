% CHKTEX-FILE 1
% CHKTEX-FILE 13
% CHKTEX-FILE 46
%!TEX root = ../practica.tex
%!TEX root = ../references.bib

%% %%%%%%%%%%%%%%%%%%%%%%%%%%%%%%%%%%%%%%%%%%%%%%%%%%%%%%%%%%%%%%%%%%
%
% Material
%
%% %%%%%%%%%%%%%%%%%%%%%%%%%%%%%%%%%%%%%%%%%%%%%%%%%%%%%%%%%%%%%%%%%%
\section{Material}%
\label{sec:material}
Se asume que el alumno cuenta con un una Raspberry Pi con sistema operativo Raspbian e interprete de Python instalado. Se aconseja encarecidamente el uso de \textit{git} como programa de control de versiones.

\begin{itemize}[noitemsep]
	\item 1 Arduino UNO o Arduino Mega
	\item 1 TRIAC BT138 o BT139
	\item 4 diodos 1N4007 o puente rectificador equivalente
	\item 1 optoacoplador MOC 3021
	\item 1 optoacoplador 4N25
	\item 1 foco incandescente (NO AHORRADOR NI LED)
	\item 1 resistencia de 68k$\Omega$, \sfrac{1}{4}Watt
	\item 1 resistencia de 10k$\Omega$, \sfrac{1}{4}Watt
	\item 2 resistencia de 4k7$\Omega$, \sfrac{1}{4}Watt
	\item 1 resistencia de  1k$\Omega$, 1Watt
	\item 2 resistencia de 470$\Omega$, \sfrac{1}{4}Watt
	\item 2 resistencia de 330$\Omega$, \sfrac{1}{4}Watt
	\item 1 LED de 5mm
	\item 1 LED ultrabrillante de 5mm
	\item 1 protoboard o circuito impreso equivalente
	\item 1 fuente de alimentación regulada a 5V y al menos 2 amperios de salida
	\item Cables y conectores varios
\end{itemize}
