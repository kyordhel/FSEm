% %% %%%%%%%%%%%%%%%%%%%%%%%%%%%%%%%%%%%%%%%%%%%%%%%%%%%%%%%%%%
% experiments.tex
%
% Author:  Mauricio Matamoros
% License: MIT
%
% %% %%%%%%%%%%%%%%%%%%%%%%%%%%%%%%%%%%%%%%%%%%%%%%%%%%%%%%%%%%
%!TEX root = ../practica.tex
%!TEX root = ../references.bib

% CHKTEX-FILE 1
% CHKTEX-FILE 13
% CHKTEX-FILE 46

\section{Experimentos}%
\label{sec:experiments}

\begin{enumerate}
	\item{} [6pt] Alambre el circuito completo y combine el código de los \Cref{sec:appendix1,sec:appendix2,sec:appendix3} para poder controlar la intensidad del brillo del foco incandecente con la Raspberry Pi usando los valores tecleados en la consola (porentaje de 0--100\% de la potencia total).

	\item{} [4pt] Modifique el código anterior para que la consola presente la potencia real modulada por el Arduino (equivalente en potencia del tiempo de encendido en milisegundos).

	\item{} [+3pt] Modifique el código del punto 2 para que el arduino pueda modular la potencia del foco incandecente entre 0\% y 100\% con una resolución máxima de 1\%. Imprima la potencia reportada por el arduino con un dígito decimal.

	\item{} [+2pt] Con base en lo aprendido, modifique el código delpunto 3 para que la Raspberry Pi sirva una página web donde se pueda modificar con un control gráfico la potencia de encendido del foco.
\end{enumerate}
