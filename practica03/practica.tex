% %% %%%%%%%%%%%%%%%%%%%%%%%%%%%%%%%%%%%%%%%%%%%%%%%%%%%%%%%%%%
% practica.tex
%
% Author:  Mauricio Matamoros
% License: MIT
%
% %% %%%%%%%%%%%%%%%%%%%%%%%%%%%%%%%%%%%%%%%%%%%%%%%%%%%%%%%%%%

% CHKTEX-FILE 1
% CHKTEX-FILE 46
\documentclass[letterpaper,10.5pt]{article}
% %% %%%%%%%%%%%%%%%%%%%%%%%%%%%%%%%%%%%%%%%%%%%%%%%%%%%%%%%%%
%
% packages.tex
%
%  Author: Mauricio Matamoros
%  Date:   2020.02.28
%
%  Contiene la lista de paquetes requeridos para generar
%  el archivo-reporte de las prácticas de laboratorio
%
% %% %%%%%%%%%%%%%%%%%%%%%%%%%%%%%%%%%%%%%%%%%%%%%%%%%%%%%%%%%
% Archivo principal de LaTeX
%!TEX root = ../reporte.tex

\usepackage[utf8]{inputenc}                  % Soporte para utf8
\usepackage[T1]{fontenc}                     % Soporte extendido de caracteres unicode
\usepackage[english,spanish,mexico]{babel}   % Define el idioma del documento a español (México) con soporte para inglés
% Standard packages
\usepackage{float}                           % Imágenes flotantes en el documento
\usepackage{ifthen}                          % Soporte if-then en macros
\usepackage{xspace}                          % Soporte de autoespaciado en macros
\usepackage{xstring}                         % Operaciones con cadenas en macros
\usepackage{wrapfig}                         % Permite colocar texto al rededor de figuras y otros flotantes
\usepackage{booktabs}                        % Embellece tablas
\usepackage{csquotes}                        % Entrecomillado automático y manejo de citas textuales
\usepackage{fancyhdr}                        % Permite reconfigurar encabezado y pie de página
\usepackage{fancyvrb}                        % Define estilos para entornos Verbatim
\usepackage{geometry}                        % Permite reconfigurar la geometría del documento
\usepackage{graphicx}                        % Permite insertar imágenes en varios formatos
\usepackage{lastpage}                        % Referencia a la última página del documento
\usepackage{listings}                        % Define estilos para entornos de código de programación (sintaxis)
\usepackage{multicol}                        % Manejo de texto en varias columnas
\usepackage{tabularx}                        % Tablas con ancho de columna variable
\usepackage{algorithm}                       % Entorno para escribir algoritmos
\usepackage{algpseudocode}                   % Entorno para escribir algoritmos en pseudocódigo
\usepackage[justification=centering]{subcaption} % Permite imágenes en viñetas
\usepackage[all]{nowidow}                    % Control de viudas y huérfanas
\usepackage[inline]{enumitem}                % Añade opciones de configuración a listas
\usepackage[usenames,dvipsnames]{xcolor}     % Permite el uso de colores en el documento
% Referencing
\usepackage{varioref}                        % Gestión de referencias variables
\usepackage{hyperref}                        % Gestión de referencias e hipervínculos
\usepackage[noabbrev,nameinlink,spanish]{cleveref} % Gestión de referencias cruzadas inteligentes con hipervínculos
\usepackage[square, comma, numbers, sort&compress]{natbib} % Gestión de referencias bibliográficas

% %% %%%%%%%%%%%%%%%%%%%%%%%%%%%%%%%%%%%%%%%%%%%%%%%%%%%%%%%%%
%
% macros.tex
%
%  Author: Mauricio Matamoros
%  Date:   2020.02.28
%
%  Contiene los macros personalizados definidos para
%  el archivo-reporte de las prácticas de laboratorio
%
% %% %%%%%%%%%%%%%%%%%%%%%%%%%%%%%%%%%%%%%%%%%%%%%%%%%%%%%%%%%
% Archivo principal de LaTeX
%!TEX root = ../reporte.tex

%CHKTEX-FILE 9
%CHKTEX-FILE 21

\newcommand{\lpar}{(}
\newcommand{\rpar}{)}
\newcommand{\textcommand}[1]{\texttt{\textbackslash#1\{\}}}
\newcommand{\textenviron}[2][y]{\texttt{\textbackslash{}begin\{#2\} #1 \textbackslash{}end\{#2\}}}


\input{setup/colorboxes}
%CHKTEX-FILE 1
%CHKTEX-FILE 7
%CHKTEX-FILE 9
% Default fixed font does not support bold face
\DeclareFixedFont{\ttb}{T1}{txtt}{bx}{n}{8} % for bold
\DeclareFixedFont{\ttm}{T1}{txtt}{m}{n}{8}  % for normal

% Custom colors
\usepackage{color}
\definecolor{keywordsColor}{rgb}{0,0,0.5}
\definecolor{customColor}{rgb}{0.6,0,0}
\definecolor{stringColor}{rgb}{0,0.5,0}

% Code highlighting python
\renewcommand{\ttdefault}{pcr}
\lstset{
	language=Python,                              % the language of the code (can be overrided per snippet)
	backgroundcolor=\color{white},                % choose the background color
	basicstyle=\footnotesize\ttfamily,            % the size of the fonts that are used for the code
	breakatwhitespace=false,                      % sets if automatic breaks should only happen at whitespace
	breaklines=true,                              % sets automatic line breaking
	captionpos=t,                                 % sets the caption-position to bottom
	commentstyle=\color{gray},                    % comment style
	deletekeywords={},                            % if you want to delete keywords from the given language
%	escapeinside={\%*}{*)},                       % if you want to add LaTeX within your code
	extendedchars=true,                           % lets you use non-ASCII characters; for 8-bits encodings only, does not work with UTF-8
	frame=tb,                                     % adds a frame around the code
	keepspaces=true,                              % keeps spaces in text, useful for keeping indentation of code (possibly needs columns=flexible)
	keywordstyle=\color{keywordsColor}\bfseries,  % keyword style
	numbers=left,                                 % where to put the line-numbers; possible values are (none, left, right)
	numbersep=5pt,                                % how far the line-numbers are from the code
	numberstyle=\tiny\color{gray},                % the style that is used for the line-numbers
	rulecolor=\color{black},                      % if not set, the frame-color may be changed on line-breaks within not-black text (e.g. comments (green here))
	showspaces=false,                             % show spaces everywhere adding particular underscores; it overrides 'showstringspaces'
	showstringspaces=false,                       % underline spaces within strings only
	showtabs=false,                               % show tabs within strings adding particular underscores
	stepnumber=1,                                 % the step between two line-numbers. If it's 1, each line will be numbered
	stringstyle=\color{stringColor},              % string literal style
	tabsize=2,                                    % sets default tabsize to 2 spaces
	title=\lstname,                               % show the filename of files included with \lstinputlisting; also try caption instead of title
	columns=fixed,                                % Using fixed column width (for e.g. nice alignment)
	otherkeywords={self},                         % if you want to add more keywords to the set
	emphstyle=\color{customColor}\bfseries,       % Custom highlighting style
	emph={__init__,__main__,True,False,None},     % Custom highlighting keywords
	xleftmargin=1cm,                              % Left margin
	xrightmargin=1cm,                             % Right margin
	% Unicode compatibility
	inputencoding=utf8,
	literate={%
	            {Á}{{\'a}}1 {É}{{\'E}}1 {Í}{{\'I}}1 {Ó}{{\'O}}1 {Ú}{{\'U}}1%
	            {á}{{\'a}}1 {é}{{\'e}}1 {í}{{\'i}}1 {ó}{{\'o}}1 {ú}{{\'u}}1%
	            {À}{{\`A}}1 {È}{{\'E}}1 {Ì}{{\`I}}1 {Ò}{{\`O}}1 {Ù}{{\`U}}1%
	            {à}{{\`a}}1 {è}{{\`e}}1 {ì}{{\`i}}1 {ò}{{\`o}}1 {ù}{{\`u}}1%
	            {Ä}{{\"A}}1 {Ë}{{\"E}}1 {Ï}{{\"I}}1 {Ö}{{\"O}}1 {Ü}{{\"U}}1%
	            {ä}{{\"a}}1 {ë}{{\"e}}1 {ï}{{\"i}}1 {ö}{{\"o}}1 {ü}{{\"u}}1%
	            {Â}{{\^A}}1 {Ê}{{\^E}}1 {Î}{{\^I}}1 {Ô}{{\^O}}1 {Û}{{\^U}}1%
	            {â}{{\^a}}1 {ê}{{\^e}}1 {î}{{\^i}}1 {ô}{{\^o}}1 {û}{{\^u}}1% CHKTEX 19
	            {Ã}{{\~a}}1 {Ẽ}{{\~E}}1 {Ĩ}{{\~I}}1 {Õ}{{\~O}}1 {Ũ}{{\~U}}1 {Ñ}{{\~N}}1%
	            {ã}{{\~a}}1 {ẽ}{{\~e}}1 {ĩ}{{\~i}}1 {õ}{{\~o}}1 {ũ}{{\~u}}1 {ñ}{{\~n}}1%
	            {œ}{{\oe}}1 {Œ}{{\OE}}1 {æ}{{\ae}}1 {Æ}{{\AE}}1 {ß}{{\ss}}1%
	            {ç}{{\c c}}1 {Ç}{{\c C}}1 {ø}{{\o}}1 {å}{{\r a}}1 {Å}{{\r A}}1%
	            {€}{{\EUR}}1 {£}{{\pounds}}1 {×}{{\(\times\)}}1% CHKTEX 21
	            {°}{{\textsuperscript{o}}}1%
	            {¹}{{\textsuperscript{1}}}1%
	            {²}{{\textsuperscript{2}}}1%
	            {³}{{\textsuperscript{3}}}1%
	            {⁴}{{\textsuperscript{4}}}1% CHKTEX 19
	            {⁵}{{\textsuperscript{5}}}1% CHKTEX 19
	            {⁶}{{\textsuperscript{6}}}1% CHKTEX 19
	            {⁷}{{\textsuperscript{7}}}1% CHKTEX 19
	            {⁸}{{\textsuperscript{8}}}1% CHKTEX 19
	            {⁹}{{\textsuperscript{9}}}1% CHKTEX 19
	            {⁰}{{\textsuperscript{0}}}1% CHKTEX 19
%	            {A}{{\textAlpha}}1
	            {α}{{\textalpha}}1%
%	            {B}{{\textBeta}}1
	            {β}{{\textbeta}}1%
	            {Γ}{{\textGamma}}1
	            {γ}{{\textgamma}}1%
	            {Δ}{{\textDelta}}1
	            {δ}{{\textdelta}}1% CHKTEX 19
%	            {E}{{\textEpsilon}}1
	            {ϵ}{{\textepsilon}}1%
%	            {Z}{{\textZeta}}1
	            {ζ}{{\textzeta}}1%
%	            {H}{{\textEta}}1
	            {η}{{\texteta}}1%
	            {Θ}{{\textTheta}}1
	            {θ}{{\texttheta}}1%
%	            {I}{{\textIota}}1
	            {ι}{{\textiota}}1%
%	            {K}{{\textKappa}}1
	            {κ}{{\textkappa}}1%
	            {Λ}{{\textLambda}}1
	            {λ}{{\textlambda}}1%
%	            {M}{{\textMu}}1
	            {μ}{{\textmu}}1%
%	            {N}{{\textNu}}1
	            {ν}{{\textnu}}1%
	            {Ξ}{{\textXi}}1
	            {ξ}{{\textxi}}1%
%	            {O}{{\textOmikron}}1
%	            {o}{{\textomikron}}1%
	            {Π}{{\textPi}}1
	            {π}{{\textpi}}1%
%	            {P}{{\textRho}}1
	            {ρ}{{\textrho}}1%
	            {Σ}{{\textSigma}}1
	            {σ}{{\textsigma}}1%
%	            {T}{{\textTau}}1
	            {τ}{{\texttau}}1%
	            {ϒ}{{\textUpsilon}}1
	            {υ}{{\textupsilon}}1%
	            {Φ}{{\textPhi}}1
	            {ϕ}{{\textphi}}1%
%	            {X}{{\textChi}}1
	            {χ}{{\textchi}}1%
	            {Ψ}{{\textPsi}}1
	            {ψ}{{\textpsi}}1%
	            {Ω}{{\textOmega}}1
	            {ω}{{\textomega}}1%
	            {ζ}{{\varsigma}}1%
%	            {}{{\straightphi}}1%
%	            {}{{\scripttheta}}1%
%	            {}{{\straighttheta}}1%
%	            {}{{\straightepsilon}}1%
	         },
}

\lstdefinestyle{c_with_comments}%
{
	language     = c,
	morecomment  = [l]{//},
	morecomment  = [s]{/*}{*/},
	breaklines,
}

\lstdefinestyle{c_without_comments}%
{
	style        = c_with_comments,
	% numbers      = none,
	% keepspaces   = false,
	morecomment  = [l][\nullfont]{//},
	morecomment  = [is]{//}{\^^M},
	morecomment  = [is]{/*}{*/},
}

\lstdefinelanguage{conf}
{
	basicstyle=\ttfamily\small,
	columns=fullflexible,
	morecomment=[s][\color{Orchid}\bfseries]{[}{]},
	morecomment=[l]{\#},
	morecomment=[l]{;},
	commentstyle=\color{gray}\ttfamily,
	% morekeywords={},
	% otherkeywords={=,:},
	% keywordstyle={\color{Green}\bfseries}
}

% \captionsetup[lstlisting]{font={small,tt}}
\captionsetup[lstlisting]{%
	font={small},
}



\DefineVerbatimEnvironment{Verbatim}{Verbatim}{%
	fontsize=\footnotesize,%
	frame=leftline,%
	framesep=2em,    % separation between frame and text
}

\RecustomVerbatimCommand{\VerbatimInput}{VerbatimInput}{%
	fontsize=\footnotesize,
%	frame=lines,            % top and bottom rule only
	frame=leftline,         % left rule only
	numbers=left,           % Line numbers on the left
	numbersep=0.25em,       % Gap between numbers and verbatim lines
	xleftmargin=4em,        % Indentation to add at the start of each line
	xrightmargin=4em,       % Right margin to add after each line
	framesep=0.5em,         % separation between frame and text
	rulecolor=\color{Gray}, % Color of the lines
	labelposition=topline,  %
	samepage=false,         % When true, prevents verbatim environment from
	                        % being broken between pages
%	commandchars=\|\(\),    % escape character and argument delimiters for
	                        % commands within the verbatim
%	commentchar=*           % comment character
}


\hypersetup{
	hidelinks,
	colorlinks=true,
	linkcolor=Blue,
	filecolor=OliveGreen,
	urlcolor=RoyalPurple,
	pdfauthor={Mauricio Matamoros},
%	pdftitle={Práctica 0X – Fundamentos de Sistemas Embebidos},
% 	pdfsubject={The Subject},
% 	pdfkeywords={Some Keywords},
% 	pdfproducer={Latex with hyperref, or other system},
% 	pdfcreator={pdflatex, or other tool}
}

\captionsetup{%
	font=small
}

\geometry{%
	margin=2cm,
	% top=3cm,
	bottom=3cm,
	% left=2cm,
	% right=2cm,
	% inner=2cm,
	% outer=2cm,
	% headheight=,
	% footsep=,
	% footskip=,
}

\pagestyle{fancy}
\renewcommand{\headrulewidth}{0.0pt}
\lhead{}
\chead{}
\rhead{}
\lfoot{}
\cfoot{}
\rfoot{Página~\thepage~de~\pageref{LastPage}}

\crefname{table}{tabla}{tablas}
\Crefname{table}{Tabla}{Tablas}
\crefname{section}{sección}{secciones}
\Crefname{section}{Sección}{Secciones}
\crefname{subsection}{subsección}{subsecciones}
\Crefname{subsection}{Subsección}{Subsecciones}
\crefname{listing}{código de ejemplo}{códigos de ejemplo}
\Crefname{listing}{Código de Ejemplo}{Códigos de Ejemplo}
\renewcommand*{\lstlistingname}{Código ejemplo}


\author{\footnotesize Autor: José Mauricio Matamoros de Maria y Campos}
\title{Práctica 3:\\Uso del puerto GPIO de la Raspberry Pi\\
{\large Fundamentos de Sistemas Embebidos}}
\date{}

% Document body
\begin{document}
\maketitle

\section{Objetivo}%
\label{sec:objective}
El alumno aprenderá a utilizar el puerto GPIO de la Raspberry Pi, configurándo varios pines como salidas digitales para el control de leds y circuitos de lógica TTL.%

% \section{Introducción}%
% \label{sec:introduction}
% Raspbian es el sistema operativo más popular para Raspberry Pi, además de ser el único con soporte oficial.
% Raspbian es una distribución de Linux basada en Debian, optimizado para la Raspberry Pi y que permite a esta operar como una PC.~La distro incorpora terminal y navegador web entre otros programas.

\section{Material}%
\label{sec:material}
Se asume que el alumno cuenta con un una Raspberry Pi con sistema operativo Raspbian e interprete de Python instalado. Se aconseja encarecidamente el uso de \textit{git} como programa de control de versiones.

\begin{multicols}{2}[%
Además, el alumno necesitará:
]
\begin{itemize}[noitemsep]
	\item 7 Diodos emisores de luz LEDS
	\item 8 resistencias de 330$\Omega$
	\item 1 Condensador de 0.1$\mu$F
	\item 1 Array Darlington ULN2003 (o 7 transistores de potencia)
	\item 1 Decodificador de 7--segmentos ánodo común
	\item 1 Display de 7 segmentos ánodo común
	\item 1 Conector DIL con cable plano tipo listón para el GPIO de la Raspberry Pi (similar al de un Disco Duro PATA, véase  \Cref{fig:cable-dil})
	\item 1 protoboard o circuito impreso equivalente
	\item 1 fuente de alimentación regulada a 5V y al menos 2 amperios de salida
	\item Cables y conectores varios
\end{itemize}
\columnbreak
\begin{figure}[H]
	\centering%
	\includegraphics[width=0.9\columnwidth]{img/p03-dil.jpg} %CHKTEX 8
	\caption{Cable plano con conector DIL}
	\label{fig:cable-dil} %CHKTEX 24
\end{figure}
% ~\\
\end{multicols}


% Se controlará el encendido y apagado de LEDS usando la Raspberry Pi

\section{Instrucciones}%
\label{sec:instructions}
\begin{enumerate}[noitemsep]
	\item Alambre el circuito tal y como se detalla en la \cref{sec:step1}
	\item Antes de conectar la Raspberry Pi, pruebe el circuito como se explica en la \cref{sec:step2}
	\item Realice los programas de las \cref{sec:step3,sec:step4,sec:step5}
	\item Analice los programas de las \cref{sec:step3,sec:step4,sec:step5}, realice los experimentos propuestos en la \cref{sec:experiments} y con los resultados obtenidos responda el cuestionario de la \cref{sec:questionnaire}.
\end{enumerate}

% %% %%%%%%%%%%%%%%%%%%%%%%%%%%%%%%%%%%%%%%%%%%%%%%%%%%%%%%%%%%%%%%%%%%
%
% Step 1
%
% %% %%%%%%%%%%%%%%%%%%%%%%%%%%%%%%%%%%%%%%%%%%%%%%%%%%%%%%%%%%%%%%%%%%
\subsection{Paso 1: Alambrado}%
\label{sec:step1}

El proceso de alambrado de esta práctica considera dos circuitos.

El primer circuito permitirá controlar con las salidas digitales del GPIO de la Raspberry Pi el encendido y apagado de siete leds mediante el uso de un encapsulado de varios controladores de potencia tipo Darlington (\textit{Darlington Array}).
De manera simimlar, el segundo circuito se auxiliará de un integrado TTL para desplegar números del 0 al 9 en un display de 7 segmentos (véase \Cref{fig:wiring-diagram}).

\begin{figure}[H]
	\centering%
	\includegraphics[width=0.9\columnwidth,height=8cm,keepaspectratio]{img/p03-diagram.pdf} %CHKTEX 8
	\caption{Diagrama de conexiones del circuito a alambrar}
	\label{fig:wiring-diagram} %CHKTEX 24
\end{figure}

\subsubsection{Subcircuito 1: leds en línea}.
Forme los siete leds en línea, cuidando de que todos tengan la misma orientación. A continuación, conecte el cátodo de cada LED a una resistencia de 300$\Omega$, y ésta a su vez a una salida libre del controlador ULN2003, de tal manera que el primer led de la fila esté conectado a la salida 1, el segundo a la salida 2, y así sucesivamente.

De manera similar, conecte las entradas 1 a 7 del ULN2003 a las salidas GPIO 18, 23, 24, 25, 8, 7, y 12 de la Raspberry Pi mediante el cable tipo listón (pines 12, 16, 18, 22, 24, 26 y 32).
La conexión de uno de los leds se conecte al GPIO12/PWM es importante, pues éste pin se utilizará para variar la intensidad del led más adelante.

Complete el alambrado del ULN2003 conectándolo a tierra, conectando el común de éste a VCC mediante el capacitor de 0.1$\mu$F. Finalmente, conecte el ánodo de todos los leds a VCC.%

\subsubsection{Subcircuito 2: Display de 7 segmentos}.
Comience conectando las salidas \emph{a} a \emph{g} del display del circuito controlador 74LS47 con los pines homónimos del display de siete segmentos.
A continuación, conecte el integrado a tierra y las terminales LT y RBI (pines 3 y 5) a VCC, dejando BI (pin 4) sin conectar tal como se muestra en la \Cref{fig:wiring-diagram}.
Como siguiente paso, conecte el ánodo común del display a VCC mediante una resistencia de 330$\Omega$.

A continuación conecte las entradas A, B, C, D a las salidas GPIO 16, 20, 21 y 26 respectivamente (pines 36, 38, 40 y 37 de la Raspberry Pi) mediante el cable tipo listón.

Finalmente, para terminar el alambrado, conecte el 74LS47 a VCC y tierra.

% %% %%%%%%%%%%%%%%%%%%%%%%%%%%%%%%%%%%%%%%%%%%%%%%%%%%%%%%%%%%%%%%%%%%
%
% Step 2
%
% %% %%%%%%%%%%%%%%%%%%%%%%%%%%%%%%%%%%%%%%%%%%%%%%%%%%%%%%%%%%%%%%%%%%
\subsection{Paso 2: Configuración y prueba del circuito}%
\label{sec:step2}
Antes de proceder, verifique conexiones con un multímetro en busca de corto circuitos. En particular verifique que exista una impedancia muy alta entre los pines 4 y 6 (VCC y tierra) del cable listón que conectará al GPIO de la Raspberry Pi.

Tras lo anterior, conecte VCC y tierra del circuito alambrado a un eliminador de corriente de 5V y cierre el circuito entre los pines 4 y 32 del cable listón. Si todo está alambrado correctamente, el último del de la fila deberá encender.

Ahora cierre el circuito entre los pines 4 y 37 del cable listón. Si todo está alambrado correctamente, el display de siete segmentos deberá mostrar un 1.

\paragraph{Importante:} Ninguno de los circuitos o resistencias debe calentarse.
Si alguno de los componentes emitiera calor, verifique las conexiones en busca de corto circuitos o reemplace los componentes dañados.

Verificadas las conexiones, instale los complementos de desarrollo del puerto de propósito general de la Raspberry Pi (deberían venir instalados por defecto).

\begin{Verbatim}[fontsize=\footnotesize]
$ sudo apt-get install python-rpi.gpio python3-rpi.gpio
\end{Verbatim}

% %% %%%%%%%%%%%%%%%%%%%%%%%%%%%%%%%%%%%%%%%%%%%%%%%%%%%%%%%%%%%%%%%%%%
%
% Step 3
%
% %% %%%%%%%%%%%%%%%%%%%%%%%%%%%%%%%%%%%%%%%%%%%%%%%%%%%%%%%%%%%%%%%%%%
\subsection{Paso 3: Led parpadeante}%
\label{sec:step3}
Con todas las conexiones probadas y verificadas, es seguro proceder al control de señales digitales utilizando el GPIO de la Raspberry Pi.

Inicie su Raspberry Pi y, ya sea mediante una terminal remota o directamente en ella, ejecute el siguiente código Python para hacer parpadear uno de los leds del circuito alambrado.

\medskip
\begin{lstlisting}
# Importa la librería de control del GPIO de la Raspberry Pi
import RPi.GPIO as GPIO
# Importa la función sleep del módulo time
from time import sleep

# Desactivar advertencias (warnings)
GPIO.setwarnings(False)
# Configurar la librería para usar el número de pin.
# Llame GPIO.setmode(GPIO.BCM) para usar el canal SOC definido por Broadcom
GPIO.setmode(GPIO.BOARD)
# Configurar el pin 32 como salida y habilitar en bajo
GPIO.setup(32, GPIO.OUT, initial=GPIO.LOW)

# El siguiente código hace parpadear el led
while True: # Bucle infinito
	sleep(0.5)                 # Espera 500ms
	GPIO.output(32, GPIO.HIGH) # Enciende el led
	sleep(0.5)                 # Espera 500ms
	GPIO.output(32, GPIO.LOW)  # Apaga el led
\end{lstlisting}
\medskip


% %% %%%%%%%%%%%%%%%%%%%%%%%%%%%%%%%%%%%%%%%%%%%%%%%%%%%%%%%%%%%%%%%%%%
%
% Step 4
%
% %% %%%%%%%%%%%%%%%%%%%%%%%%%%%%%%%%%%%%%%%%%%%%%%%%%%%%%%%%%%%%%%%%%%
\subsection{Paso 4: Led parpadeante con PWM}%
\label{sec:step4}
En lugar de utilizar tiempos de espera (mismos que consumen tiempo de procesamiento y energía), es posible hacer parpadear el led de manera mucho más precisa y rápida utilizando uno de los moduladores de ancho de pulso (en inglés \emph{Pulse Width Modulation} o \emph{PWM}) por hardware que incorpora la Raspberry Pi.

Ejecute el siguiente código Python para hacer parpadear uno de los leds del circuito alambrado.

\medskip
\begin{lstlisting}
# Importa la librería de control del GPIO de la Raspberry Pi
import RPi.GPIO as GPIO
# Importa la función sleep del módulo time
from time import sleep

# Desactivar advertencias (warnings)
GPIO.setwarnings(False)
# Configurar la librería para usar el número de pin.
GPIO.setmode(GPIO.BOARD)
# Configurar el pin 32 como salida y habilitar en bajo
GPIO.setup(32, GPIO.OUT, initial=GPIO.LOW)
# Inicializar el pin 32 como PWM a una frecuencia de 2Hz
pwm = GPIO.PWM(32, 1)

# El siguiente código hace parpadear el led
pwm.start(50)

flag = True
while flag:
	try:
		dutyCycle = int(input("Ingrese ciclo de trabajo: "))
		pwm.ChangeDutyCycle(dutyCycle)
	except:
		flag = False
		pwm.ChangeDutyCycle(0)
# Detiene el PWM
pwm.stop()
# Reinicia los puertos GPIO (cambian de salida a entrada)
GPIO.cleanup()
\end{lstlisting}


% %% %%%%%%%%%%%%%%%%%%%%%%%%%%%%%%%%%%%%%%%%%%%%%%%%%%%%%%%%%%%%%%%%%%
%
% Step 5
%
% %% %%%%%%%%%%%%%%%%%%%%%%%%%%%%%%%%%%%%%%%%%%%%%%%%%%%%%%%%%%%%%%%%%%
\subsection{Paso 5: Display de siete segmentos}%
\label{sec:step5}
El último ejemplo consiste en mostrar números en el display de siete segmentos.

Analice y ejecute el código mostrado a continuación.
\medskip
\begin{lstlisting}
# Importa la librería de control del GPIO de la Raspberry Pi
import RPi.GPIO as GPIO
# Importa la función sleep del módulo time
from time import sleep

# Desactivar advertencias (warnings)
GPIO.setwarnings(False)
# Configurar la librería para usar el número de pin.
GPIO.setmode(GPIO.BOARD)
# Configurar pines 36, 38, 40 y 37 como salida y habilitar en bajo
GPIO.setup(32, GPIO.OUT, initial=GPIO.LOW)
GPIO.setup(38, GPIO.OUT, initial=GPIO.LOW)
GPIO.setup(40, GPIO.OUT, initial=GPIO.LOW)
GPIO.setup(37, GPIO.OUT, initial=GPIO.LOW)

# Mapea bits a los pines de la GPIO
def bcd7(num):
	GPIO.output(32, GPIO.HIGH if num & 0x00000008 else GPIO.LOW )
	GPIO.output(38, GPIO.HIGH if num & 0x00000004 else GPIO.LOW )
	GPIO.output(40, GPIO.HIGH if num & 0x00000002 else GPIO.LOW )
	GPIO.output(37, GPIO.HIGH if num & 0x00000001 else GPIO.LOW )

flag = True
while flag:
	try:
		num = int(input("Ingrese número entero: "))
		bcd(num)
	except:
		flag = False
# Reinicia los puertos GPIO (cambian de salida a entrada)
GPIO.cleanup()
\end{lstlisting}

% %% %%%%%%%%%%%%%%%%%%%%%%%%%%%%%%%%%%%%%%%%%%%%%%%%%%%%%%%%%%%%%%%%%%
%
% Experiments
%
% %% %%%%%%%%%%%%%%%%%%%%%%%%%%%%%%%%%%%%%%%%%%%%%%%%%%%%%%%%%%%%%%%%%%
\section{Experimentos}%
\label{sec:experiments}

\begin{enumerate}
	\item{} [1pt] Modifique el código de la \cref{sec:step3} para todos los leds de la fila parpadeen.
	\item{} [1pt] Modifique el código de las \cref{sec:step3,sec:step5} para que los leds de la fila enciendan de manera continua en una marquesina de izquierda a derecha.
	\item{} [2pt] Modifique el código de las \cref{sec:step3,sec:step5} para que los leds de la fila enciendan de manera continua en una marquesina de derecha a izquierda con velocidad variable definida por el usuario.
	\item{} [1pt] Con base en las modificaciones anteriores, genere una marquesina que haga parecer que \enquote{la luz rebota} al llegar a las orillas (efecto ping-pong) con velocidad variable definida por el usuario.
	\item{} [3pt] Tomando como base el código de la \cref{sec:step4}, haga que uno de los leds encienda gradualmente a lo largo de un segundo hasta adquirir máxima potencia, permanezca encendido medio segundo, y después se apague gradualmente a lo largo de otro segundo.
\end{enumerate}

% %% %%%%%%%%%%%%%%%%%%%%%%%%%%%%%%%%%%%%%%%%%%%%%%%%%%%%%%%%%%%%%%%%%%
%
% Questionnaire
%
% %% %%%%%%%%%%%%%%%%%%%%%%%%%%%%%%%%%%%%%%%%%%%%%%%%%%%%%%%%%%%%%%%%%%
\section{Cuestionario}%
\label{sec:questionnaire}
\begin{enumerate}
	\item{} [0.5pt] Explique por qué usar corrimientos es la manera más eficiente de generar una marquesina.
	\item{} [0.5pt] Explique las ventajas o desventajas que tiene utilizar un modulador de ancho de pulso sobre tiempos de espera programados (\emph{delays}).
	\item{} [0.5pt] ¿Sería posible generar una marquesina circular utilizando el 74LS47 y el display de siete segmentos? Justifique su respuesta.
	\item{} [0.5pt] ¿Es posible configurar cualquier pin de la GPIO como PWM? Justifique su respuesta.
\end{enumerate}

\appendix

\cleardoublepage
\section{Programa Ejemplo: \texttt{blink.py}}%
\label{sec:appendix1}
\lstinputlisting[language=C,firstline=13]{src/blink.py}

\cleardoublepage
\section{Programa Ejemplo: \texttt{bcd.py}}%
\label{sec:appendix2}
\lstinputlisting[language=Python,firstline=14]{src/bcd.py}

\cleardoublepage
\section{Programa Ejemplo: \texttt{pwm.py}}%
\label{sec:appendix3}
\lstinputlisting[language=C,firstline=11]{src/pwm.py}

\cleardoublepage
\section{Programa Ejemplo: \texttt{pwm\_fast.py}}%
\label{sec:appendix3}
\lstinputlisting[language=C,firstline=11]{src/pwm_fast.py}%CHKTEX 25

\end{document}
