% %% %%%%%%%%%%%%%%%%%%%%%%%%%%%%%%%%%%%%%%%%%%%%%%%%%%%%%%%%%
%
% packages.tex
%
%  Author: Mauricio Matamoros
%  Date:   2020.02.28
%
%  Contiene la lista de paquetes requeridos para generar
%  el archivo-reporte de las prácticas de laboratorio
%
% %% %%%%%%%%%%%%%%%%%%%%%%%%%%%%%%%%%%%%%%%%%%%%%%%%%%%%%%%%%
% Archivo principal de LaTeX
%!TEX root = ../reporte.tex

\usepackage[utf8]{inputenc}                  % Soporte para utf8
\usepackage[T1]{fontenc}                     % Soporte extendido de caracteres unicode
\usepackage[english,spanish,mexico]{babel}   % Define el idioma del documento a español (México) con soporte para inglés
% Standard packages
\usepackage{float}                           % Imágenes flotantes en el documento
\usepackage{ifthen}                          % Soporte if-then en macros
\usepackage{xspace}                          % Soporte de autoespaciado en macros
\usepackage{xstring}                         % Operaciones con cadenas en macros
\usepackage{wrapfig}                         % Permite colocar texto al rededor de figuras y otros flotantes
\usepackage{booktabs}                        % Embellece tablas
\usepackage{csquotes}                        % Entrecomillado automático y manejo de citas textuales
\usepackage{fancyhdr}                        % Permite reconfigurar encabezado y pie de página
\usepackage{fancyvrb}                        % Define estilos para entornos Verbatim
\usepackage{geometry}                        % Permite reconfigurar la geometría del documento
\usepackage{graphicx}                        % Permite insertar imágenes en varios formatos
\usepackage{lastpage}                        % Referencia a la última página del documento
\usepackage{listings}                        % Define estilos para entornos de código de programación (sintaxis)
\usepackage{multicol}                        % Manejo de texto en varias columnas
\usepackage{tabularx}                        % Tablas con ancho de columna variable
\usepackage{algorithm}                       % Entorno para escribir algoritmos
\usepackage{algpseudocode}                   % Entorno para escribir algoritmos en pseudocódigo
\usepackage[justification=centering]{subcaption} % Permite imágenes en viñetas
\usepackage[all]{nowidow}                    % Control de viudas y huérfanas
\usepackage[inline]{enumitem}                % Añade opciones de configuración a listas
\usepackage[usenames,dvipsnames]{xcolor}     % Permite el uso de colores en el documento
% Referencing
\usepackage{varioref}                        % Gestión de referencias variables
\usepackage{hyperref}                        % Gestión de referencias e hipervínculos
\usepackage[noabbrev,nameinlink,spanish]{cleveref} % Gestión de referencias cruzadas inteligentes con hipervínculos
\usepackage[square, comma, numbers, sort&compress]{natbib} % Gestión de referencias bibliográficas
