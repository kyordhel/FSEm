% CHKTEX-FILE 1
% CHKTEX-FILE 46
% \documentclass[letterpaper,10.5pt,twocolumn]{article}
\documentclass[letterpaper,10.5pt]{article}

% %% %%%%%%%%%%%%%%%%%%%%%%%%%%%%%%%%%%%%%%%%%%%%%%%%%%%%%%%%%
%
% packages.tex
%
%  Author: Mauricio Matamoros
%  Date:   2020.02.28
%
%  Contiene la lista de paquetes requeridos para generar
%  el archivo-reporte de las prácticas de laboratorio
%
% %% %%%%%%%%%%%%%%%%%%%%%%%%%%%%%%%%%%%%%%%%%%%%%%%%%%%%%%%%%
% Archivo principal de LaTeX
%!TEX root = ../reporte.tex

\usepackage[utf8]{inputenc}                  % Soporte para utf8
\usepackage[T1]{fontenc}                     % Soporte extendido de caracteres unicode
\usepackage[english,spanish,mexico]{babel}   % Define el idioma del documento a español (México) con soporte para inglés
% Standard packages
\usepackage{float}                           % Imágenes flotantes en el documento
\usepackage{ifthen}                          % Soporte if-then en macros
\usepackage{xspace}                          % Soporte de autoespaciado en macros
\usepackage{xstring}                         % Operaciones con cadenas en macros
\usepackage{wrapfig}                         % Permite colocar texto al rededor de figuras y otros flotantes
\usepackage{booktabs}                        % Embellece tablas
\usepackage{csquotes}                        % Entrecomillado automático y manejo de citas textuales
\usepackage{fancyhdr}                        % Permite reconfigurar encabezado y pie de página
\usepackage{fancyvrb}                        % Define estilos para entornos Verbatim
\usepackage{geometry}                        % Permite reconfigurar la geometría del documento
\usepackage{graphicx}                        % Permite insertar imágenes en varios formatos
\usepackage{lastpage}                        % Referencia a la última página del documento
\usepackage{listings}                        % Define estilos para entornos de código de programación (sintaxis)
\usepackage{multicol}                        % Manejo de texto en varias columnas
\usepackage{tabularx}                        % Tablas con ancho de columna variable
\usepackage{algorithm}                       % Entorno para escribir algoritmos
\usepackage{algpseudocode}                   % Entorno para escribir algoritmos en pseudocódigo
\usepackage[justification=centering]{subcaption} % Permite imágenes en viñetas
\usepackage[all]{nowidow}                    % Control de viudas y huérfanas
\usepackage[inline]{enumitem}                % Añade opciones de configuración a listas
\usepackage[usenames,dvipsnames]{xcolor}     % Permite el uso de colores en el documento
% Referencing
\usepackage{varioref}                        % Gestión de referencias variables
\usepackage{hyperref}                        % Gestión de referencias e hipervínculos
\usepackage[noabbrev,nameinlink,spanish]{cleveref} % Gestión de referencias cruzadas inteligentes con hipervínculos
\usepackage[square, comma, numbers, sort&compress]{natbib} % Gestión de referencias bibliográficas


\newcommand{\lpar}{(}\newcommand{\rpar}{)} %CHKTEX 9
\newcommand{\IIC}{I\textsuperscript{2}C\xspace}
\newcommand{\GND}{\textsc{Gnd}\xspace}
\newcommand{\VCC}{\textsc{Vcc}\xspace}
\newcommand{\VDD}{\textsc{Vdd}\xspace}
\newcommand{\textbi}[1]{\textbf{\textit{#1}}}
\newcommand{\degreesC}[1]{%
	#1\textsuperscript{o}C\xspace{}%
}
\newcommand{\degreesF}[1]{%
	#1\textsuperscript{o}F\xspace{}%
}

% \newcommand{\VCC}{V\textsubscript{CC}\xspace{}}
% \newcommand{\GND}{\textsc{Gnd}\xspace{}}
% CHKTEX-FILE 26
% CHKTEX-FILE 36

\tcbuselibrary{most}
% \tcbuselibrary{listings,breakable}
% \usetikzlibrary{shadings,shadows}
% \usetikzlibrary{decorations.pathmorphing}
% \usetikzlibrary{patterns}
% \usetikzlibrary{spy}
% \usetikzlibrary{arrows.meta}

\newtcolorbox{importantbox}[1]{%
	enhanced,
	colback=red!5!white,%
	colframe=red!75!black,%
	fonttitle=\bfseries,%
	center title,
	title={#1},%
	drop fuzzy shadow
}


\newtcolorbox{greenbox}[1]{%
	enhanced,
	colback=Green!5!white,%
	colframe=Green!75!black,%
	fonttitle=\bfseries,%
	center title,
	title={#1},%
	drop fuzzy shadow
}

\newtcolorbox{marker}[1][]{%
	enhanced,
	before skip=2mm,after skip=3mm,
	boxrule=0.4pt,left=5mm,right=2mm,top=1mm,bottom=1mm,
	colback=yellow!50,
	colframe=yellow!20!black,
	sharp corners,rounded corners=southeast,arc is angular,arc=3mm,
%	underlay={%
%		\path[fill=tcbcolback!80!black] ([yshift=3mm]interior.south east)--++(-0.4,-0.1)--++(0.1,-0.2);
%		\path[draw=tcbcolframe,shorten <=-0.05mm,shorten >=-0.05mm] ([yshift=3mm]interior.south east)--++(-0.4,-0.1)--++(0.1,-0.2);
%		\path[fill=yellow!50!black,draw=none] (interior.south west) rectangle node[white]{\Huge\bfseries !} ([xshift=4mm]interior.north west);
%	},
	drop fuzzy shadow,#1
}

%CHKTEX-FILE 1
%CHKTEX-FILE 7
%CHKTEX-FILE 9
% Default fixed font does not support bold face
\DeclareFixedFont{\ttb}{T1}{txtt}{bx}{n}{8} % for bold
\DeclareFixedFont{\ttm}{T1}{txtt}{m}{n}{8}  % for normal

% Custom colors
\usepackage{color}
\definecolor{keywordsColor}{rgb}{0,0,0.5}
\definecolor{customColor}{rgb}{0.6,0,0}
\definecolor{stringColor}{rgb}{0,0.5,0}

% Code highlighting python
\renewcommand{\ttdefault}{pcr}
\lstset{
	language=Python,                              % the language of the code (can be overrided per snippet)
	backgroundcolor=\color{white},                % choose the background color
	basicstyle=\footnotesize\ttfamily,            % the size of the fonts that are used for the code
	breakatwhitespace=false,                      % sets if automatic breaks should only happen at whitespace
	breaklines=true,                              % sets automatic line breaking
	captionpos=t,                                 % sets the caption-position to bottom
	commentstyle=\color{gray},                    % comment style
	deletekeywords={},                            % if you want to delete keywords from the given language
%	escapeinside={\%*}{*)},                       % if you want to add LaTeX within your code
	extendedchars=true,                           % lets you use non-ASCII characters; for 8-bits encodings only, does not work with UTF-8
	frame=tb,                                     % adds a frame around the code
	keepspaces=true,                              % keeps spaces in text, useful for keeping indentation of code (possibly needs columns=flexible)
	keywordstyle=\color{keywordsColor}\bfseries,  % keyword style
	numbers=left,                                 % where to put the line-numbers; possible values are (none, left, right)
	numbersep=5pt,                                % how far the line-numbers are from the code
	numberstyle=\tiny\color{gray},                % the style that is used for the line-numbers
	rulecolor=\color{black},                      % if not set, the frame-color may be changed on line-breaks within not-black text (e.g. comments (green here))
	showspaces=false,                             % show spaces everywhere adding particular underscores; it overrides 'showstringspaces'
	showstringspaces=false,                       % underline spaces within strings only
	showtabs=false,                               % show tabs within strings adding particular underscores
	stepnumber=1,                                 % the step between two line-numbers. If it's 1, each line will be numbered
	stringstyle=\color{stringColor},              % string literal style
	tabsize=2,                                    % sets default tabsize to 2 spaces
	title=\lstname,                               % show the filename of files included with \lstinputlisting; also try caption instead of title
	columns=fixed,                                % Using fixed column width (for e.g. nice alignment)
	otherkeywords={self},                         % if you want to add more keywords to the set
	emphstyle=\color{customColor}\bfseries,       % Custom highlighting style
	emph={__init__,__main__,True,False,None},     % Custom highlighting keywords
	xleftmargin=1cm,                              % Left margin
	xrightmargin=1cm,                             % Right margin
	% Unicode compatibility
	inputencoding=utf8,
	literate={%
	            {Á}{{\'a}}1 {É}{{\'E}}1 {Í}{{\'I}}1 {Ó}{{\'O}}1 {Ú}{{\'U}}1%
	            {á}{{\'a}}1 {é}{{\'e}}1 {í}{{\'i}}1 {ó}{{\'o}}1 {ú}{{\'u}}1%
	            {À}{{\`A}}1 {È}{{\'E}}1 {Ì}{{\`I}}1 {Ò}{{\`O}}1 {Ù}{{\`U}}1%
	            {à}{{\`a}}1 {è}{{\`e}}1 {ì}{{\`i}}1 {ò}{{\`o}}1 {ù}{{\`u}}1%
	            {Ä}{{\"A}}1 {Ë}{{\"E}}1 {Ï}{{\"I}}1 {Ö}{{\"O}}1 {Ü}{{\"U}}1%
	            {ä}{{\"a}}1 {ë}{{\"e}}1 {ï}{{\"i}}1 {ö}{{\"o}}1 {ü}{{\"u}}1%
	            {Â}{{\^A}}1 {Ê}{{\^E}}1 {Î}{{\^I}}1 {Ô}{{\^O}}1 {Û}{{\^U}}1%
	            {â}{{\^a}}1 {ê}{{\^e}}1 {î}{{\^i}}1 {ô}{{\^o}}1 {û}{{\^u}}1% CHKTEX 19
	            {Ã}{{\~a}}1 {Ẽ}{{\~E}}1 {Ĩ}{{\~I}}1 {Õ}{{\~O}}1 {Ũ}{{\~U}}1 {Ñ}{{\~N}}1%
	            {ã}{{\~a}}1 {ẽ}{{\~e}}1 {ĩ}{{\~i}}1 {õ}{{\~o}}1 {ũ}{{\~u}}1 {ñ}{{\~n}}1%
	            {œ}{{\oe}}1 {Œ}{{\OE}}1 {æ}{{\ae}}1 {Æ}{{\AE}}1 {ß}{{\ss}}1%
	            {ç}{{\c c}}1 {Ç}{{\c C}}1 {ø}{{\o}}1 {å}{{\r a}}1 {Å}{{\r A}}1%
	            {€}{{\EUR}}1 {£}{{\pounds}}1 {×}{{\(\times\)}}1% CHKTEX 21
	            {°}{{\textsuperscript{o}}}1%
	            {¹}{{\textsuperscript{1}}}1%
	            {²}{{\textsuperscript{2}}}1%
	            {³}{{\textsuperscript{3}}}1%
	            {⁴}{{\textsuperscript{4}}}1% CHKTEX 19
	            {⁵}{{\textsuperscript{5}}}1% CHKTEX 19
	            {⁶}{{\textsuperscript{6}}}1% CHKTEX 19
	            {⁷}{{\textsuperscript{7}}}1% CHKTEX 19
	            {⁸}{{\textsuperscript{8}}}1% CHKTEX 19
	            {⁹}{{\textsuperscript{9}}}1% CHKTEX 19
	            {⁰}{{\textsuperscript{0}}}1% CHKTEX 19
%	            {A}{{\textAlpha}}1
	            {α}{{\textalpha}}1%
%	            {B}{{\textBeta}}1
	            {β}{{\textbeta}}1%
	            {Γ}{{\textGamma}}1
	            {γ}{{\textgamma}}1%
	            {Δ}{{\textDelta}}1
	            {δ}{{\textdelta}}1% CHKTEX 19
%	            {E}{{\textEpsilon}}1
	            {ϵ}{{\textepsilon}}1%
%	            {Z}{{\textZeta}}1
	            {ζ}{{\textzeta}}1%
%	            {H}{{\textEta}}1
	            {η}{{\texteta}}1%
	            {Θ}{{\textTheta}}1
	            {θ}{{\texttheta}}1%
%	            {I}{{\textIota}}1
	            {ι}{{\textiota}}1%
%	            {K}{{\textKappa}}1
	            {κ}{{\textkappa}}1%
	            {Λ}{{\textLambda}}1
	            {λ}{{\textlambda}}1%
%	            {M}{{\textMu}}1
	            {μ}{{\textmu}}1%
%	            {N}{{\textNu}}1
	            {ν}{{\textnu}}1%
	            {Ξ}{{\textXi}}1
	            {ξ}{{\textxi}}1%
%	            {O}{{\textOmikron}}1
%	            {o}{{\textomikron}}1%
	            {Π}{{\textPi}}1
	            {π}{{\textpi}}1%
%	            {P}{{\textRho}}1
	            {ρ}{{\textrho}}1%
	            {Σ}{{\textSigma}}1
	            {σ}{{\textsigma}}1%
%	            {T}{{\textTau}}1
	            {τ}{{\texttau}}1%
	            {ϒ}{{\textUpsilon}}1
	            {υ}{{\textupsilon}}1%
	            {Φ}{{\textPhi}}1
	            {ϕ}{{\textphi}}1%
%	            {X}{{\textChi}}1
	            {χ}{{\textchi}}1%
	            {Ψ}{{\textPsi}}1
	            {ψ}{{\textpsi}}1%
	            {Ω}{{\textOmega}}1
	            {ω}{{\textomega}}1%
	            {ζ}{{\varsigma}}1%
%	            {}{{\straightphi}}1%
%	            {}{{\scripttheta}}1%
%	            {}{{\straighttheta}}1%
%	            {}{{\straightepsilon}}1%
	         },
}

\lstdefinestyle{c_with_comments}%
{
	language     = c,
	morecomment  = [l]{//},
	morecomment  = [s]{/*}{*/},
	breaklines,
}

\lstdefinestyle{c_without_comments}{%
	style        = c_with_comments,
	% numbers      = none,
	% keepspaces   = false,
	morecomment  = [l][\nullfont]{//},
	morecomment  = [is]{//}{\^^M},
	morecomment  = [is]{/*}{*/},
	emptylines   = *1,
}

\lstdefinestyle{py_without_comments}{%
	language     = python,
	morecomment  = [l][\nullfont]{\#},
	% morecomment  = [il]{\#},
	% morecomment  = [is]{\#}{\^^M},
	emptylines   = *1,
}

\lstdefinestyle{py_without_doclines}{%
	morecomment  = [is]{'''}{'''},%CHKTEX 23
	morecomment  = [is]{"""}{"""},%CHKTEX 18
	morecomment  = [is]{\#'''}{'''},%CHKTEX 23
	morecomment  = [is]{\#"""}{"""},%CHKTEX 18
}

\lstdefinelanguage{conf}
{
	basicstyle=\ttfamily\small,
	columns=fullflexible,
	morecomment=[s][\color{Orchid}\bfseries]{[}{]},
	morecomment=[l]{\#},
	morecomment=[l]{;},
	commentstyle=\color{gray}\ttfamily,
	% morekeywords={},
	% otherkeywords={=,:},
	% keywordstyle={\color{Green}\bfseries}
}

% \captionsetup[lstlisting]{font={small,tt}}
\captionsetup[lstlisting]{%
	font={small},
}



\DefineVerbatimEnvironment{Verbatim}{Verbatim}{%
	fontsize=\footnotesize,%
	frame=leftline,%
	framesep=2em,    % separation between frame and text
}

\RecustomVerbatimCommand{\VerbatimInput}{VerbatimInput}{%
	fontsize=\footnotesize,
%	frame=lines,            % top and bottom rule only
	frame=leftline,         % left rule only
	numbers=left,           % Line numbers on the left
	numbersep=0.25em,       % Gap between numbers and verbatim lines
	xleftmargin=4em,        % Indentation to add at the start of each line
	xrightmargin=4em,       % Right margin to add after each line
	framesep=0.5em,         % separation between frame and text
	rulecolor=\color{Gray}, % Color of the lines
	labelposition=topline,  %
	samepage=false,         % When true, prevents verbatim environment from
	                        % being broken between pages
%	commandchars=\|\(\),    % escape character and argument delimiters for
	                        % commands within the verbatim
%	commentchar=*           % comment character
}


% CHKTEX-FILE 1
% CHKTEX-FILE 13
% CHKTEX-FILE 18
% CHKTEX-FILE 35
\documentclass[letterpaper,12pt,twocolumn]{article}
% Input encodign

% %% %%%%%%%%%%%%%%%%%%%%%%%%%%%%%%%%%%%%%%%%%%%%%%%%%%%%%%%%%
%
% packages.tex
%
%  Author: Mauricio Matamoros
%  Date:   2020.02.28
%
%  Contiene la lista de paquetes requeridos para generar
%  el archivo-reporte de las prácticas de laboratorio
%
% %% %%%%%%%%%%%%%%%%%%%%%%%%%%%%%%%%%%%%%%%%%%%%%%%%%%%%%%%%%
% Archivo principal de LaTeX
%!TEX root = ../reporte.tex

\usepackage[utf8]{inputenc}                  % Soporte para utf8
\usepackage[T1]{fontenc}                     % Soporte extendido de caracteres unicode
\usepackage[english,spanish,mexico]{babel}   % Define el idioma del documento a español (México) con soporte para inglés
% Standard packages
\usepackage{float}                           % Imágenes flotantes en el documento
\usepackage{ifthen}                          % Soporte if-then en macros
\usepackage{xspace}                          % Soporte de autoespaciado en macros
\usepackage{xstring}                         % Operaciones con cadenas en macros
\usepackage{wrapfig}                         % Permite colocar texto al rededor de figuras y otros flotantes
\usepackage{booktabs}                        % Embellece tablas
\usepackage{csquotes}                        % Entrecomillado automático y manejo de citas textuales
\usepackage{fancyhdr}                        % Permite reconfigurar encabezado y pie de página
\usepackage{fancyvrb}                        % Define estilos para entornos Verbatim
\usepackage{geometry}                        % Permite reconfigurar la geometría del documento
\usepackage{graphicx}                        % Permite insertar imágenes en varios formatos
\usepackage{lastpage}                        % Referencia a la última página del documento
\usepackage{listings}                        % Define estilos para entornos de código de programación (sintaxis)
\usepackage{multicol}                        % Manejo de texto en varias columnas
\usepackage{tabularx}                        % Tablas con ancho de columna variable
\usepackage{algorithm}                       % Entorno para escribir algoritmos
\usepackage{algpseudocode}                   % Entorno para escribir algoritmos en pseudocódigo
\usepackage[justification=centering]{subcaption} % Permite imágenes en viñetas
\usepackage[all]{nowidow}                    % Control de viudas y huérfanas
\usepackage[inline]{enumitem}                % Añade opciones de configuración a listas
\usepackage[usenames,dvipsnames]{xcolor}     % Permite el uso de colores en el documento
% Referencing
\usepackage{varioref}                        % Gestión de referencias variables
\usepackage{hyperref}                        % Gestión de referencias e hipervínculos
\usepackage[noabbrev,nameinlink,spanish]{cleveref} % Gestión de referencias cruzadas inteligentes con hipervínculos
\usepackage[square, comma, numbers, sort&compress]{natbib} % Gestión de referencias bibliográficas


\newcommand{\lpar}{(}\newcommand{\rpar}{)} %CHKTEX 9
\newcommand{\IIC}{I\textsuperscript{2}C\xspace}
\newcommand{\GND}{\textsc{Gnd}\xspace}
\newcommand{\VCC}{\textsc{Vcc}\xspace}
\newcommand{\VDD}{\textsc{Vdd}\xspace}
\newcommand{\textbi}[1]{\textbf{\textit{#1}}}
\newcommand{\degreesC}[1]{%
	#1\textsuperscript{o}C\xspace{}%
}
\newcommand{\degreesF}[1]{%
	#1\textsuperscript{o}F\xspace{}%
}

% \newcommand{\VCC}{V\textsubscript{CC}\xspace{}}
% \newcommand{\GND}{\textsc{Gnd}\xspace{}}
% CHKTEX-FILE 1
% CHKTEX-FILE 13
% CHKTEX-FILE 18
% CHKTEX-FILE 35
\documentclass[letterpaper,12pt,twocolumn]{article}
% Input encodign

% %% %%%%%%%%%%%%%%%%%%%%%%%%%%%%%%%%%%%%%%%%%%%%%%%%%%%%%%%%%
%
% packages.tex
%
%  Author: Mauricio Matamoros
%  Date:   2020.02.28
%
%  Contiene la lista de paquetes requeridos para generar
%  el archivo-reporte de las prácticas de laboratorio
%
% %% %%%%%%%%%%%%%%%%%%%%%%%%%%%%%%%%%%%%%%%%%%%%%%%%%%%%%%%%%
% Archivo principal de LaTeX
%!TEX root = ../reporte.tex

\usepackage[utf8]{inputenc}                  % Soporte para utf8
\usepackage[T1]{fontenc}                     % Soporte extendido de caracteres unicode
\usepackage[english,spanish,mexico]{babel}   % Define el idioma del documento a español (México) con soporte para inglés
% Standard packages
\usepackage{float}                           % Imágenes flotantes en el documento
\usepackage{ifthen}                          % Soporte if-then en macros
\usepackage{xspace}                          % Soporte de autoespaciado en macros
\usepackage{xstring}                         % Operaciones con cadenas en macros
\usepackage{wrapfig}                         % Permite colocar texto al rededor de figuras y otros flotantes
\usepackage{booktabs}                        % Embellece tablas
\usepackage{csquotes}                        % Entrecomillado automático y manejo de citas textuales
\usepackage{fancyhdr}                        % Permite reconfigurar encabezado y pie de página
\usepackage{fancyvrb}                        % Define estilos para entornos Verbatim
\usepackage{geometry}                        % Permite reconfigurar la geometría del documento
\usepackage{graphicx}                        % Permite insertar imágenes en varios formatos
\usepackage{lastpage}                        % Referencia a la última página del documento
\usepackage{listings}                        % Define estilos para entornos de código de programación (sintaxis)
\usepackage{multicol}                        % Manejo de texto en varias columnas
\usepackage{tabularx}                        % Tablas con ancho de columna variable
\usepackage{algorithm}                       % Entorno para escribir algoritmos
\usepackage{algpseudocode}                   % Entorno para escribir algoritmos en pseudocódigo
\usepackage[justification=centering]{subcaption} % Permite imágenes en viñetas
\usepackage[all]{nowidow}                    % Control de viudas y huérfanas
\usepackage[inline]{enumitem}                % Añade opciones de configuración a listas
\usepackage[usenames,dvipsnames]{xcolor}     % Permite el uso de colores en el documento
% Referencing
\usepackage{varioref}                        % Gestión de referencias variables
\usepackage{hyperref}                        % Gestión de referencias e hipervínculos
\usepackage[noabbrev,nameinlink,spanish]{cleveref} % Gestión de referencias cruzadas inteligentes con hipervínculos
\usepackage[square, comma, numbers, sort&compress]{natbib} % Gestión de referencias bibliográficas


\newcommand{\lpar}{(}\newcommand{\rpar}{)} %CHKTEX 9
\newcommand{\IIC}{I\textsuperscript{2}C\xspace}
\newcommand{\GND}{\textsc{Gnd}\xspace}
\newcommand{\VCC}{\textsc{Vcc}\xspace}
\newcommand{\VDD}{\textsc{Vdd}\xspace}
\newcommand{\textbi}[1]{\textbf{\textit{#1}}}
\newcommand{\degreesC}[1]{%
	#1\textsuperscript{o}C\xspace{}%
}
\newcommand{\degreesF}[1]{%
	#1\textsuperscript{o}F\xspace{}%
}

% \newcommand{\VCC}{V\textsubscript{CC}\xspace{}}
% \newcommand{\GND}{\textsc{Gnd}\xspace{}}
% CHKTEX-FILE 1
% CHKTEX-FILE 13
% CHKTEX-FILE 18
% CHKTEX-FILE 35
\documentclass[letterpaper,12pt,twocolumn]{article}
% Input encodign

\input{setup/packages}
\input{setup/macros}
\input{setup/document}
\input{setup/listings}

\author{\footnotesize Autor: José Mauricio Matamoros de Maria y Campos}
\title{Especificación para reportes de lectura}
\date{}


% Document body
\begin{document}
\maketitle

\section*{Reportes de lectura}
El objetivo de las lecturas asignadas como tarea es fomentar en el alumno el hábito de la lectura, tanto de documentos técnicos como literarios, ya sean estos en español o en inglés, además de incrementar las habilidades de comprensión de lectura, concreción de ideas, síntesis y redacción de texto en prosa.

Leer es muy importante en el aprendizaje pues la mayor parte de lo que se aprende se hace por medio de la lectura.
Cuando se quiere aprender algo, lo primero que se hace es leer.

Cuando se lee un texto con fines de estudio se resaltan las ideas importantes, se resume, se separan las ideas principales y se formulan cuestionarios para mejor comprender el texto. Cuando se lee por entretenimiento (por ejemplo un cuento o una novela), el texto se analiza y se relacionan hechos y situaciones a fin de poder disfrutar de él.

Un reporte de lectura es un informe escrito acerca del texto que se leyó.
Este informe debe contener los siguientes datos:

\begin{itemize}[noitemsep]
	\item Título del texto y nombre del autor
	\item Tema o asunto que trata
	\item Resumen, síntesis o reseña del texto
	\item Análisis crítico del contenido del texto
	\item Opinión personal del contenido de la lectura
	\item Conclusiones de la lectura
\end{itemize}

En el caso de texto literario no se espera un resumen, síntesis o reseña del texto, sino que deberá comentarse sobre la temática del mismo y resaltar los puntos más importantes, dejando claras sus impresiones y comentarios sobre el texto.

Tome en cuenta que ni la síntesis ni los resúmenes implican copiar y pegar párrafos enteros del texto. Lo importante es distinguir las ideas principales, hacerlas propias y (salvo que sean definiciones), parafrasearlas, es decir,  expresarlas en sus propias palabras.

Pasos para elaborar un reporte de lectura
\begin{enumerate}[noitemsep]
	\item Lea atentamente el texto. Evite distractores como música fuerte, televisión, teléfono, etc.
	\item Anote los términos y palabras que no conozca o entienda e investigue su significado en un diccionario.
	\item Una vez aclarados los términos desconocidos y palabras nuevas, vuelva a leer el texto para comprenderlo mejor.
	\item En la segunda lectura subraye o marque las ideas principales del texto.
	%Procure no escribir en los libros, especialmente si no le pertenecen. Si desea hacer anotaciones, utilice un lápiz blando y escriba con suavidad, o de preferencia use fotocopias.
	\item Tome las ideas principales, abstraígalas, analícelas y sólo entonces redacte su escrito.
\end{enumerate}
\end{document}

%CHKTEX-FILE 1
%CHKTEX-FILE 7
%CHKTEX-FILE 9
% Default fixed font does not support bold face
\DeclareFixedFont{\ttb}{T1}{txtt}{bx}{n}{8} % for bold
\DeclareFixedFont{\ttm}{T1}{txtt}{m}{n}{8}  % for normal

% Custom colors
\usepackage{color}
\definecolor{keywordsColor}{rgb}{0,0,0.5}
\definecolor{customColor}{rgb}{0.6,0,0}
\definecolor{stringColor}{rgb}{0,0.5,0}

% Code highlighting python
\renewcommand{\ttdefault}{pcr}
\lstset{
	language=Python,                              % the language of the code (can be overrided per snippet)
	backgroundcolor=\color{white},                % choose the background color
	basicstyle=\footnotesize\ttfamily,            % the size of the fonts that are used for the code
	breakatwhitespace=false,                      % sets if automatic breaks should only happen at whitespace
	breaklines=true,                              % sets automatic line breaking
	captionpos=t,                                 % sets the caption-position to bottom
	commentstyle=\color{gray},                    % comment style
	deletekeywords={},                            % if you want to delete keywords from the given language
%	escapeinside={\%*}{*)},                       % if you want to add LaTeX within your code
	extendedchars=true,                           % lets you use non-ASCII characters; for 8-bits encodings only, does not work with UTF-8
	frame=tb,                                     % adds a frame around the code
	keepspaces=true,                              % keeps spaces in text, useful for keeping indentation of code (possibly needs columns=flexible)
	keywordstyle=\color{keywordsColor}\bfseries,  % keyword style
	numbers=left,                                 % where to put the line-numbers; possible values are (none, left, right)
	numbersep=5pt,                                % how far the line-numbers are from the code
	numberstyle=\tiny\color{gray},                % the style that is used for the line-numbers
	rulecolor=\color{black},                      % if not set, the frame-color may be changed on line-breaks within not-black text (e.g. comments (green here))
	showspaces=false,                             % show spaces everywhere adding particular underscores; it overrides 'showstringspaces'
	showstringspaces=false,                       % underline spaces within strings only
	showtabs=false,                               % show tabs within strings adding particular underscores
	stepnumber=1,                                 % the step between two line-numbers. If it's 1, each line will be numbered
	stringstyle=\color{stringColor},              % string literal style
	tabsize=2,                                    % sets default tabsize to 2 spaces
	title=\lstname,                               % show the filename of files included with \lstinputlisting; also try caption instead of title
	columns=fixed,                                % Using fixed column width (for e.g. nice alignment)
	otherkeywords={self},                         % if you want to add more keywords to the set
	emphstyle=\color{customColor}\bfseries,       % Custom highlighting style
	emph={__init__,__main__,True,False,None},     % Custom highlighting keywords
	xleftmargin=1cm,                              % Left margin
	xrightmargin=1cm,                             % Right margin
	% Unicode compatibility
	inputencoding=utf8,
	literate={%
	            {Á}{{\'a}}1 {É}{{\'E}}1 {Í}{{\'I}}1 {Ó}{{\'O}}1 {Ú}{{\'U}}1%
	            {á}{{\'a}}1 {é}{{\'e}}1 {í}{{\'i}}1 {ó}{{\'o}}1 {ú}{{\'u}}1%
	            {À}{{\`A}}1 {È}{{\'E}}1 {Ì}{{\`I}}1 {Ò}{{\`O}}1 {Ù}{{\`U}}1%
	            {à}{{\`a}}1 {è}{{\`e}}1 {ì}{{\`i}}1 {ò}{{\`o}}1 {ù}{{\`u}}1%
	            {Ä}{{\"A}}1 {Ë}{{\"E}}1 {Ï}{{\"I}}1 {Ö}{{\"O}}1 {Ü}{{\"U}}1%
	            {ä}{{\"a}}1 {ë}{{\"e}}1 {ï}{{\"i}}1 {ö}{{\"o}}1 {ü}{{\"u}}1%
	            {Â}{{\^A}}1 {Ê}{{\^E}}1 {Î}{{\^I}}1 {Ô}{{\^O}}1 {Û}{{\^U}}1%
	            {â}{{\^a}}1 {ê}{{\^e}}1 {î}{{\^i}}1 {ô}{{\^o}}1 {û}{{\^u}}1% CHKTEX 19
	            {Ã}{{\~a}}1 {Ẽ}{{\~E}}1 {Ĩ}{{\~I}}1 {Õ}{{\~O}}1 {Ũ}{{\~U}}1 {Ñ}{{\~N}}1%
	            {ã}{{\~a}}1 {ẽ}{{\~e}}1 {ĩ}{{\~i}}1 {õ}{{\~o}}1 {ũ}{{\~u}}1 {ñ}{{\~n}}1%
	            {œ}{{\oe}}1 {Œ}{{\OE}}1 {æ}{{\ae}}1 {Æ}{{\AE}}1 {ß}{{\ss}}1%
	            {ç}{{\c c}}1 {Ç}{{\c C}}1 {ø}{{\o}}1 {å}{{\r a}}1 {Å}{{\r A}}1%
	            {€}{{\EUR}}1 {£}{{\pounds}}1 {×}{{\(\times\)}}1% CHKTEX 21
	            {°}{{\textsuperscript{o}}}1%
	            {¹}{{\textsuperscript{1}}}1%
	            {²}{{\textsuperscript{2}}}1%
	            {³}{{\textsuperscript{3}}}1%
	            {⁴}{{\textsuperscript{4}}}1% CHKTEX 19
	            {⁵}{{\textsuperscript{5}}}1% CHKTEX 19
	            {⁶}{{\textsuperscript{6}}}1% CHKTEX 19
	            {⁷}{{\textsuperscript{7}}}1% CHKTEX 19
	            {⁸}{{\textsuperscript{8}}}1% CHKTEX 19
	            {⁹}{{\textsuperscript{9}}}1% CHKTEX 19
	            {⁰}{{\textsuperscript{0}}}1% CHKTEX 19
%	            {A}{{\textAlpha}}1
	            {α}{{\textalpha}}1%
%	            {B}{{\textBeta}}1
	            {β}{{\textbeta}}1%
	            {Γ}{{\textGamma}}1
	            {γ}{{\textgamma}}1%
	            {Δ}{{\textDelta}}1
	            {δ}{{\textdelta}}1% CHKTEX 19
%	            {E}{{\textEpsilon}}1
	            {ϵ}{{\textepsilon}}1%
%	            {Z}{{\textZeta}}1
	            {ζ}{{\textzeta}}1%
%	            {H}{{\textEta}}1
	            {η}{{\texteta}}1%
	            {Θ}{{\textTheta}}1
	            {θ}{{\texttheta}}1%
%	            {I}{{\textIota}}1
	            {ι}{{\textiota}}1%
%	            {K}{{\textKappa}}1
	            {κ}{{\textkappa}}1%
	            {Λ}{{\textLambda}}1
	            {λ}{{\textlambda}}1%
%	            {M}{{\textMu}}1
	            {μ}{{\textmu}}1%
%	            {N}{{\textNu}}1
	            {ν}{{\textnu}}1%
	            {Ξ}{{\textXi}}1
	            {ξ}{{\textxi}}1%
%	            {O}{{\textOmikron}}1
%	            {o}{{\textomikron}}1%
	            {Π}{{\textPi}}1
	            {π}{{\textpi}}1%
%	            {P}{{\textRho}}1
	            {ρ}{{\textrho}}1%
	            {Σ}{{\textSigma}}1
	            {σ}{{\textsigma}}1%
%	            {T}{{\textTau}}1
	            {τ}{{\texttau}}1%
	            {ϒ}{{\textUpsilon}}1
	            {υ}{{\textupsilon}}1%
	            {Φ}{{\textPhi}}1
	            {ϕ}{{\textphi}}1%
%	            {X}{{\textChi}}1
	            {χ}{{\textchi}}1%
	            {Ψ}{{\textPsi}}1
	            {ψ}{{\textpsi}}1%
	            {Ω}{{\textOmega}}1
	            {ω}{{\textomega}}1%
	            {ζ}{{\varsigma}}1%
%	            {}{{\straightphi}}1%
%	            {}{{\scripttheta}}1%
%	            {}{{\straighttheta}}1%
%	            {}{{\straightepsilon}}1%
	         },
}

\lstdefinestyle{c_with_comments}%
{
	language     = c,
	morecomment  = [l]{//},
	morecomment  = [s]{/*}{*/},
	breaklines,
}

\lstdefinestyle{c_without_comments}{%
	style        = c_with_comments,
	% numbers      = none,
	% keepspaces   = false,
	morecomment  = [l][\nullfont]{//},
	morecomment  = [is]{//}{\^^M},
	morecomment  = [is]{/*}{*/},
	emptylines   = *1,
}

\lstdefinestyle{py_without_comments}{%
	language     = python,
	morecomment  = [l][\nullfont]{\#},
	% morecomment  = [il]{\#},
	% morecomment  = [is]{\#}{\^^M},
	emptylines   = *1,
}

\lstdefinestyle{py_without_doclines}{%
	morecomment  = [is]{'''}{'''},%CHKTEX 23
	morecomment  = [is]{"""}{"""},%CHKTEX 18
	morecomment  = [is]{\#'''}{'''},%CHKTEX 23
	morecomment  = [is]{\#"""}{"""},%CHKTEX 18
}

\lstdefinelanguage{conf}
{
	basicstyle=\ttfamily\small,
	columns=fullflexible,
	morecomment=[s][\color{Orchid}\bfseries]{[}{]},
	morecomment=[l]{\#},
	morecomment=[l]{;},
	commentstyle=\color{gray}\ttfamily,
	% morekeywords={},
	% otherkeywords={=,:},
	% keywordstyle={\color{Green}\bfseries}
}

% \captionsetup[lstlisting]{font={small,tt}}
\captionsetup[lstlisting]{%
	font={small},
}



\DefineVerbatimEnvironment{Verbatim}{Verbatim}{%
	fontsize=\footnotesize,%
	frame=leftline,%
	framesep=2em,    % separation between frame and text
}

\RecustomVerbatimCommand{\VerbatimInput}{VerbatimInput}{%
	fontsize=\footnotesize,
%	frame=lines,            % top and bottom rule only
	frame=leftline,         % left rule only
	numbers=left,           % Line numbers on the left
	numbersep=0.25em,       % Gap between numbers and verbatim lines
	xleftmargin=4em,        % Indentation to add at the start of each line
	xrightmargin=4em,       % Right margin to add after each line
	framesep=0.5em,         % separation between frame and text
	rulecolor=\color{Gray}, % Color of the lines
	labelposition=topline,  %
	samepage=false,         % When true, prevents verbatim environment from
	                        % being broken between pages
%	commandchars=\|\(\),    % escape character and argument delimiters for
	                        % commands within the verbatim
%	commentchar=*           % comment character
}



\author{\footnotesize Autor: José Mauricio Matamoros de Maria y Campos}
\title{Especificación para reportes de lectura}
\date{}


% Document body
\begin{document}
\maketitle

\section*{Reportes de lectura}
El objetivo de las lecturas asignadas como tarea es fomentar en el alumno el hábito de la lectura, tanto de documentos técnicos como literarios, ya sean estos en español o en inglés, además de incrementar las habilidades de comprensión de lectura, concreción de ideas, síntesis y redacción de texto en prosa.

Leer es muy importante en el aprendizaje pues la mayor parte de lo que se aprende se hace por medio de la lectura.
Cuando se quiere aprender algo, lo primero que se hace es leer.

Cuando se lee un texto con fines de estudio se resaltan las ideas importantes, se resume, se separan las ideas principales y se formulan cuestionarios para mejor comprender el texto. Cuando se lee por entretenimiento (por ejemplo un cuento o una novela), el texto se analiza y se relacionan hechos y situaciones a fin de poder disfrutar de él.

Un reporte de lectura es un informe escrito acerca del texto que se leyó.
Este informe debe contener los siguientes datos:

\begin{itemize}[noitemsep]
	\item Título del texto y nombre del autor
	\item Tema o asunto que trata
	\item Resumen, síntesis o reseña del texto
	\item Análisis crítico del contenido del texto
	\item Opinión personal del contenido de la lectura
	\item Conclusiones de la lectura
\end{itemize}

En el caso de texto literario no se espera un resumen, síntesis o reseña del texto, sino que deberá comentarse sobre la temática del mismo y resaltar los puntos más importantes, dejando claras sus impresiones y comentarios sobre el texto.

Tome en cuenta que ni la síntesis ni los resúmenes implican copiar y pegar párrafos enteros del texto. Lo importante es distinguir las ideas principales, hacerlas propias y (salvo que sean definiciones), parafrasearlas, es decir,  expresarlas en sus propias palabras.

Pasos para elaborar un reporte de lectura
\begin{enumerate}[noitemsep]
	\item Lea atentamente el texto. Evite distractores como música fuerte, televisión, teléfono, etc.
	\item Anote los términos y palabras que no conozca o entienda e investigue su significado en un diccionario.
	\item Una vez aclarados los términos desconocidos y palabras nuevas, vuelva a leer el texto para comprenderlo mejor.
	\item En la segunda lectura subraye o marque las ideas principales del texto.
	%Procure no escribir en los libros, especialmente si no le pertenecen. Si desea hacer anotaciones, utilice un lápiz blando y escriba con suavidad, o de preferencia use fotocopias.
	\item Tome las ideas principales, abstraígalas, analícelas y sólo entonces redacte su escrito.
\end{enumerate}
\end{document}

%CHKTEX-FILE 1
%CHKTEX-FILE 7
%CHKTEX-FILE 9
% Default fixed font does not support bold face
\DeclareFixedFont{\ttb}{T1}{txtt}{bx}{n}{8} % for bold
\DeclareFixedFont{\ttm}{T1}{txtt}{m}{n}{8}  % for normal

% Custom colors
\usepackage{color}
\definecolor{keywordsColor}{rgb}{0,0,0.5}
\definecolor{customColor}{rgb}{0.6,0,0}
\definecolor{stringColor}{rgb}{0,0.5,0}

% Code highlighting python
\renewcommand{\ttdefault}{pcr}
\lstset{
	language=Python,                              % the language of the code (can be overrided per snippet)
	backgroundcolor=\color{white},                % choose the background color
	basicstyle=\footnotesize\ttfamily,            % the size of the fonts that are used for the code
	breakatwhitespace=false,                      % sets if automatic breaks should only happen at whitespace
	breaklines=true,                              % sets automatic line breaking
	captionpos=t,                                 % sets the caption-position to bottom
	commentstyle=\color{gray},                    % comment style
	deletekeywords={},                            % if you want to delete keywords from the given language
%	escapeinside={\%*}{*)},                       % if you want to add LaTeX within your code
	extendedchars=true,                           % lets you use non-ASCII characters; for 8-bits encodings only, does not work with UTF-8
	frame=tb,                                     % adds a frame around the code
	keepspaces=true,                              % keeps spaces in text, useful for keeping indentation of code (possibly needs columns=flexible)
	keywordstyle=\color{keywordsColor}\bfseries,  % keyword style
	numbers=left,                                 % where to put the line-numbers; possible values are (none, left, right)
	numbersep=5pt,                                % how far the line-numbers are from the code
	numberstyle=\tiny\color{gray},                % the style that is used for the line-numbers
	rulecolor=\color{black},                      % if not set, the frame-color may be changed on line-breaks within not-black text (e.g. comments (green here))
	showspaces=false,                             % show spaces everywhere adding particular underscores; it overrides 'showstringspaces'
	showstringspaces=false,                       % underline spaces within strings only
	showtabs=false,                               % show tabs within strings adding particular underscores
	stepnumber=1,                                 % the step between two line-numbers. If it's 1, each line will be numbered
	stringstyle=\color{stringColor},              % string literal style
	tabsize=2,                                    % sets default tabsize to 2 spaces
	title=\lstname,                               % show the filename of files included with \lstinputlisting; also try caption instead of title
	columns=fixed,                                % Using fixed column width (for e.g. nice alignment)
	otherkeywords={self},                         % if you want to add more keywords to the set
	emphstyle=\color{customColor}\bfseries,       % Custom highlighting style
	emph={__init__,__main__,True,False,None},     % Custom highlighting keywords
	xleftmargin=1cm,                              % Left margin
	xrightmargin=1cm,                             % Right margin
	% Unicode compatibility
	inputencoding=utf8,
	literate={%
	            {Á}{{\'a}}1 {É}{{\'E}}1 {Í}{{\'I}}1 {Ó}{{\'O}}1 {Ú}{{\'U}}1%
	            {á}{{\'a}}1 {é}{{\'e}}1 {í}{{\'i}}1 {ó}{{\'o}}1 {ú}{{\'u}}1%
	            {À}{{\`A}}1 {È}{{\'E}}1 {Ì}{{\`I}}1 {Ò}{{\`O}}1 {Ù}{{\`U}}1%
	            {à}{{\`a}}1 {è}{{\`e}}1 {ì}{{\`i}}1 {ò}{{\`o}}1 {ù}{{\`u}}1%
	            {Ä}{{\"A}}1 {Ë}{{\"E}}1 {Ï}{{\"I}}1 {Ö}{{\"O}}1 {Ü}{{\"U}}1%
	            {ä}{{\"a}}1 {ë}{{\"e}}1 {ï}{{\"i}}1 {ö}{{\"o}}1 {ü}{{\"u}}1%
	            {Â}{{\^A}}1 {Ê}{{\^E}}1 {Î}{{\^I}}1 {Ô}{{\^O}}1 {Û}{{\^U}}1%
	            {â}{{\^a}}1 {ê}{{\^e}}1 {î}{{\^i}}1 {ô}{{\^o}}1 {û}{{\^u}}1% CHKTEX 19
	            {Ã}{{\~a}}1 {Ẽ}{{\~E}}1 {Ĩ}{{\~I}}1 {Õ}{{\~O}}1 {Ũ}{{\~U}}1 {Ñ}{{\~N}}1%
	            {ã}{{\~a}}1 {ẽ}{{\~e}}1 {ĩ}{{\~i}}1 {õ}{{\~o}}1 {ũ}{{\~u}}1 {ñ}{{\~n}}1%
	            {œ}{{\oe}}1 {Œ}{{\OE}}1 {æ}{{\ae}}1 {Æ}{{\AE}}1 {ß}{{\ss}}1%
	            {ç}{{\c c}}1 {Ç}{{\c C}}1 {ø}{{\o}}1 {å}{{\r a}}1 {Å}{{\r A}}1%
	            {€}{{\EUR}}1 {£}{{\pounds}}1 {×}{{\(\times\)}}1% CHKTEX 21
	            {°}{{\textsuperscript{o}}}1%
	            {¹}{{\textsuperscript{1}}}1%
	            {²}{{\textsuperscript{2}}}1%
	            {³}{{\textsuperscript{3}}}1%
	            {⁴}{{\textsuperscript{4}}}1% CHKTEX 19
	            {⁵}{{\textsuperscript{5}}}1% CHKTEX 19
	            {⁶}{{\textsuperscript{6}}}1% CHKTEX 19
	            {⁷}{{\textsuperscript{7}}}1% CHKTEX 19
	            {⁸}{{\textsuperscript{8}}}1% CHKTEX 19
	            {⁹}{{\textsuperscript{9}}}1% CHKTEX 19
	            {⁰}{{\textsuperscript{0}}}1% CHKTEX 19
%	            {A}{{\textAlpha}}1
	            {α}{{\textalpha}}1%
%	            {B}{{\textBeta}}1
	            {β}{{\textbeta}}1%
	            {Γ}{{\textGamma}}1
	            {γ}{{\textgamma}}1%
	            {Δ}{{\textDelta}}1
	            {δ}{{\textdelta}}1% CHKTEX 19
%	            {E}{{\textEpsilon}}1
	            {ϵ}{{\textepsilon}}1%
%	            {Z}{{\textZeta}}1
	            {ζ}{{\textzeta}}1%
%	            {H}{{\textEta}}1
	            {η}{{\texteta}}1%
	            {Θ}{{\textTheta}}1
	            {θ}{{\texttheta}}1%
%	            {I}{{\textIota}}1
	            {ι}{{\textiota}}1%
%	            {K}{{\textKappa}}1
	            {κ}{{\textkappa}}1%
	            {Λ}{{\textLambda}}1
	            {λ}{{\textlambda}}1%
%	            {M}{{\textMu}}1
	            {μ}{{\textmu}}1%
%	            {N}{{\textNu}}1
	            {ν}{{\textnu}}1%
	            {Ξ}{{\textXi}}1
	            {ξ}{{\textxi}}1%
%	            {O}{{\textOmikron}}1
%	            {o}{{\textomikron}}1%
	            {Π}{{\textPi}}1
	            {π}{{\textpi}}1%
%	            {P}{{\textRho}}1
	            {ρ}{{\textrho}}1%
	            {Σ}{{\textSigma}}1
	            {σ}{{\textsigma}}1%
%	            {T}{{\textTau}}1
	            {τ}{{\texttau}}1%
	            {ϒ}{{\textUpsilon}}1
	            {υ}{{\textupsilon}}1%
	            {Φ}{{\textPhi}}1
	            {ϕ}{{\textphi}}1%
%	            {X}{{\textChi}}1
	            {χ}{{\textchi}}1%
	            {Ψ}{{\textPsi}}1
	            {ψ}{{\textpsi}}1%
	            {Ω}{{\textOmega}}1
	            {ω}{{\textomega}}1%
	            {ζ}{{\varsigma}}1%
%	            {}{{\straightphi}}1%
%	            {}{{\scripttheta}}1%
%	            {}{{\straighttheta}}1%
%	            {}{{\straightepsilon}}1%
	         },
}

\lstdefinestyle{c_with_comments}%
{
	language     = c,
	morecomment  = [l]{//},
	morecomment  = [s]{/*}{*/},
	breaklines,
}

\lstdefinestyle{c_without_comments}{%
	style        = c_with_comments,
	% numbers      = none,
	% keepspaces   = false,
	morecomment  = [l][\nullfont]{//},
	morecomment  = [is]{//}{\^^M},
	morecomment  = [is]{/*}{*/},
	emptylines   = *1,
}

\lstdefinestyle{py_without_comments}{%
	language     = python,
	morecomment  = [l][\nullfont]{\#},
	% morecomment  = [il]{\#},
	% morecomment  = [is]{\#}{\^^M},
	emptylines   = *1,
}

\lstdefinestyle{py_without_doclines}{%
	morecomment  = [is]{'''}{'''},%CHKTEX 23
	morecomment  = [is]{"""}{"""},%CHKTEX 18
	morecomment  = [is]{\#'''}{'''},%CHKTEX 23
	morecomment  = [is]{\#"""}{"""},%CHKTEX 18
}

\lstdefinelanguage{conf}
{
	basicstyle=\ttfamily\small,
	columns=fullflexible,
	morecomment=[s][\color{Orchid}\bfseries]{[}{]},
	morecomment=[l]{\#},
	morecomment=[l]{;},
	commentstyle=\color{gray}\ttfamily,
	% morekeywords={},
	% otherkeywords={=,:},
	% keywordstyle={\color{Green}\bfseries}
}

% \captionsetup[lstlisting]{font={small,tt}}
\captionsetup[lstlisting]{%
	font={small},
}



\DefineVerbatimEnvironment{Verbatim}{Verbatim}{%
	fontsize=\footnotesize,%
	frame=leftline,%
	framesep=2em,    % separation between frame and text
}

\RecustomVerbatimCommand{\VerbatimInput}{VerbatimInput}{%
	fontsize=\footnotesize,
%	frame=lines,            % top and bottom rule only
	frame=leftline,         % left rule only
	numbers=left,           % Line numbers on the left
	numbersep=0.25em,       % Gap between numbers and verbatim lines
	xleftmargin=4em,        % Indentation to add at the start of each line
	xrightmargin=4em,       % Right margin to add after each line
	framesep=0.5em,         % separation between frame and text
	rulecolor=\color{Gray}, % Color of the lines
	labelposition=topline,  %
	samepage=false,         % When true, prevents verbatim environment from
	                        % being broken between pages
%	commandchars=\|\(\),    % escape character and argument delimiters for
	                        % commands within the verbatim
%	commentchar=*           % comment character
}



\author{\footnotesize Autor: José Mauricio Matamoros de Maria y Campos}
\title{Especificación para reportes de lectura}
\date{}


% Document body
\begin{document}
\maketitle

\section*{Reportes de lectura}
El objetivo de las lecturas asignadas como tarea es fomentar en el alumno el hábito de la lectura, tanto de documentos técnicos como literarios, ya sean estos en español o en inglés, además de incrementar las habilidades de comprensión de lectura, concreción de ideas, síntesis y redacción de texto en prosa.

Leer es muy importante en el aprendizaje pues la mayor parte de lo que se aprende se hace por medio de la lectura.
Cuando se quiere aprender algo, lo primero que se hace es leer.

Cuando se lee un texto con fines de estudio se resaltan las ideas importantes, se resume, se separan las ideas principales y se formulan cuestionarios para mejor comprender el texto. Cuando se lee por entretenimiento (por ejemplo un cuento o una novela), el texto se analiza y se relacionan hechos y situaciones a fin de poder disfrutar de él.

Un reporte de lectura es un informe escrito acerca del texto que se leyó.
Este informe debe contener los siguientes datos:

\begin{itemize}[noitemsep]
	\item Título del texto y nombre del autor
	\item Tema o asunto que trata
	\item Resumen, síntesis o reseña del texto
	\item Análisis crítico del contenido del texto
	\item Opinión personal del contenido de la lectura
	\item Conclusiones de la lectura
\end{itemize}

En el caso de texto literario no se espera un resumen, síntesis o reseña del texto, sino que deberá comentarse sobre la temática del mismo y resaltar los puntos más importantes, dejando claras sus impresiones y comentarios sobre el texto.

Tome en cuenta que ni la síntesis ni los resúmenes implican copiar y pegar párrafos enteros del texto. Lo importante es distinguir las ideas principales, hacerlas propias y (salvo que sean definiciones), parafrasearlas, es decir,  expresarlas en sus propias palabras.

Pasos para elaborar un reporte de lectura
\begin{enumerate}[noitemsep]
	\item Lea atentamente el texto. Evite distractores como música fuerte, televisión, teléfono, etc.
	\item Anote los términos y palabras que no conozca o entienda e investigue su significado en un diccionario.
	\item Una vez aclarados los términos desconocidos y palabras nuevas, vuelva a leer el texto para comprenderlo mejor.
	\item En la segunda lectura subraye o marque las ideas principales del texto.
	%Procure no escribir en los libros, especialmente si no le pertenecen. Si desea hacer anotaciones, utilice un lápiz blando y escriba con suavidad, o de preferencia use fotocopias.
	\item Tome las ideas principales, abstraígalas, analícelas y sólo entonces redacte su escrito.
\end{enumerate}
\end{document}


\makeatletter
\hypersetup{%
	pdftitle={FSEm: Tarea 8},%CHKTEX 13
	pdfsubject={Fundamentos de Sistemas Embebidos},
	pdfauthor={Mauricio Matamoros}
}
\makeatother

\title{Tarea 8:\\Cuando algo huele raro\\
{\large Fundamentos de Sistemas Embebidos}}
\author{\footnotesize Autor: José Mauricio Matamoros de Maria y Campos}
% \date{Entrega: Lunes 8 de Marzo, 2021}
\date{}



% Document body
\begin{document}
\maketitle
\thispagestyle{empty}

% \section{Introducción}
% \label{sec:introduction}

\section{Instrucciones}%
\label{sec:instructions}
Lea detenidamente el Capítulo 17: \emph{Smells} del libro Clean Code de Robert C. Martin y escriba una breve síntesis de aproximadamente 250 palabras donde se discutan los puntos más relevantes presentados por el autor.
% Ponga especial atención en
% \begin{enumerate*}[label=\roman*\rpar]
% 	\item qué el código se ordena en funciones,
% 	\item por qué es importante limitar el alcance de las funciones
% 	y
% 	\item cómo se diferencia una función bien diseñada de una mal diseñada.
% \end{enumerate*}

\paragraph{Especificaciones técnicas}
\begin{itemize}
	\item Sin carátula.
	\item La síntesis deberá ser de al menos media cuartilla.
	\item La longitud máxima del texto no deberá exceder de una cuartilla.
	\item No se requieren citas ni bibliografía.
\end{itemize}

\paragraph{Rúbrica}
\begin{itemize}
	\item \points{100 puntos} Calificación base
	\item \points{-10 puntos} Cada error ortográfico o gramatical
\end{itemize}
Aplican criterios estándar para entregables y documentos escritos de acuerdo con el documento general de evaluación.

\end{document}
