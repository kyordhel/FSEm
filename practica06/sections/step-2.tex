% %% %%%%%%%%%%%%%%%%%%%%%%%%%%%%%%%%%%%%%%%%%%%%%%%%%%%%%%%%%%
% step-2.tex
%
% Author:  Mauricio Matamoros
% License: MIT
%
% %% %%%%%%%%%%%%%%%%%%%%%%%%%%%%%%%%%%%%%%%%%%%%%%%%%%%%%%%%%%

%!TEX root = ../practica.tex
%!TEX root = ../references.bib

% CHKTEX-FILE 1
% CHKTEX-FILE 13
% CHKTEX-FILE 46

\subsection{Paso 2: Lectura del sensor LM35}%
\label{sec:step2}
Antes de proceder, verifique conexiones con un multímetro en busca de corto circuitos.
% En particular verifique que exista una impedancia muy alta entre los pines 4 y 6 (\VCC y \GND) de la Raspberry Pi, así como entre los pines 5V, \GND y \textsc{AREF} del Arduino.
En particular verifique que exista una impedancia muy alta entre los pines 5V, \GND y \textsc{Aref} del Arduino.

Para leer la temperatura con el Arduino se necesitan convertir los valores discretos leídos por el ADC del microcontrolador en valores de temperatura.
Esto se puede realizar mediante un simple análisis debido a la linearidad del LM35.
Se tienen dos lecturas en el ADC: V\textsubscript{OUT+} y V\textsubscript{OUT-}, de las cuales la segunda es la referencia del LM35 y por lo tanto, la diferencia entre estos voltajes será proporcional a la temperatura en escala centígrada.
Esto expresado matemáticamente es:

\begin{equation*}
	T[^{o}C] \propto V_{diff} = V_{OUT+}-V_{OUT-}
\end{equation*}
\noindent o bien
\begin{equation*}
	T[^{o}C] = k \times V_{diff} = k \times \big( V_{OUT+}-V_{OUT-} \big)
\end{equation*}

\noindent lo que implica que en $T = 0^{o}C; V_{OUT+}=V_{OUT-} \rightarrow V_{diff} = 0$

\medskip
Es necesario entonces calcular la constante de proporcionalidad $k$.
Sabemos que el ADC entregará lecturas de 0 a 1024 para los voltajes registradoes entre \GND y \textsc{Aref} (0V y 2.72V respectivamente), además de que $1^{o}C = 0.01V$.
Luego entonces
\begin{align*}
	T[^{o}C] &= V_{diff} \times \frac{2.72[V]}{ 1024 \times 0.01[\tfrac{V}{^{o}C}] }  \\
	T[^{o}C] &= V_{diff} \times \frac{2.72}{ 10.24 }[^{o}C]  \\
\end{align*}

\noindent o bien, generalizando para todo voltaje de referencia:
\begin{equation*}
	T[^{o}C] = V_{diff} \times \frac{A_{REF}}{ 10.24 }[^{o}C]
\end{equation*}

Esta fórmula de conversión de unidades deberá programarse en el microcontrolador que adquiera los valores discretos de temperatura del sensor.

\medskip
\begin{greenbox}{\large TIP: Calibración}
	El voltaje de referencia $V_{AREF}$ variará respecto a su valor teórico dependiendo de la tolerancia de las resistencias (típicamente ±5\%) y de las impedancias de los demás componentes conectados, incluyendo al Arduino mismo.

	\smallskip

	Se recomienda usar siempre un multímetro para corroborar el voltaje entre $V_{AREF}$ y tierra y poder actualizar dicho valor en el código.\footnotemark{}
\end{greenbox}
\footnotetext{El valor real de $V_{AREF}$ variará entre 2.4V y 3.1V con resistencias con tolerancia de 5\%, equivalente a una variación de 70°C.}
\medskip

El programa de ejemplo para el Arduino\footnotemark{} se presenta a continuación:

\begin{minipage}{\linewidth}
\lstinputlisting[%
	language=C,
	caption={\texttt{arduino-code.cpp}:40--54, \texttt{read\_temp} function},
	label={lst:arduino-code-readtemp},
	style=c_without_comments,
	showlines=false,
	firstline=40,
	lastline=54]{src/arduino-code.cpp}
\end{minipage}


En ocasiones los valores pueden fluctuar ligeramente debido a ruido o variaciones de voltaje.
Para evitar este tipo de imprecisiones es común utilizar técnicas de filtrado, y uno de los métodos más simples y comúnes es el promedio de varias lecturas consecutivas tal y como se muestra a continuación:

\begin{minipage}{\linewidth}
\lstinputlisting[%
	language=C,
	caption={\texttt{arduino-code.cpp}:59--64, \texttt{read\_avg\_temp} function},
	label={lst:arduino-code-readavgtemp},
	style=c_without_comments,
	firstline=59,
	lastline=64]{src/arduino-code.cpp}
\end{minipage}

Otro método mucho más eficaz y seguro es llevar un registro de las últimas $N$ lecturas del sensor en un buffer circular y estimar el siguiente valor probable, descartando aquellas lecturas que estén fuera de rango posible, es decir cuando $\Delta{}t \geq \epsilon{}_0$.

\VerbatimFootnotes{}
\footnotetext{%
	Para compilar los archivos de código del Arduino utilice el
	\href{https://www.arduino.cc/en/software}{Arduino IDE} (no olvide cambiar la extensión de los archivos de \texttt{cpp} a \texttt{ino})
	o bien instale las herramientas de compilación por línea de comandos con

	\begin{Verbatim}
# apt install gcc-avr binutils-avr gdb-avr avr-libc avrdude
	\end{Verbatim}

	y posteriormente utilice gcc-avr para compilar los archivos fuente.

	Recuerde que no es posible programar el arduino via \IIC{}.
}