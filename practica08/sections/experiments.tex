% %% %%%%%%%%%%%%%%%%%%%%%%%%%%%%%%%%%%%%%%%%%%%%%%%%%%%%%%%%%%
% experiments.tex
%
% Author:  Mauricio Matamoros
% License: MIT
%
% %% %%%%%%%%%%%%%%%%%%%%%%%%%%%%%%%%%%%%%%%%%%%%%%%%%%%%%%%%%%
%!TEX root = ../practica.tex
%!TEX root = ../references.bib

% CHKTEX-FILE 1
% CHKTEX-FILE 13
% CHKTEX-FILE 46

\section{Experimentos}%
\label{sec:experiments}

\begin{enumerate}
	\item{} [2pts] Implemente el controlador on-off descrito en las \Cref{sec:step-1,sec:appendix1} y verifique que la temperatura dentro de la caja se mantenga en 50°C ± 10°C.
	Reporte el conjunto de valores $threshold$ utilizados por el controlador.%
	\label{enu:ctrl-onoff}

	\item{} [2pts] Con base en la teoría de la \Cref{sec:control-p} y el código presentado en el \Cref{sec:appendix1}, desarrolle un controlador proporcional que mantenga la temperatura dentro de la caja entre 45°C y 55°C.
	Reporte el valor de $K_P$ y el valor medio de potencia suministrado al elemento calefactor (planta).

	\item{} [2pts] Modifique el programa anterior para que el controlador mantenga la temperatura ingresada por línea de comandos con una incertidumbre de ±5°C. El rango de operación es entre 30°C y 100°C.\footnote{El valor mínimo de operación no podrá ser menor a la temperatura ambiente.}%
	\label{enu:ctrl-p}

	\item{} [2pts] Con base en la teoría de la \Cref{sec:control-p,sec:control-i} y el código presentado en el \Cref{sec:appendix1}, desarrolle un controlador proporcional-integral (PI) que mantenga la temperatura dentro de la caja entre 45°C y 55°C.
	Reporte los valores de $K_P$ y $K_I$ así como el valor medio de potencia suministrado al elemento calefactor (planta).%
	\label{enu:ctrl-pi}

	\item{} [2pts] Tomando como condición inicial $T_0$ la temperatura ambiente y $T_f = 50^{o}C$, grafique la respuesta del sistema $y(t) = T(t)$ con cada uno de los controladores implementados en el intervalo $0s \leq t \leq 180s$, reportando las constante $K_P$ utilizada.%
	\label{enu:graph}
	\begin{itemize}[nosep]
		\item Controlador On/Off (1 pt).
		\item Controlador P. Reporte la $K_P$ (1 pt).
	\end{itemize}
\end{enumerate}

\paragraph{Puntos Extra}
\begin{itemize}
	\item{} [+2pts] Con base en la teoría de las \Cref{sec:control-p,sec:control-d} y el código presentado en el \Cref{sec:appendix1}, desarrolle un controlador proporcional-derivativo (PD) que mantenga la temperatura dentro de la caja entre 45°C y 55°C.

	\item{} [+2pts] Con base en la teoría de las \Cref{sec:control-p,sec:control-i,sec:control-d,sec:control-pid} y el código presentado en el \Cref{sec:appendix1}, desarrolle un controlador proporcional-integral-derivativo (PID) que mantenga la temperatura dentro de la caja entre 45°C y 55°C.
	Reporte los valores de $K_P$, $K_I$ y $K_D$ así como el valor medio de potencia suministrado al elemento calefactor (planta).

	\item{} [+2pts] Optimice las constantes del controlador para que el error $\big\lvert e[k]\big\rvert \leq 2^{o}C$.

	\item{} [+2pts] Con base en lo aprendido, modifique el código del punto~\ref{enu:ctrl-p} para que la Raspberry Pi sirva una página web donde se pueda modificar con un control gráfico la temperatura deseada del sistema.

	\item{} [+2pts] Modifique el servidor web implementado para que éste muestre una gráfica en tiempo real con la temperatura registrada por el sensor del sistema.

	\item{} [+5pts] Tomando como base la gráfica realizada en el punto~\ref{enu:graph}, anexe las gráficas de la respuesta del sistema $y(t) = T(t)$ con cada uno de los controladores implementados con $0s \leq t \leq 180s$, reportando las constantes utilizadas:
	\begin{itemize}[nosep]
		\item Controlador PI. Reporte las $K_P$, $K_I$ utilizadas (1 pt).
		\item Controlador PD. Reporte las $K_P$ y $K_D$ utilizadas (2 pts).
		\item Controlador PID. Reporte las $K_P$, $K_I$ y $K_D$ utilizadas (2 pts).
	\end{itemize}
\end{itemize}

