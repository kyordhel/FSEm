% %% %%%%%%%%%%%%%%%%%%%%%%%%%%%%%%%%%%%%%%%%%%%%%%%%%%%%%%%%%%
% intro-control-pi.tex
%
% Author:  Mauricio Matamoros
% License: MIT
%
% %% %%%%%%%%%%%%%%%%%%%%%%%%%%%%%%%%%%%%%%%%%%%%%%%%%%%%%%%%%%
%!TEX root = ../practica.tex
%!TEX root = ../references.bib

% CHKTEX-FILE 1
% CHKTEX-FILE 13
% CHKTEX-FILE 46

\subsubsection{Control integral}%
\label{sec:control-i}
Un control es integral cuando la salida $v(t)$ del controlador es proporcional a la integral del error $e(t)$:

\begin{equation*}
v(t) = K_{I} \int e(t) dt
\end{equation*}

\noindent o en el dominio de la frecuencia:

\begin{equation*}
V(s) = \frac{K_I}{s} E(s)
\end{equation*}

\noindent por lo tanto

\begin{equation}
G_{c}(s) = \frac{V(s)}{E(s)} = \frac{K_I}{s}
\label{eqn:ctrl-i}
\end{equation}

El objetivo de un control integral es el de reducir el error de estado estable, aunque esta corrección viene a costa de sobrecorregir el error, por lo que la respuesta del sistema tiende a oscilatoria o incluso a volverse inestable.
Es por esto que un controlador I suele venir acompañado de otros elementos que ayuden a estabilizar el sistema y a reducir las oscilaciones.

Cuando se combina con un controlador tipo proporcional para formar un control PI \Citep{Hernandez2010}:
\begin{itemize}[noitemsep]
	\item El amortiguamiento se reduce.
	\item El máximo pico de sobreimpulso se incrementa.
	\item Decrece el tiempo de elevación.
	\item Se mejoran los márgenes de ganancia y fase.
	\item El tipo de sistema se incrementa en una unidad.
	\item El error de estado estable mejora por el incremento del tipo de sistema.
\end{itemize}

