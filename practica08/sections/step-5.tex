% %% %%%%%%%%%%%%%%%%%%%%%%%%%%%%%%%%%%%%%%%%%%%%%%%%%%%%%%%%%%
% step-5.tex
%
% Author:  Mauricio Matamoros
% License: MIT
%
% %% %%%%%%%%%%%%%%%%%%%%%%%%%%%%%%%%%%%%%%%%%%%%%%%%%%%%%%%%%%

%!TEX root = ../practica.tex
%!TEX root = ../references.bib

% CHKTEX-FILE 1
% CHKTEX-FILE 13
% CHKTEX-FILE 46

\subsection{Paso 5: Generación de la imagen del S.O. embebido}%
\label{sec:step5}
Aunque muy sencillo, compilar el kernel y los paquetes, y generar el archivo de imagen es un proceso largo que consume bastante tiempo.

Para crear la imagen del sistema embebido, sitúese en el directorio raíz de \emph{buildroot} y ejecute el comando make.

\begin{Verbatim}[gobble=1]
	$ make
\end{Verbatim}

\noindent
Eso es todo.
El proceso de compilación tardará aproximadamente entre 90 minutos y 4 horas dependiendo de la velocidad de la computadora huesped.

\medskip{}
\noindent
Relájese y espere.

\begin{center}
	\includegraphics[width=0.8\columnwidth,height=5cm,keepaspectratio]{img/compiling.png}
\end{center}
