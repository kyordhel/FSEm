% %% %%%%%%%%%%%%%%%%%%%%%%%%%%%%%%%%%%%%%%%%%%%%%%%%%%%%%%%%%%
% step-4.tex
%
% Author:  Mauricio Matamoros
% License: MIT
%
% %% %%%%%%%%%%%%%%%%%%%%%%%%%%%%%%%%%%%%%%%%%%%%%%%%%%%%%%%%%%

%!TEX root = ../practica.tex
%!TEX root = ../references.bib

% CHKTEX-FILE 1
% CHKTEX-FILE 13
% CHKTEX-FILE 46

\subsection{Paso 4: Configuración del \emph{root filesystem}}%
\label{sec:step4}
El siguiente paso consiste en configurar el sistema de archivos raíz o \emph{root filesystem}.
Tras compilar el kernel y los paquetes solicitados, \emph{buildroot} tomará el directorio especificado y sobrepondrá el contenido de éste al sistema de archivos raíz que está siendo generado, un proceso conocido como superposición u \emph{overlay}.

El directorio de superposición puede ser cualquiera en el disco duro.
Sin embargo, se estila colocarlo como un subdirectorio de nombre \texttt{rootfs-overlay} dentro de \texttt{board/<company>/<boardname>}
que, si inspecciona con \texttt{ls}, notará que contiene ya varios archivos de configuración para nuestra tarjeta Raspberry Pi 4 de 64bits.
En nuestro caso la ruta relativa será:

\begin{Verbatim}[gobble=1]
	board/raspberrypi4-64/rootfs-overlay
\end{Verbatim}

Es en este directorio donde colocaremos todos los archivos y directorios que querramos contenga la imagen generada.

Ejecute los siguientes comando para crear el directorio \emph{home} del usuario \emph{pi} creado en el paso anterior y otros directorios que serán necesarios para configuar al sistema embebido.

\begin{Verbatim}[gobble=1]
	mkdir -p board/raspberrypi4-64/rootfs-overlay/home/pi
	mkdir -p board/raspberrypi4-64/rootfs-overlay/etc/init.d
\end{Verbatim}

El sistema creado por \emph{buildroot} es un sistema mínimo.
Esto quiere decir que no se incluye nada que no sea indispensable a menos que sea declarado de forma explícita,
y esto se logra añadiendo los archivos de configuración pertinente al sistema de archivos raíz o \emph{root filesystem} (en adelante rootFS).
Por ejemplo, los controladores para \IIC{} no se cargan de forma predeterminada.

Para especificar el rootFS en \emph{buildroot} ejecute \texttt{make menuconfig} y localice la opción \emph{root filesystem overlay directories} bajo el menú de configuración del sistema (\emph{system configuration}) con el valor del directorio creado anteriormente, es decir:

\begin{Verbatim}[gobble=1]
	board/raspberrypi4-64/rootfs-overlay
\end{Verbatim}

Para indicarle al kernel que debe cargar los controladores para \IIC{} haremos uso de dos archivos.
El primero será el archivo \texttt{S02modules} localizado en \texttt{etc/init.d}, directorio que contiene todos los scripts de arranque del sistema y que son ejecutados en orden.%
\footnote{
	Los controladores tienen muy alta prioridad, de allí que se tengan que cargar primero que otros servicios.
	Es por esto que el nombre del archivo comienza con S02, y será ejecutado después de todos los S00 y S01.
}
\begin{samepage}
Procedemos a crear el archivo, marcarlo como ejecutable y editarlo con los siguientes dos comandos:

\begin{Verbatim}[gobble=1]
	touch board/raspberrypi4-64/rootfs-overlay/etc/init.d/S02modules
	chmod +x board/raspberrypi4-64/rootfs-overlay/etc/init.d/S02modules
	nano board/raspberrypi4-64/rootfs-overlay/etc/init.d/S02modules
\end{Verbatim}
\end{samepage}

\medskip{}

\begin{importantbox}{Advertencia}
	Los archivos que está a punto de modificar son relativos al directorio de la tarjeta dentro de buildroot.

	\medskip\bfseries
	Por ningún motivo modifique los archivos en el directorio \texttt{/etc} de su sistema anfitrión.
\end{importantbox}

\medskip{}
% \newpage

\begin{samepage}
\noindent
A coninuación, agregue el siguiente contenido al archivo:

\noindent
\begin{minipage}{\columnwidth}
\begin{lstlisting}[%
	title={\texttt{rootfs-overlay/etc/init.d/S02modules}},%
	firstnumber=1]
#!/bin/sh
case "${1}" in
    start)
        # Exits if /etc/modules does not exists or lacks valid entries
        [ -r /etc/modules ] && egrep -qv '^($|#)' /etc/modules || exit 0
        # Load listed modules
        while read module args; do
            # Skip comments and empty lines.
            case "$module" in
                ""|"#"*) continue ;;
            esac
            # Try to load each module
            printf "Loading ${module}"
            modprobe ${module} ${args} > /dev/null
            [ $? = 0 ] && echo "OK" || echo "FAIL"
        done < /etc/modules
        exit 0
        ;;

    *)
        echo "Usage: ${0} {start}"
        exit 1
        ;;
esac
\end{lstlisting}
\end{minipage}
\end{samepage}

\texttt{S02modules} es un script de carga de controladores (módulos) simple y genérico que cargará de forma dinámica todos los controladores listados en otro archivo llamado \texttt{modules} en el directorio \texttt{/etc}, si existe.
Procedamos pues a crear dicho archivo con un editor de texto como \texttt{nano}.

\begin{Verbatim}[gobble=1]
	nano board/raspberrypi4-64/rootfs-overlay/etc/modules
\end{Verbatim}

\begin{samepage}
\noindent
A coninuación, agregue el siguiente contenido al archivo:

\noindent
\begin{minipage}{\columnwidth}
\begin{lstlisting}[%
	title={\texttt{rootfs-overlay/etc/modules}},%
	firstnumber=1]
# List of modules to be loaded during startup
i2c-bcm2835
i2c-dev
\end{lstlisting}
\end{minipage}
\end{samepage}

Estos archivos harán que el kernel cargue los controladores para \IIC{} al arranque, ¡pero \texttt{/dev/i2c-1} sólo será accesible por el superusuario \emph{root}!
Para solucionar este inconveniente crearemos otro script de arranque similar a \texttt{S02modules} que cambie los permisos de todos los periféricos \IIC{} cambiando su grupo a i2c al cual el usuario \emph{pi} que creamos en el paso anterior ya tiene acceso.

\begin{samepage}
Procedemos a crear el archivo, marcarlo como ejecutable y editarlo con los siguientes dos comandos:

\begin{Verbatim}[gobble=1]
	touch board/raspberrypi4-64/rootfs-overlay/etc/init.d/S10i2cperms
	chmod +x board/raspberrypi4-64/rootfs-overlay/etc/init.d/S10i2cperms
	nano board/raspberrypi4-64/rootfs-overlay/etc/init.d/S10i2cperms
\end{Verbatim}
\end{samepage}

\begin{samepage}
\noindent
A continuación, agregue el siguiente contenido al archivo:

\noindent
\begin{minipage}{\columnwidth}
\begin{lstlisting}[%
	title={\texttt{rootfs-overlay/etc/init.d/S10i2cperms}},%
	firstnumber=1]
#!/bin/sh
case "${1}" in
    start)
        # Loops over all i2c devices, if any
        for iic in /dev/i2c*; do
            chown root:i2c "${iic}"
            chmod ug+rw "${iic}"
        done
        unset iic
        ;;

    *)
        echo "Usage: ${0} {start}"
        exit 1
        ;;
esac
\end{lstlisting}
\end{minipage}
\end{samepage}

\medskip{}

Por último, crearemos los archivos que permitan la ejecución automática de cualquer script de arranque que nos sea conveniente:

\begin{enumerate}
	\item Copie el archivo anexo \texttt{test.pyc} al directorio del usuario \emph{pi}; es decir a\\
	\texttt{board/raspberrypi4-64/rootfs-overlay/home/pi}.

	\begin{samepage}
	\item Cree un archivo de inicio \texttt{start.sh} en el mismo directorio, y márquelo como ejecutable con los siguientes comandos:
	\begin{Verbatim}[gobble=2]
		touch board/raspberrypi4-64/rootfs-overlay/home/pi/start.sh
		chmod +x board/raspberrypi4-64/rootfs-overlay/home/pi/start.sh
	\end{Verbatim}
	\end{samepage}

	\begin{samepage}
	\item Utilizando editor de texto plano como \texttt{nano} o \texttt{vim}, edite el archivo anterior y agregue el siguiente contenido al archivo:
	\lstinputlisting[%
		language=sh,%
		title={\texttt{rootfs-overlay/home/pi/start.sh}},%
		linerange={1-43}%
	]{src/board/raspberrypi4-64/rootfs-overlay/home/pi/start.sh}
	\end{samepage}

	\begin{samepage}
	\item Cree un archivo de autoarranque o \emph{daemon} \texttt{S99autostart} en el mismo directorio \texttt{etc/init.d/} del \emph{overlay}, y márquelo como ejecutable con los siguientes dos comandos:
	\begin{Verbatim}[gobble=2]
		touch board/raspberrypi4-64/rootfs-overlay/etc/init.d/S99autostart
		chmod +x board/raspberrypi4-64/rootfs-overlay/etc/init.d/S99autostart
	\end{Verbatim}
	\end{samepage}

	\begin{samepage}
	\item Utilizando editor de texto plano como \texttt{nano} o \texttt{vim}, edite el archivo \texttt{S99autostart} y agregue el siguiente contenido al archivo:\\
\noindent
\begin{minipage}{\columnwidth}
\begin{lstlisting}[%
	title={\texttt{rootfs-overlay/etc/init.d/S99autostart}},%
	firstnumber=1]
#!/bin/sh
case "${1}" in
    start)
        # Exits if /home/pi/start.sh does not exist
        [ -f /home/pi/start.sh ] || exit 0
        # Else executes it
        su - pi -c /home/pi/start.sh
        exit 0
        ;;

    *)
        echo "Usage: ${0} {start}"
        exit 1
        ;;
esac
\end{lstlisting}
\end{minipage}
\end{samepage}

\end{enumerate}

\noindent
¡Listo! Ahora su sistema puede ejecutar programas como el usuario \emph{pi} automáticamente al arrancar.
Sólo tiene que modidificar el archivo \texttt{home/pi/start.sh} de acuerdo a sus necesidades.

\bigskip{}

Opcionalmente, en caso de que se haya llevado a cabo la personalización con logotipo~(véase~\Cref{sec:step3}), procederemos a automatizar la solución del inconveniente y borrar la pantalla de forma automática cuando se inicie sesión.
Esto se logra de forma simple añadiendo dicho comando al final del archivo \texttt{.profile} ubicado en cada uno de los directorios de usuario, en nuestro caso \texttt{/root} y \texttt{/home/pi}.

\begin{samepage}
Para resolver el inconveniente el caso del súperusuario \emph{root} ejecute los siguientes comandos:
\begin{Verbatim}[gobble=1]
	$ mkdir -p board/raspberrypi/rootfs-overlay/root
	$ echo "clear" > board/raspberrypi/rootfs-overlay/root/.profile
\end{Verbatim}
\end{samepage}

\begin{samepage}
A continuación, repetimos el procedimiento para la cuenta de usuario \emph{pi}:

\begin{Verbatim}[gobble=1]
	$ echo "clear" > board/raspberrypi/rootfs-overlay/home/pi/.profile
\end{Verbatim}
\end{samepage}

Listo.
Hemos terminado de configurar el sistema de archivos.
