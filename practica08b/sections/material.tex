% %% %%%%%%%%%%%%%%%%%%%%%%%%%%%%%%%%%%%%%%%%%%%%%%%%%%%%%%%%%%
% material.tex
%
% Author:  Mauricio Matamoros
% License: MIT
%
% %% %%%%%%%%%%%%%%%%%%%%%%%%%%%%%%%%%%%%%%%%%%%%%%%%%%%%%%%%%%

%!TEX root = ../practica.tex
%!TEX root = ../references.bib

% CHKTEX-FILE 1
% CHKTEX-FILE 13
% CHKTEX-FILE 46

\section{Material}%
\label{sec:material}
Se aconseja encarecidamente el uso de \textit{git} como programa de control de versiones.

\begin{itemize}[noitemsep]
	\item 1 PC o laptop con:
	\begin{enumerate}[label=\roman*\rpar,noitemsep]
		\item sistema operativo POSIX (preferentemente Linux basado en Debian)
		\item conexión a internet
		\item 20GiB de espacio en disco disponible
	\end{enumerate}
	\item 1 Raspberry Pi
	\item 1 Memoria microSD de 1GB o más
	\item Alambrado de la práctica 7: \enquote{Desplegado de temperatura en un display digital}
	% \item 1 Display de cristal líquido $LCD-16\times2$
	% \item 1 adaptador \IIC{} a $LCD-16\times2$ con módulo expansor PCF8574
	% \item 1 sensor de temperatura DS18B20 en encapsulado TO-92\footnotemark{}
	% \item 2 resistencias de 1k$\Omega$
	% \item 1 resistencia de 10k$\Omega$
	% \item 1 diodo emisor de luz \textsc{Led} rojo
	% \item 1 diodo emisor de luz \textsc{Led} verde
	% \item 1 protoboard o circuito impreso equivalente
	\item 1 fuente de alimentación regulada a 5V y al menos 2 amperios de salida
	\item Cables y conectores varios
\end{itemize}
\footnotetext{En lugar del sensor DS18B20 es posible usar el alambrado de arduino con LM35 de la práctica 6.} %chktex 42

\noindent\textbf{Nota:}
Para esta práctica requerirá los programas desarrollados para la práctica \emph{desplegado de temperatura en un display digital usando la Raspberry Pi}.
