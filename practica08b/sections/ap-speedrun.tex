% %% %%%%%%%%%%%%%%%%%%%%%%%%%%%%%%%%%%%%%%%%%%%%%%%%%%%%%%%%%%
% ap-speedrun.tex
%
% Author:  Mauricio Matamoros
% License: MIT
%
% %% %%%%%%%%%%%%%%%%%%%%%%%%%%%%%%%%%%%%%%%%%%%%%%%%%%%%%%%%%%
% !TEX root = ../practica.tex
% !TEX root = ../references.bib

\section{Speedrun: imágen lista en 15 minutos}%
\label{sec:speedrun}

En esta sección se explica cómo completar la configuración de la imagen del sistema operativo embebido (pasos~2, 3 y~4) en menos de 15 minutos.

\begin{enumerate}
	\item Cree un directorio de trabajo y sítúese allí:
	\begin{Verbatim}[gobble=2]
		$ mkdir ~/myPiLinux
		$ cd ~/myPiLinux
	\end{Verbatim}

	\item Descargue buildroot y descomprímalo.
	Luego cambie al directorio de buildroot:
	\begin{Verbatim}[gobble=2]
		$ wget https://buildroot.org/downloads/buildroot-2025.02.tar.xz -O - | tar -xJf -
		$ cd buildroot-2025.02.tar.xz
	\end{Verbatim}

	\item Copie el archivo de fuentes anexo a esta práctica al directorio de buildroot.

	\item Descomprma el archivo anexo con:
	\begin{Verbatim}[gobble=2]
		$ tar -h -xJf p08-sources.tar.xz
	\end{Verbatim}

	\item Seleccione la plataforma objetivo y ejecute \texttt{menuconfig}:
	\begin{Verbatim}[gobble=2]
		$ make raspberrypi4_64_defconfig
		$ make menuconfig
	\end{Verbatim}

	\item Establezca los valores que se muestran en la \Cref{tbl:menuconfig-changes}.

	\item Finalmente compile la solución con \texttt{make} para generar la imagen.
	\begin{Verbatim}[gobble=2]
		$ make
	\end{Verbatim}
\end{enumerate}

\begin{greenbox}{Paciencia}
El proceso de compilación tardará entre 90 minutos y 4 horas.

\medskip{}

Utilice ese tiempo para estudiar a fondo los pasos de la \Cref{sec:instructions} y preparar adecuadamente los experimentos propuestos de la \Cref{sec:experiments}.
\end{greenbox}