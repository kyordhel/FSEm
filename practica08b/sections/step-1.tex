% %% %%%%%%%%%%%%%%%%%%%%%%%%%%%%%%%%%%%%%%%%%%%%%%%%%%%%%%%%%%
% step-1.tex
%
% Author:  Mauricio Matamoros
% License: MIT
%
% %% %%%%%%%%%%%%%%%%%%%%%%%%%%%%%%%%%%%%%%%%%%%%%%%%%%%%%%%%%%

%!TEX root = ../practica.tex
%!TEX root = ../references.bib

% CHKTEX-FILE 1
% CHKTEX-FILE 13
% CHKTEX-FILE 46

\subsection{Paso 1: Configuración del entorno de desarrollo}%
\label{sec:step1}%
%
\begin{greenbox}{Sistema base: Debian}
	Los pasos de esta sección suponen que la máquina de desarrollo o anfitrión cuenta con un sistema operativo Debian o compatible instalado (ubuntu, mint, etc\dots{}).

	Si usted cuenta con otra distribución como Fedora, Arch, o incluso iOS, es posible realizar la práctica pero tendrá que utilizar los comandos apropiados para su sistema operativo.

	\medskip{}
	\textbf{Nota:} No es posible llevar a cabo la práctica en Windows.
\end{greenbox}

La configuración del entorno de desarrollo es relativamente sencilla.
Para esto basta con instalar los paquetes de compilación cruzada para ARM, así como los paquetes adicionales requeridos por buildroot como \texttt{flex}, \texttt{bison} o \texttt{ncurses}.

\noindent{}
Todo el procedimiento se reduce a una sóla línea:
\begin{Verbatim}[gobble=1]
	# apt install -y git bc bison flex libssl-dev make libc6-dev libncurses5-dev build-essential
	crossbuild-essential-arm64 crossbuild-essential-armhf
\end{Verbatim}

A continuación, cree un nuevo directorio en su \emph{home} y sitúese allí; por ejemplo:

\begin{Verbatim}[gobble=1]
	$ mkdir ~/myPiLinux
	$ cd ~/myPiLinux
\end{Verbatim}
